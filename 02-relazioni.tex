\documentclass{article}

% --- Encoding e lingua ---
\usepackage[utf8]{inputenc}
\usepackage[italian]{babel}

% --- Margini e layout ---
\usepackage{geometry}
\geometry{a4paper, margin=1in}

% --- Font sans-serif ---
\usepackage[scaled]{helvet}
\renewcommand{\familydefault}{\sfdefault}

% --- Matematica ---
\usepackage{amsmath}
\usepackage{amssymb}

% --- Liste personalizzate ---
\usepackage{enumitem}

% --- Hyperlink ---
\usepackage{hyperref}
\hypersetup{
	colorlinks=true,
	linkcolor=white, % Cambiato per visibilità su sfondo nero
	filecolor=magenta,
	urlcolor=white,  % Cambiato per visibilità su sfondo nero
	citecolor=green, % Aggiunto per completezza
	pdftitle={Introduzione ai Modelli Logici},
	pdfpagemode=FullScreen,
}

% --- Colori e sfondo nero ---
\usepackage{xcolor}
\pagecolor{black}
\color{white}

% --- Evidenziazione del codice (richiede -shell-escape) ---
% Compilare con: pdflatex -shell-escape nomefile.tex
\usepackage{minted}
\setminted{
	frame=lines,     % Cornice attorno al codice
	framesep=2mm,
	fontsize=\small,
	breaklines=true, % A capo automatico per linee lunghe
	style=monokai   % Stile di highlighting (assicurati che pygments lo supporti)
}
\usemintedstyle{monokai} % Applica lo stile globalmente per minted


% --- Titolo ---
\title{Introduzione ai Modelli Logici}
\author{Basato sulle slide del Prof. Danilo Montesi}
\date{\today}

\begin{document}
	
	
	\maketitle
	\tableofcontents
	\newpage
	
	\section{Introduzione ai Modelli Logici}
	
	\subsection{Modelli Tradizionali}
	\begin{itemize}
		\item Gerarchico
		\item Reticolare (Network)
		\item \textbf{Relazionale}
	\end{itemize}
	
	\subsection{Modelli Recenti}
	\begin{itemize}
		\item Orientato agli Oggetti (meno comune)
		\item Basato su XML (complementare al relazionale)
	\end{itemize}
	
	\subsection{Caratteristiche dei Modelli}
	\begin{itemize}
		\item Gerarchico e Reticolare: Usano \textbf{riferimenti espliciti} (puntatori) tra i record.
		\item Relazionale: È "\textbf{value-based}". I riferimenti tra i dati in relazioni diverse sono rappresentati tramite i \textbf{valori} stessi all'interno delle tuple.
	\end{itemize}
	Il modello value-based offre indipendenza dalla struttura fisica e facilita la condivisione dei dati.
	
	\section{Il Modello Relazionale}
	
	\subsection{Fondamenti}
	\begin{itemize}
		\item Definito da \textbf{E. F. Codd} nel 1970 per l'indipendenza dei dati.
		\item Implementato nei DBMS reali dal 1981.
		\item Basato sul concetto logico di "\textit{relazione}" dalla teoria degli insiemi, ma con differenze pratiche.
		\item Le relazioni sono rappresentate come \textbf{tabelle}.
	\end{itemize}
	
	\subsection{Relazione Logica vs. Relazione (Modello Relazionale)}
	\begin{itemize}
		\item \textbf{Relazione Logica (Teoria Insiemi):} Sottoinsieme del prodotto cartesiano di domini. È un \textbf{insieme di tuple ordinate}.
		\item \textbf{Relazione (Modello Relazionale):} È un \textbf{insieme} con proprietà specifiche:
		\begin{itemize}
			\item Non c'è \textbf{ordine tra le tuple} (righe).
			\item Le \textbf{tuple sono tutte distinte}.
			\item Gli attributi (colonne) sono identificati da nomi univoci e non dalla posizione (struttura \textbf{non posizionale}).
			\item I valori all'interno di ogni colonna sono \textbf{omogenei} (dallo stesso dominio).
		\end{itemize}
	\end{itemize}
	
	\section{Informazione Parziale (Valori Mancanti)}
	
	\subsection{Il Problema}
	\begin{itemize}
		\item Il modello relazionale ha una struttura rigida.
		\item Non tutte le informazioni sono sempre complete o disponibili.
		\item Come rappresentare l'assenza di un valore?
	\end{itemize}
	
	\subsection{Il Valore NULL}
	\begin{itemize}
		\item Tecnica \textbf{rough but effective} nel modello relazionale.
		\item \textbf{NULL} indica l'\textbf{assenza di un valore} atteso.
		\item \textbf{NULL non è un valore del dominio}.
		\item Una tupla $t$ per un attributo $A$ può avere $t[A]$ pari ad un valore del dominio $\text{dom}(A)$ o NULL.
		\item Esistono concettualmente diversi tipi di NULL (unknown, inexistent, uninformative), ma i DBMS moderni spesso non li distinguono.
		\item Un eccessivo uso di NULL può complicare l'uso dei dati.
	\end{itemize}
	
	\section{Vincoli di Integrità}
	
	\subsection{Scopo}
	\begin{itemize}
		\item Garantire che le istanze del database rappresentino \textbf{informazioni corrette} rispetto al mondo reale (anche se sintatticamente valide).
		\item Supportare la \textbf{qualità dei dati}.
		\item Utili nella \textbf{progettazione} del database e nell'ottimizzazione delle \textbf{query}.
		\item Un vincolo può essere visto come un predicato (funzione booleana) sull'istanza.
	\end{itemize}
	I DBMS supportano ed enforcement alcuni tipi di vincoli; altri devono essere gestiti a livello applicativo.
	
	\subsection{Tipi di Vincoli}
	\begin{itemize}
		\item \textbf{Intra-relazionali:} Riguardano una singola relazione.
		\begin{itemize}
			\item Su \textbf{valori} (o di dominio): Regole sui singoli attributi (es. età $> 0$).
			\item Su \textbf{tuple}: Regole sui valori all'interno di una singola tupla (es. stipendio lordo = netto + deduzioni).
		\end{itemize}
		\item \textbf{Inter-relazionali:} Riguardano più relazioni.
	\end{itemize}
	
	\section{Identificazione delle Tuple (Chiavi)}
	
	\subsection{Concetti di Base}
	\begin{itemize}
		\item Ogni tupla in una relazione deve essere \textbf{identificabile univocamente}.
		\item \textbf{Superkey:} Un insieme di attributi che identifica univocamente le tuple in una relazione. Non esistono due tuple distinte che abbiano gli stessi valori per tutti gli attributi della Superkey.
		\item \textbf{Key (Chiave):} Una Superkey \textbf{minimale}. Rimuovendo qualsiasi attributo dalla Key, l'insieme rimanente non è più una Superkey.
		\item Ogni relazione possiede \textbf{almeno una Key} (l'insieme di tutti gli attributi è sempre una Superkey).
	\end{itemize}
	
	\subsection{Primary Key (PK)}
	\begin{itemize}
		\item Una delle Key di una relazione scelta per essere l'\textbf{identificatore principale} delle tuple.
		\item Le \textbf{Primary Key non possono contenere valori NULL}.
		\item Spesso indicata sottolineando gli attributi.
	\end{itemize}
	
	\subsection{Importanza delle Chiavi}
	\begin{itemize}
		\item Garantiscono l'\textbf{accessibilità} di ogni singola tupla.
		\item Sono cruciali per \textbf{correlare} le tuple tra relazioni diverse, fondamento del modello value-based.
		\item La presenza di NULL nelle Keys compromette l'identificazione e la creazione di riferimenti.
	\end{itemize}
	
	\section{Integrità Referenziale (Foreign Key)}
	
	\subsection{Concetto}
	\begin{itemize}
		\item Le correlazioni basate sui valori tra relazioni (tipicamente sulle PK) devono essere \textbf{coerenti}.
		\item Assicura che i riferimenti tra tabelle siano validi.
	\end{itemize}
	
	\subsection{Foreign Key (FK)}
	\begin{itemize}
		\item Un vincolo di integrità inter-relazionale.
		\item Un insieme di attributi \textbf{FK} in una relazione $R_1$ \textbf{fa riferimento} alla \textbf{Primary Key PK} di un'altra relazione $R_2$.
		\item \textbf{Regole per una FK:}
		\begin{enumerate}
			\item Gli attributi della FK devono avere lo stesso \textbf{dominio} degli attributi della PK a cui si riferiscono.
			\item Un valore negli attributi FK di una tupla in $R_1$ deve essere:
			\begin{itemize}
				\item \textbf{NULL}
				\item \textbf{OPPURE} deve \textbf{esistere come valore nella PK} di \textbf{qualche tupla} nella relazione $R_2$ (referenziata).
			\end{itemize}
		\end{enumerate}
	\end{itemize}
	
	\subsection{Gestione delle Violazioni (su DELETE/UPDATE)}
	Quando un'operazione sulla relazione referenziata $R_2$ causerebbe una violazione (es. cancellazione di una tupla in $R_2$ referenziata da $R_1$), il DBMS può agire:
	\begin{itemize}
		\item \textbf{Restrict (Comportamento Standard):} Impedisce l'operazione su $R_2$.
		\item \textbf{Cascade:} Estende l'operazione alle tuple referenzianti in $R_1$ (es. cancella le tuple in $R_1$).
		\item \textbf{Set NULL:} Imposta i valori della FK a NULL nelle tuple referenzianti in $R_1$.
		\item \textbf{Set Default:} Imposta i valori della FK a un valore predefinito nelle tuple referenzianti in $R_1$.
	\end{itemize}
	
	Per le Foreign Key composte da più attributi, l'\textbf{ordine} degli attributi è rilevante per il mapping con la Primary Key referenziata.
	
	
\end{document}