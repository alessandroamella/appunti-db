\documentclass{article}

% --- Encoding e lingua ---
\usepackage[utf8]{inputenc}
\usepackage[italian]{babel}

% --- Margini e layout ---
\usepackage{geometry}
\geometry{a4paper, margin=1in}

% --- Font sans-serif ---
\usepackage[scaled]{helvet}
\renewcommand{\familydefault}{\sfdefault}
\usepackage[T1]{fontenc}

% --- Matematica ---
\usepackage{amsmath}
\usepackage{amssymb}

% --- Liste personalizzate ---
\usepackage{enumitem}
\setlist{nosep} % Rimuove spazio extra tra gli item delle liste

% --- Immagini ---
\usepackage{float}
% \usepackage{tikz} % Non utilizzato attivamente in questo output, ma presente nel template
% \usetikzlibrary{shapes.geometric, positioning, calc}

% --- Hyperlink ---
\usepackage{hyperref}
\hypersetup{
	colorlinks=true,
	linkcolor=white,     % Cambiato per migliore visibilità su sfondo nero
	filecolor=magenta,
	urlcolor=blue,
	citecolor=green,    % Aggiunto per coerenza
	pdftitle={Appunti sul Modello Relazionale dei Dati},
	pdfpagemode=FullScreen,
}

% --- Colori e sfondo nero ---
\usepackage{xcolor}
\pagecolor{black}
\color{white}

% --- Evidenziazione del codice (richiede -shell-escape) ---
% Compilare con: pdflatex -shell-escape nomefile.tex
\usepackage{minted}
\setminted{
	frame=lines,     % Cornice attorno al codice
	framesep=2mm,
	fontsize=\small,
	breaklines=true, % A capo automatico per linee lunghe
	style=monokai,   % Stile di highlighting
}

% --- Titolo ---
\title{Appunti sul Modello Relazionale dei Dati}
\author{Basato sulle slide del Prof. Danilo Montesi}
\date{\today}

\begin{document}
	
	\maketitle
	\tableofcontents
	\newpage
	
	\section{Introduzione ai Modelli Logici}
	I database utilizzano diversi approcci per organizzare logicamente i dati.
	\begin{itemize}
		\item \textbf{Modelli Tradizionali:}
		\begin{itemize}
			\item \textbf{Gerarchico:} Struttura ad albero (es. file system). Ogni "figlio" ha un solo "genitore". Navigazione rigida.
			\item \textbf{Di Rete (Network):} Evoluzione del gerarchico, permette a un "figlio" di avere più "genitori". Più flessibile ma complesso.
			\item \textbf{Relazionale:} Il modello dominante. Dati organizzati in tabelle.
		\end{itemize}
		\item \textbf{Modelli Più Recenti:}
		\begin{itemize}
			\item \textbf{Object Oriented:} Dati visti come oggetti con proprietà e metodi. Poco comune per DBMS generici.
			\item \textbf{XML:} Per dati semi-strutturati, spesso complementare al relazionale (es. salvare configurazioni complesse in una cella).
		\end{itemize}
	\end{itemize}
	
	\textbf{Caratteristica distintiva:}
	\begin{itemize}
		\item I modelli Gerarchico e Network usano \textbf{riferimenti espliciti (puntatori)} tra record.
		\item Il modello \textbf{Relazionale è "value-based"}: i collegamenti avvengono tramite valori condivisi (es. \texttt{userID} in \texttt{Posts} che corrisponde a \texttt{id} in \texttt{Users}).
	\end{itemize}
	
	\section{Il Modello Relazionale: Fondamenti}
	\begin{itemize}
		\item \textbf{Definito da E. F. Codd nel 1970:} Obiettivo principale era l'\textbf{indipendenza dei dati}, separando la rappresentazione logica dalla memorizzazione fisica.
		\item \textit{Prisma/ORM Insight:} Un ORM astrae ulteriormente, ma il DBMS relazionale sottostante già opera questa separazione.
		\item \textbf{Implementato nei DBMS reali dal 1981.}
		\item \textbf{Basato sulla definizione logica di "relazione"} (dalla teoria degli insiemi), con differenze pratiche.
		\item \textbf{Le relazioni sono rappresentate tramite tabelle.}
	\end{itemize}
	
	\textbf{Terminologia Importante:}
	\begin{itemize}
		\item \textbf{Relazione Logica (Teoria degli Insiemi):} Un sottoinsieme del prodotto cartesiano di due o più insiemi (domini).
		\item \textbf{Relazione (Modello Relazionale):} Una tabella con righe e colonne.
		\item \textbf{Relationship (Modello Entità-Relazione - ER):} Descrive un legame specifico tra entità (es. "uno studente \textit{si iscrive a} un corso").
	\end{itemize}
	
	\section{Relazioni Logiche vs. Tabelle}
	\subsection{Relazione Logica (Matematica)}
	\begin{itemize}
		\item Dati due insiemi (domini) $D_1 = \{a,b\}$ e $D_2 = \{x,y,z\}$.
		\item Il \textbf{prodotto cartesiano} $D_1 \times D_2$ è l'insieme di tutte le coppie ordinate possibili: $\{(a,x), (a,y), (a,z), (b,x), (b,y), (b,z)\}$.
		\item Una \textbf{relazione $r$} è un \textit{sottoinsieme} di questo prodotto cartesiano, es: $r = \{(a,x), (a,z), (b,y)\}$.
		\item Questo si estende a $n$ domini $D_1, \dots, D_n$. Una tupla è $(d_1, \dots, d_n)$.
		\item \textbf{Proprietà di una relazione logica (come insieme):}
		\begin{enumerate}
			\item \textbf{Nessun ordine tra le tuple:} L'ordine delle righe non ha significato.
			\item \textbf{Le tuple sono tutte distinte:} Non ci possono essere righe duplicate.
			\item \textbf{Ogni n-upla è ordinata:} L'ordine dei valori \textit{all'interno} di una tupla (cioè, l'ordine delle colonne) è significativo.
		\end{enumerate}
	\end{itemize}
	\textbf{Esempio Posizionale:}
	\texttt{Matches $\subseteq$ string $\times$ string $\times$ int $\times$ int} \\
	Una tupla: \texttt{(Barca, Bayern, 3, 1)}
	Qui il significato dipende dalla \textit{posizione}: (SquadraCasa, SquadraOspite, GolCasa, GolOspite).
	
	\section{Strutture Dati Non Posizionali}
	Nelle tabelle reali, non ci affidiamo solo alla posizione delle colonne.
	\begin{itemize}
		\item \textbf{Ogni colonna ha un nome univoco (attributo)} associato a un dominio (tipo di dato). L'attributo definisce il "ruolo" del dominio.
		\item Esempio: \texttt{Home (string)}, \texttt{Away (string)}, \texttt{GoalsH (int)}, \texttt{GoalsA (int)}.
		\item \textbf{La struttura dati diventa non posizionale:} L'ordine specifico delle colonne nella definizione della tabella è irrilevante per la logica.
		\item \textit{SQL Insight:} \texttt{SELECT Home, Away FROM Matches} e \texttt{SELECT Away, Home FROM Matches} accedono agli stessi dati; cambia solo la presentazione.
	\end{itemize}
	
	\textbf{Una tabella rappresenta una relazione se:}
	\begin{enumerate}
		\item Ogni riga può assumere qualsiasi posizione.
		\item Ogni colonna può assumere qualsiasi posizione (identificate dal nome).
		\item Tutte le righe sono differenti.
		\item Tutti i nomi delle colonne (intestazioni) sono differenti.
		\item I valori all'interno di una colonna sono omogenei (stesso tipo di dato).
	\end{enumerate}
	
	\section{Il Modello "Value-Based" (Basato su Valori)}
	Questo è un concetto chiave.
	\begin{itemize}
		\item I riferimenti (collegamenti) tra dati in relazioni (tabelle) diverse sono rappresentati tramite \textbf{valori} nelle tuple (righe).
		\item \textbf{Esempio Pratico:}
		\begin{itemize}
			\item Tabella \texttt{STUDENT (Number, Surname, Name, ...)}
			\item Tabella \texttt{EXAM (Student\_ID, Lecture\_ID, Grade, ...)}
			\item Per collegare un esame a uno studente, \texttt{EXAM.Student\_ID} conterrà un valore che corrisponde a un valore in \texttt{STUDENT.Number}.
		\end{itemize}
		\item \textit{SQL Insight:} Questo è come funzionano le \texttt{FOREIGN KEY} e le clausole \texttt{JOIN ... ON table1.column = table2.column}.
	\end{itemize}
	
	\textbf{Vantaggi della struttura "value-based":}
	\begin{enumerate}
		\item Indipendenza dalla struttura fisica dei dati.
		\item Memorizzazione solo dei dati rilevanti.
		\item Utente e programmatore vedono gli stessi dati.
		\item Dati facilmente condivisibili tra ambienti diversi.
		\item I collegamenti basati su valori possono essere "navigati" in entrambe le direzioni.
	\end{enumerate}
	
	\section{Definizioni Chiave: Schema, Tupla, Istanza}
	\begin{itemize}
		\item \textbf{Schema di una Relazione (Tabella):}
		\begin{itemize}
			\item Nome della relazione seguito dall'elenco dei suoi attributi (colonne).
			\item Notazione: $R(A_1, A_2, \dots, A_n)$
			\item Esempio: \texttt{STUDENTS (Number, Surname, Name, YearOfBirth)}
			\item \textit{SQL Insight:} Corrisponde a \texttt{CREATE TABLE STUDENTS (...)}.
		\end{itemize}
		\item \textbf{Schema di un Database:}
		\begin{itemize}
			\item Insieme degli schemi di tutte le relazioni nel database.
			\item Esempio: $R = \{\text{STUDENTS}(\dots), \text{EXAMS}(\dots), \text{LECTURES}(\dots)\}$
			\item \textit{Prisma/ORM Insight:} Il tuo file \texttt{schema.prisma} definisce lo schema.
		\end{itemize}
		\item \textbf{Tupla (Riga):}
		\begin{itemize}
			\item Una tupla $t$ su un insieme di attributi $X$ è una mappatura che associa a ogni attributo $A \in X$ un valore dal dominio di $A$.
			\item $t[A]$ esprime il valore della tupla $t$ per l'attributo $A$.
			\item Esempio: Se $t = (6554, \text{Rossi, Mario}, 1978/12/05)$, allora $t[\text{Name}] = \text{Mario}$.
		\end{itemize}
		\item \textbf{Istanza di una Relazione (Contenuto di una Tabella):}
		\begin{itemize}
			\item Insieme \textit{finito} di tuple che soddisfano lo schema. Dati attuali in un dato momento.
		\end{itemize}
		\item \textbf{Istanza di un Database Relazionale:}
		\begin{itemize}
			\item Insieme di istanze di relazione, una per ogni schema. Tutti i dati in tutte le tabelle.
		\end{itemize}
	\end{itemize}
	\textbf{Schema vs. Istanza:} Lo schema è la "definizione" (statico), l'istanza sono i "dati reali" (dinamica).
	
	\section{Gestione di Strutture Dati Annidate}
	Il modello relazionale classico (Prima Forma Normale - 1NF) richiede valori \textbf{atomici}.
	\begin{itemize}
		\item \textbf{Esempio:} Una ricevuta con una \textit{lista} di prodotti.
		\begin{minted}{text}
Ricevuta 1235, Data 2002/10/12, Totale 39.20
Items:
- 3 x Coperto @ 3.00
- 2 x Antipasto @ 6.20
- ...
		\end{minted}
		\item \textbf{Rappresentazione relazionale (unnesting):} Tabelle separate collegate da chiavi.
		\begin{enumerate}
			\item Tabella \texttt{RECEIPT (Number, Date, Total)}
			\item Tabella \texttt{COURSE\_ITEM (ReceiptNumber, Qty, Description, Price)}
		\end{enumerate}
		\texttt{ReceiptNumber} in \texttt{COURSE\_ITEM} è una chiave esterna.
		\item \textbf{Considerazioni sull' "unnesting":}
		\begin{itemize}
			\item \textbf{Ordine delle righe:} Aggiungere colonna \texttt{LineNumber} o \texttt{ItemOrder}.
			\item \textbf{Righe ripetute:} \texttt{LineNumber} diventa essenziale per distinguerle.
		\end{itemize}
	\end{itemize}
	
	\section{Informazioni Parziali e Valori NULL}
	Spesso i dati sono incompleti.
	\begin{itemize}
		\item \textbf{Soluzioni errate per dati mancanti:} Usare valori specifici (0, "", "99").
		\begin{itemize}
			\item Problemi: Valore "non usato" potrebbe non esistere, o diventare utile; complessità applicativa.
		\end{itemize}
		\item \textbf{Soluzione del Modello Relazionale: Valore \texttt{NULL}}
		\begin{itemize}
			\item \texttt{NULL} indica l'\textbf{assenza di un valore}. Non è 0, non è stringa vuota.
			\item \texttt{NULL} \textbf{non appartiene al dominio} dell'attributo.
			\item Per ogni attributo $A$, $t[A]$ può mappare a un valore in $\text{dom}(A)$ o a \texttt{NULL}.
			\item Si possono definire vincoli per non ammettere \texttt{NULL} (es. \texttt{NOT NULL}).
		\end{itemize}
		\item \textbf{Diversi significati di \texttt{NULL} (concettuali):}
		\begin{itemize}
			\item Valore sconosciuto.
			\item Valore inesistente/non applicabile.
			\item Valore non informativo.
		\end{itemize}
		\item \textit{SQL Insight:} Si interroga con \texttt{IS NULL} e \texttt{IS NOT NULL}.
		\item \textbf{Troppi \texttt{NULL}:} Possibile segno di progettazione non ottimale.
	\end{itemize}
	
	\section{Vincoli di Integrità}
	Regole che i dati devono rispettare per garantire correttezza e consistenza.
	\begin{itemize}
		\item Un vincolo è una \textbf{funzione booleana (predicato)}: per ogni istanza, è vero o falso.
		\item \textbf{Perché usare i vincoli?}
		\begin{enumerate}
			\item Descrizione accurata dello scenario reale.
			\item Supportano la "qualità dei dati".
			\item Utili nella progettazione del Database.
			\item Usati dal DBMS per l'ottimizzazione delle query.
		\end{enumerate}
		\item \textbf{Supporto dei DBMS:}
		\begin{itemize}
			\item Molti tipi supportati nativamente (\texttt{NOT NULL, UNIQUE, PRIMARY KEY, FOREIGN KEY, CHECK}).
			\item Vincoli non supportati devono essere implementati a livello applicativo.
		\end{itemize}
	\end{itemize}
	
	\subsection{Tipi di Vincoli}
	\begin{enumerate}
		\item \textbf{Vincoli Intra-relazionali (su una singola tabella):}
		\begin{itemize}
			\item \textbf{Sui valori (o Vincoli di Dominio):} Valori ammissibili per una colonna.
			\item Esempio: \texttt{Grade} tra 18 e 30.
			\item \textit{SQL Insight:} \texttt{CHECK (Grade >= 18 AND Grade <= 30)}.
			\item \textbf{Sulle tuple:} Valori di più colonne \textit{nella stessa riga}.
			\item Esempio: \texttt{GrossPay = Deductions + Net}.
			\item \textit{SQL Insight:} \texttt{CHECK (GrossPay = Deductions + Net)}.
		\end{itemize}
		\item \textbf{Vincoli Inter-relazionali (tra più tabelle):}
		\begin{itemize}
			\item Integrità referenziale (chiavi esterne).
			\item Esempio: \texttt{IDStudente} in \texttt{ESAMI} deve esistere in \texttt{STUDENTI}.
			\item \textit{SQL Insight:} \texttt{FOREIGN KEY}.
		\end{itemize}
	\end{enumerate}
	
	\subsection{Vincoli di Tupla (e di Dominio)}
	\begin{itemize}
		\item \textbf{Vincoli di Tupla:} Regole sui valori di ogni tupla, indipendentemente dalle altre.
		\item \textbf{Vincoli di Dominio (caso specifico):} Coinvolgono un singolo attributo.
		\begin{itemize}
			\item Sintassi: Espressioni booleane che confrontano valori del dominio o espressioni aritmetiche.
		\end{itemize}
	\end{itemize}
	
	\subsection{Chiavi (Superchiavi, Chiavi Candidate, Chiave Primaria)}
	Fondamentali per identificare univocamente le tuple e stabilire relazioni.
	\begin{itemize}
		\item \textbf{Superchiave (Superkey):}
		\begin{itemize}
			\item Insieme di attributi $K$ tali che non esistono due tuple distinte $t_1, t_2$ con $t_1[K] = t_2[K]$.
			\item I valori combinati degli attributi in $K$ sono unici per ogni riga.
			\item Esempio: In \texttt{STUDENTS (Number, ...)}, \texttt{\{Number\}} è superchiave. Anche \texttt{\{Number, Surname\}} lo è. L'insieme di \textit{tutti} gli attributi è sempre una superchiave.
		\end{itemize}
		\item \textbf{Chiave (o Chiave Candidata - Candidate Key):}
		\begin{itemize}
			\item Una superchiave \textbf{minimale} (rimuovendo un attributo, cessa di essere superchiave).
			\item Una relazione può avere più chiavi candidate.
		\end{itemize}
		\item \textbf{Vincoli, Schema e Istanze:}
		\begin{itemize}
			\item Le chiavi sono proprietà dello \textbf{schema}, non dedotte da una particolare \textbf{istanza}.
		\end{itemize}
		\item \textbf{Esistenza delle Chiavi:}
		\begin{itemize}
			\item Ogni relazione DEVE avere almeno una chiave.
		\end{itemize}
		\item \textbf{Importanza delle Chiavi:}
		\begin{enumerate}
			\item Garantiscono identificazione univoca e accessibilità.
			\item Permettono di correlare tuple tra relazioni (modello "value-based").
		\end{enumerate}
		\item \textbf{Chiavi e Valori \texttt{NULL}:}
		\begin{itemize}
			\item Attributi parte di una chiave candidata dovrebbero essere \texttt{NOT NULL}.
		\end{itemize}
		\item \textbf{Chiave Primaria (Primary Key - PK):}
		\begin{itemize}
			\item Una chiave candidata scelta come meccanismo principale di identificazione.
			\item \textbf{NON PUÒ contenere valori \texttt{NULL}}.
			\item Ogni relazione ha al massimo una PK. Spesso sottolineata.
			\item \textit{SQL Insight:} \texttt{PRIMARY KEY (attribute\_list)} implica \texttt{UNIQUE} e \texttt{NOT NULL}.
		\end{itemize}
	\end{itemize}
	
	\subsection{Integrità Referenziale (Chiavi Esterne e Azioni Compensative)}
	Garantisce coerenza dei collegamenti tra tabelle.
	\begin{itemize}
		\item \textbf{Vincolo di Integrità Referenziale (o Chiave Esterna - Foreign Key - FK):}
		\begin{itemize}
			\item Un insieme di attributi $X$ in $R_1$ (tabella referenziante) è una FK che referenzia la PK (o una chiave candidata univoca) di $R_2$ (tabella referenziata) se:
			\begin{enumerate}
				\item Gli attributi $X$ in $R_1$ e la PK di $R_2$ hanno domini compatibili.
				\item Per ogni tupla in $R_1$, i valori di $X$ devono:
				\begin{itemize}
					\item Essere \texttt{NULL} (se permesso).
					\item Oppure, corrispondere a un valore esistente nella PK di una tupla in $R_2$.
				\end{itemize}
			\end{enumerate}
		\end{itemize}
		\item Esempio:
		\begin{minted}{sql}
-- Tabella POLICEMAN
-- ID (PK), Surname, Name

-- Tabella INFRINGEMENT
-- Code (PK), Date, Policeman_ID (FK -> POLICEMAN.ID), ...
CREATE TABLE INFRINGEMENT (
Code INT PRIMARY KEY,
EventDate DATE,
Policeman_ID INT,
-- ... altre colonne ...
FOREIGN KEY (Policeman_ID) REFERENCES POLICEMAN(ID)
);
		\end{minted}
		\item \textbf{Chiavi Esterne e \texttt{NULL}:}
		\begin{itemize}
			\item Una FK può contenere \texttt{NULL} se la relazione è opzionale.
			\item Esempio: \texttt{EMPLOYEE (ID, Name, Project\_Code)}. Se \texttt{Project\_Code} è FK, un impiegato può avere \texttt{Project\_Code = NULL}.
		\end{itemize}
		\item \textbf{Azioni Compensative (se si viola l'integrità referenziale):}
		\begin{itemize}
			\item Azioni su DELETE/UPDATE sulla tabella referenziata ($R_2$):
			\begin{enumerate}
				\item \textbf{\texttt{RESTRICT} (o \texttt{NO ACTION} - default):} Operazione rifiutata.
				\item \textbf{\texttt{CASCADE}:}
				\begin{itemize}
					\item \texttt{ON DELETE CASCADE}: Elimina righe referenzianti in $R_1$.
					\item \texttt{ON UPDATE CASCADE}: Aggiorna valori FK in $R_1$.
				\end{itemize}
				\item \textbf{\texttt{SET NULL}:}
				\begin{itemize}
					\item \texttt{ON DELETE SET NULL}: Imposta FK in $R_1$ a \texttt{NULL}.
					\item \texttt{ON UPDATE SET NULL}: (simile).
				\end{itemize}
				\item \textbf{\texttt{SET DEFAULT}:}
				\begin{itemize}
					\item \texttt{ON DELETE SET DEFAULT}: Imposta FK in $R_1$ al valore di default.
				\end{itemize}
			\end{enumerate}
		\end{itemize}
		\item \textit{SQL Insight:}
		\begin{minted}{sql}
FOREIGN KEY (Project_Code) REFERENCES PROJECT(Code)
ON DELETE SET NULL
ON UPDATE CASCADE;
		\end{minted}
		\item \textbf{Vincoli su Attributi Multipli (Chiavi Composte):}
		\begin{itemize}
			\item PK o FK possono essere composte da più attributi.
			\item L'ordine degli attributi nella definizione della FK deve corrispondere a quello della PK referenziata.
		\end{itemize}
	\end{itemize}
	
\end{document}
