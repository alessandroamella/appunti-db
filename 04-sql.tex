\documentclass{article}

% --- Encoding e lingua ---
\usepackage[utf8]{inputenc}
\usepackage[italian]{babel}

% --- Margini e layout ---
\usepackage{geometry}
\geometry{a4paper, margin=1in}

% --- Font sans-serif ---
\usepackage[scaled]{helvet}
\renewcommand{\familydefault}{\sfdefault}

% --- Matematica ---
\usepackage{amsmath}
\usepackage{amssymb}

% --- Liste personalizzate ---
\usepackage{enumitem}

% --- Hyperlink ---
\usepackage{hyperref}
\hypersetup{
	colorlinks=true,
	linkcolor=white, % Modificato per visibilità su sfondo nero
	filecolor=magenta,
	urlcolor=blue,
	pdftitle={Appunti su Basic SQL},
	pdfpagemode=FullScreen,
}

% --- Colori e sfondo nero ---
\usepackage{xcolor}
\pagecolor{black}
\color{white}

% --- Evidenziazione del codice (richiede -shell-escape) ---
% Compilare con: pdflatex -shell-escape nomefile.tex
\usepackage{minted}
\setminted{
	frame=lines,     % Cornice attorno al codice
	framesep=2mm,
	fontsize=\small,
	breaklines=true, % A capo automatico per linee lunghe
	style=monokai,   % Stile di highlighting (assicurati che pygments lo supporti)
}
\newminted[sqlcode]{sql}{} % Comando personalizzato per SQL più breve


% --- Comandi personalizzati per algebra relazionale (meno usati qui, ma presenti nel template) ---
\newcommand{\Rel}[1]{\textit{#1}} % Per i nomi delle relazioni
\newcommand{\Attr}[1]{\textsf{#1}} % Per i nomi degli attributi

\newcommand{\myunion}{\cup}
\newcommand{\myintersection}{\cap}
\newcommand{\mydifference}{-}
\newcommand{\myrename}[2]{\rho_{#1}(#2)}
\newcommand{\myselectop}[2]{\sigma_{#1}(#2)}
\newcommand{\myproject}[2]{\pi_{#1}(#2)}
\newcommand{\mycartesian}{\times}
\newcommand{\mynaturaljoin}{\bowtie}
\newcommand{\mythetajoin}[3]{#1 \bowtie_{#2} #3}

% --- Comandi personalizzati per logica ---
\newcommand{\mylandop}{\wedge}
\newcommand{\myvel}{\vee}
\newcommand{\mynegop}{\neg}
\newcommand{\myforallop}{\forall}
\newcommand{\myexistsop}{\exists}

% --- Join esterni (outer join) ---
\def\ojoin{\setbox0=\hbox{$\mynaturaljoin$}%
	\rule[-.02ex]{.25em}{.4pt}\llap{\rule[\ht0]{.25em}{.4pt}}}
\newcommand{\myleftouterjoin}{\mathbin{\ojoin\mkern-5.8mu\mynaturaljoin}}
\newcommand{\myrightouterjoin}{\mathbin{\mynaturaljoin\mkern-5.8mu\ojoin}}
\newcommand{\myfullouterjoin}{\mathbin{\ojoin\mkern-5.8mu\mynaturaljoin\mkern-5.8mu\ojoin}}


% --- Titolo ---
\title{Appunti su Basic SQL}
\author{Basato sulle slide del Prof. Danilo Montesi}
\date{\today}

\begin{document}
	
	\maketitle
	\tableofcontents
	\newpage
	
	\section{Introduzione a SQL}
	
	\subsection{Cos'è SQL}
	\begin{itemize}
		\item Acronimo di ``Structured Query Language'', oggi considerato un ``nome proprio''.
	\end{itemize}
	
	\subsection{Caratteristiche Principali}
	\begin{itemize}
		\item Implementa sia \textbf{DDL (Data Definition Language)}: comandi per definire la struttura del database (tabelle, schemi, indici, ecc.).
		\item Implementa sia \textbf{DML (Data Manipulation Language)}: comandi per interrogare e modificare i dati.
	\end{itemize}
	
	\subsection{Standard vs. Dialetti}
	\begin{itemize}
		\item Esiste uno standard ISO, ma ogni DBMS (PostgreSQL, MySQL, SQL Server, Oracle, SQLite) ha le sue piccole variazioni ed estensioni grammaticali.
		\item \textit{Esempio Pratico:} La sintassi per l'auto-incremento di un ID può variare (\texttt{SERIAL} in PostgreSQL, \texttt{AUTO\_INCREMENT} in MySQL).
	\end{itemize}
	
	\subsection{Storia}
	\begin{itemize}
		\item Predecessore: SEQUEL (1974).
		\item Prime implementazioni: SQL/DS e Oracle (1981).
		\item Standard ``de facto'' dal 1983, con molte evoluzioni (SQL-86, SQL-89, SQL-92, SQL:1999, ecc.) che hanno introdotto:
		\begin{itemize}
			\item Integrità referenziale (SQL-89)
			\item Funzioni come \texttt{COALESCE}, \texttt{NULLIF}, \texttt{CASE} (SQL-92)
			\item Concetti object-relational, trigger, funzioni esterne (SQL:1999)
			\item Supporto per Java e XML (SQL:2003, SQL:2006)
		\end{itemize}
	\end{itemize}
	
	\section{DDL (Data Definition Language) - Definire la Struttura}
	
	\subsection{Database e Schemi}
	\begin{itemize}
		\item \texttt{CREATE DATABASE db\_name;}
		\begin{itemize}
			\item Crea un nuovo database, che è un contenitore per tabelle, viste, trigger, ecc.
			\item \textit{Esempio Pratico (SQLite):} Quando esegui \texttt{sqlite3 miadatabase.db}, stai creando un file che funge da database.
			\item \textit{Nota:} In alcuni DBMS come MySQL, \texttt{CREATE SCHEMA} è un sinonimo di \texttt{CREATE DATABASE}.
		\end{itemize}
		\item \texttt{CREATE SCHEMA schema\_name [AUTHORIZATION 'user\_name'];}
		\begin{itemize}
			\item Uno schema è uno spazio dei nomi all'interno di un database. Serve a organizzare gli oggetti del database.
			\item L'utente che esegue il comando diventa il proprietario dello schema, a meno che non sia specificato con \texttt{AUTHORIZATION}.
			\item \textit{Esempio Pratico (PostgreSQL):} Spesso si usa lo schema \texttt{public} di default, ma potresti creare schemi come \texttt{accounting}, \texttt{sales} per separare logicamente le tabelle.
		\end{itemize}
	\end{itemize}
	
	\subsection{Tabelle}
	\begin{itemize}
		\item Sintassi base:
		\begin{minted}{sql}
			CREATE TABLE table_name (
			colonna1 TIPO_DATI [vincoli],
			colonna2 TIPO_DATI [vincoli],
			...
			);
		\end{minted}
		\item Definisce una nuova tabella (relazione) con le sue colonne (attributi), i tipi di dato per ciascuna colonna e i vincoli iniziali.
		\item \textit{Esempio Pratico (corrispettivo User con Prisma):}
		\begin{minted}[bgcolor=black!70]{text}
			// Prisma Schema
			model User {
				id        Int      @id @default(autoincrement())
				email     String   @unique
				name      String?
			}
		\end{minted}
		Equivalente a:
		\begin{minted}{sql}
			-- SQL (es. PostgreSQL)
			CREATE TABLE User (
			id SERIAL PRIMARY KEY,
			email VARCHAR(255) UNIQUE NOT NULL,
			name VARCHAR(255)
			);
		\end{minted}
		\item Esempio dalla slide:
		\begin{minted}{sql}
			CREATE TABLE EMPLOYEE (
			Number CHARACTER(6) PRIMARY KEY,
			Name CHARACTER(20) NOT NULL,
			Surname CHARACTER(20) NOT NULL,
			Dept CHARACTER(15),
			Wage NUMERIC(9) DEFAULT 0,
			FOREIGN KEY(Dept) REFERENCES DEPARTMENT(Dept)
			);
		\end{minted}
	\end{itemize}
	
	\subsection{Tipi di Dati}
	\subsubsection{Tipi Base}
	\begin{itemize}
		\item \texttt{CHARACTER(n)}, \texttt{VARCHAR(n)}: Stringhe di caratteri a lunghezza fissa o variabile.
		\item \texttt{NUMERIC(p,s)}, \texttt{INTEGER}, \texttt{SMALLINT}, \texttt{DECIMAL}: Numeri interi o decimali.
		\item \texttt{DATE}, \texttt{TIME}, \texttt{TIMESTAMP}, \texttt{INTERVAL}: Per date e orari.
		\item \texttt{BOOLEAN}: Valori vero/falso.
		\item \texttt{BLOB}, \texttt{CLOB}: (Binary/Character Large Object) Per grandi quantità di dati binari o testuali.
	\end{itemize}
	\subsubsection{Tipi Personalizzati (Domini)}
	\begin{itemize}
		\item \texttt{CREATE DOMAIN domain\_name AS tipo\_base [DEFAULT valore\_default] [CHECK (condizione)];}
		\item Permette di definire un tipo di dato riutilizzabile con vincoli e valori di default specifici.
		\item Esempio dalla slide:
		\begin{minted}{sql}
			CREATE DOMAIN Grade
			AS SMALLINT DEFAULT NULL
			CHECK (value >= 18 AND value <= 30);
		\end{minted}
		Questo \texttt{Grade} può poi essere usato come tipo di dato per una colonna.
	\end{itemize}
	
	\subsection{Vincoli (Constraints)}
	Servono a garantire l'integrità e la coerenza dei dati.
	\subsubsection{Vincoli comuni}
	\begin{itemize}
		\item \texttt{NOT NULL}: La colonna non può contenere valori \texttt{NULL}.
		\item \texttt{UNIQUE}: I valori nella colonna (o combinazione di colonne) devono essere unici.
		\item \texttt{PRIMARY KEY}:
		\begin{itemize}
			\item Identifica univocamente ogni riga. Implica \texttt{NOT NULL} e \texttt{UNIQUE}.
			\item Solo una per tabella. Può essere su colonna singola o multipla.
			\item Esempio (inline): \texttt{Number CHARACTER(6) PRIMARY KEY}
			\item Esempio (standalone): \texttt{PRIMARY KEY (Number)}
		\end{itemize}
		\item \textbf{Attenzione (Slide 23):}
		\begin{itemize}
			\item \texttt{UNIQUE (Surname, Name)}: La \textit{combinazione} di cognome e nome deve essere unica.
			\item \texttt{Surname CHARACTER(20) UNIQUE, Name CHARACTER(20) UNIQUE}: Il cognome deve essere unico \textbf{e} il nome deve essere unico (indipendentemente).
		\end{itemize}
	\end{itemize}
	\subsubsection{\texttt{FOREIGN KEY} e Integrità Referenziale}
	\begin{itemize}
		\item \texttt{FOREIGN KEY (colonna\_fk) REFERENCES tabella\_riferita (colonna\_pk\_riferita)}
		\item Garantisce che i valori nella \texttt{colonna\_fk} esistano nella \texttt{colonna\_pk\_riferita} della \texttt{tabella\_riferita}.
		\item \textit{Esempio Pratico (Relazione Post-User con Prisma):}
		\begin{minted}[bgcolor=black!70]{text}
			// Prisma Schema
			model User {
				id    Int     @id @default(autoincrement())
				posts Post[]
			}
			model Post {
				id        Int     @id @default(autoincrement())
				author    User    @relation(fields: [authorId], references: [id])
				authorId  Int // Foreign Key
			}
		\end{minted}
	\end{itemize}
	\subsubsection{Azioni Referenziali Triggerate}
	Cosa succede se un record referenziato viene cancellato o aggiornato: \texttt{ON DELETE | ON UPDATE}
	\begin{itemize}
		\item \texttt{CASCADE}: Propaga l'azione (es. se cancello un utente, cancella anche i suoi post).
		\item \texttt{SET NULL}: Imposta la foreign key a \texttt{NULL}.
		\item \texttt{SET DEFAULT}: Imposta la foreign key al suo valore di default.
		\item \texttt{NO ACTION} / \texttt{RESTRICT}: Impedisce l'operazione (spesso il default).
	\end{itemize}
	\subsubsection{\texttt{CHECK}}
	\begin{itemize}
		\item \texttt{CHECK (condizione)}: Specifica una condizione che deve essere vera per ogni riga.
		\item Esempio: \texttt{CHECK (Wage > 0)}
	\end{itemize}
	
	\subsection{Modificare la Struttura}
	\begin{itemize}
		\item \texttt{ALTER DOMAIN domain\_name [...opzioni...]};
		\item \texttt{ALTER TABLE table\_name [...opzioni...]};
		\begin{itemize}
			\item Opzioni: \texttt{ADD COLUMN}, \texttt{DROP COLUMN col\_name [RESTRICT|CASCADE]}, \texttt{ALTER COLUMN}, \texttt{ADD CONSTRAINT}, \texttt{DROP CONSTRAINT}.
		\end{itemize}
		\item \texttt{DROP DOMAIN domain\_name;}
		\item \texttt{DROP TABLE table\_name;} (cancella la tabella e tutti i suoi dati!)
	\end{itemize}
	
	\subsection{Indici}
	\begin{itemize}
		\item \texttt{CREATE INDEX index\_name ON table\_name (colonna1, [colonna2, ...]);}
		\item Migliorano le performance delle query.
		\item Strutture dati fisiche, non logiche.
		\item Le \texttt{PRIMARY KEY} e le colonne \texttt{UNIQUE} creano automaticamente un indice.
		\item \textit{Esempio Pratico:} Se fai spesso ricerche di utenti per \texttt{email}, un indice su \texttt{User(email)} velocizzerà molto.
	\end{itemize}
	
	
	\section{DML (Data Manipulation Language) - Interrogare e Modificare i Dati}
	
	\subsection{Interrogazioni (Query) - \texttt{SELECT}}
	La struttura base è:
	\begin{minted}{sql}
		SELECT [DISTINCT] {* | lista_colonne | espressioni [AS alias_colonna]}
		FROM tabella1 [AS alias_tabella1]
		[, tabella2 [AS alias_tabella2] ... |
		JOIN_TYPE tabella2 ON condizione_join]
		[WHERE condizione_filtro_righe]
		[GROUP BY lista_colonne_raggruppamento]
		[HAVING condizione_filtro_gruppi]
		[ORDER BY lista_colonne_ordinamento [ASC|DESC]];
	\end{minted}
	
	\subsubsection{Ordine Concettuale di Esecuzione}
	\begin{enumerate}
		\item \texttt{FROM} (e \texttt{JOIN}s)
		\item \texttt{WHERE}
		\item \texttt{GROUP BY}
		\item \texttt{HAVING}
		\item \texttt{SELECT}
		\item \texttt{DISTINCT}
		\item \texttt{ORDER BY}
	\end{enumerate}
	
	\subsubsection{Clausole base e alias}
	\begin{itemize}
		\item \texttt{SELECT *}: Seleziona tutte le colonne.
		\item Rinominare Colonne e Tabelle (Alias): \texttt{AS nome\_alias}
		\begin{sqlcode}
			SELECT P.Name AS GivenName FROM PEOPLE AS P;
		\end{sqlcode}
	\end{itemize}
	
	\subsubsection{Condizioni \texttt{WHERE}}
	\begin{itemize}
		\item Operatori logici: \texttt{AND}, \texttt{OR}, \texttt{NOT}.
		\item Operatori di confronto: \texttt{=}, \texttt{<>}, \texttt{<}, \texttt{>}, \texttt{<=}, \texttt{>=}.
		\item \texttt{LIKE}: Pattern matching (\texttt{\%} per zero o più caratteri, \texttt{\_} per un singolo carattere).
		\begin{sqlcode}
			WHERE Name LIKE 'J_m%';
		\end{sqlcode}
		\item \texttt{IS NULL} / \texttt{IS NOT NULL}: Per verificare valori \texttt{NULL}.
	\end{itemize}
	
	\subsubsection{\texttt{DISTINCT}}
	\begin{itemize}
		\item Rimuove le righe duplicate dal risultato.
	\end{itemize}
	
	\subsubsection{\texttt{JOIN}s}
	Combinano righe da due o più tabelle.
	\begin{itemize}
		\item \textbf{Implicit JOIN} (sconsigliato):
		\begin{sqlcode}
			SELECT ... FROM TableA, TableB WHERE TableA.id = TableB.a_id;
		\end{sqlcode}
		\item \textbf{Explicit JOIN} (preferito):
		\begin{itemize}
			\item \texttt{INNER JOIN} (o solo \texttt{JOIN}): Solo righe con corrispondenza in entrambe.
			\begin{sqlcode}
				SELECT ... FROM TableA INNER JOIN TableB ON TableA.id = TableB.a_id;
			\end{sqlcode}
			\item \texttt{LEFT [OUTER] JOIN}: Tutte le righe da sinistra, e le corrispondenti da destra (o \texttt{NULL}).
			\item \texttt{RIGHT [OUTER] JOIN}: Tutte le righe da destra, e le corrispondenti da sinistra (o \texttt{NULL}).
			\item \texttt{FULL [OUTER] JOIN}: Tutte le righe da entrambe; \texttt{NULL} dove non c'è corrispondenza.
			\item \texttt{NATURAL JOIN}: Join automatico su colonne con lo stesso nome (usare con cautela).
		\end{itemize}
		\item \textit{Esempio Pratico (Left Join):} Trovare tutti gli utenti e i loro post.
		\begin{sqlcode}
			SELECT U.name, P.title
			FROM User U LEFT JOIN Post P ON U.id = P.authorId;
		\end{sqlcode}
	\end{itemize}
	
	\subsubsection{Espressioni nella Target List}
	\begin{sqlcode}
		SELECT Income / 2 AS halvedIncome FROM PEOPLE;
	\end{sqlcode}
	
	\subsubsection{Ordinamento}
	\begin{itemize}
		\item \texttt{ORDER BY colonna [ASC|DESC];} (\texttt{ASC} è il default).
	\end{itemize}
	
	\subsubsection{Operazioni sugli Insiemi (Set Operations)}
	Le query devono avere lo stesso numero di colonne e tipi compatibili.
	\begin{itemize}
		\item \texttt{UNION}: Combina risultati, rimuovendo duplicati.
		\item \texttt{UNION ALL}: Come \texttt{UNION}, ma mantiene i duplicati.
		\item \texttt{INTERSECT}: Righe presenti in entrambi i risultati.
		\item \texttt{EXCEPT} (o \texttt{MINUS}): Righe nel primo risultato ma non nel secondo.
		\item \textit{Nota sulla denominazione delle colonne:} I nomi sono presi dalla prima query \texttt{SELECT}.
	\end{itemize}
	
	\subsection{Subquery (Query Annidate)}
	Una query all'interno di un'altra.
	
	\subsubsection{Nelle clausole \texttt{WHERE}}
	\begin{itemize}
		\item Con operatori di confronto: la subquery deve restituire un valore scalare.
		\begin{sqlcode}
			SELECT Name FROM PEOPLE WHERE Income = (SELECT MAX(Income) FROM PEOPLE);
		\end{sqlcode}
		\item \texttt{IN}: Verifica se un valore è nel set di risultati della subquery.
		\begin{sqlcode}
			SELECT Name FROM PEOPLE WHERE Name IN (SELECT Father FROM FATHERHOOD);
		\end{sqlcode}
		\item \texttt{ANY} / \texttt{SOME}, \texttt{ALL}: Usati con operatori di confronto.
		\begin{itemize}
			\item \texttt{valore > ANY (subquery)}: vero se \texttt{valore} $>$ di almeno un valore della subquery.
			\item \texttt{valore > ALL (subquery)}: vero se \texttt{valore} $>$ di tutti i valori della subquery.
		\end{itemize}
		\item \texttt{EXISTS}: Vero se la subquery restituisce almeno una riga.
		\begin{sqlcode}
			SELECT Name FROM PEOPLE P
			WHERE EXISTS (SELECT * FROM FATHERHOOD F WHERE F.Father = P.Name);
		\end{sqlcode}
		\item \texttt{NOT EXISTS}: Vero se la subquery non restituisce righe.
	\end{itemize}
	
	\subsubsection{Visibilità (Scope)}
	\begin{itemize}
		\item Una subquery può fare riferimento a colonne della query esterna (subquery correlata).
		\item La query esterna non può fare riferimento a colonne definite solo nella subquery.
		\item Se un nome di colonna è ambiguo, si assume quello dello scope più interno.
	\end{itemize}
	
	\subsubsection{Nelle clausole \texttt{FROM} (Derived Tables)}
	La subquery agisce come una tabella temporanea e deve avere un alias.
	\begin{sqlcode}
		SELECT P.Name, J.Child
		FROM PEOPLE P, (SELECT Child FROM FATHERHOOD WHERE Father='Jim') AS J
		WHERE P.Name = J.Child;
	\end{sqlcode}
	
	\subsubsection{Nelle clausole \texttt{SELECT} (Scalar Subqueries)}
	La subquery deve restituire un singolo valore per ogni riga della query esterna.
	\begin{sqlcode}
		SELECT C.Num, (SELECT COUNT(*) FROM ORDERS O WHERE O.CustomerNum = C.Num) AS OrderCount
		FROM CUSTOMER C;
	\end{sqlcode}
	
	\subsection{Funzioni Aggregate e Raggruppamento}
	
	\subsubsection{Funzioni Aggregate}
	\begin{itemize}
		\item \texttt{COUNT()}, \texttt{SUM()}, \texttt{AVG()}, \texttt{MIN()}, \texttt{MAX()}.
		\item Operano su un insieme di righe e restituiscono un singolo valore.
		\begin{itemize}
			\item \texttt{COUNT(*)}: conta tutte le righe.
			\item \texttt{COUNT(colonna)}: conta le righe dove \texttt{colonna} non è \texttt{NULL}.
			\item \texttt{COUNT(DISTINCT colonna)}: conta i valori unici non \texttt{NULL}.
			\item Le altre funzioni ignorano i \texttt{NULL}.
		\end{itemize}
		\item \textbf{Attenzione:} Non mischiare colonne non aggregate con funzioni aggregate nella \texttt{SELECT} list a meno che le colonne non aggregate non siano nella \texttt{GROUP BY}.
		\begin{itemize}
			\item Errato: \texttt{SELECT Name, MAX(Income) FROM PEOPLE;}
			\item Corretto: \texttt{SELECT MAX(Income) FROM PEOPLE;}
		\end{itemize}
	\end{itemize}
	
	\subsubsection{\texttt{GROUP BY lista\_colonne\_raggruppamento}}
	\begin{itemize}
		\item Raggruppa le righe che hanno gli stessi valori nelle colonne specificate.
		\item Le funzioni aggregate vengono applicate a ciascun gruppo.
		\item Esempio:
		\begin{sqlcode}
			SELECT Dept, AVG(Wage) FROM EMPLOYEE GROUP BY Dept;
		\end{sqlcode}
	\end{itemize}
	
	\subsubsection{\texttt{HAVING condizione\_filtro\_gruppi}}
	\begin{itemize}
		\item Filtra i gruppi creati da \texttt{GROUP BY}. La condizione in \texttt{HAVING} di solito coinvolge funzioni aggregate.
		\item \texttt{WHERE} filtra le righe \textit{prima} del raggruppamento, \texttt{HAVING} filtra i gruppi \textit{dopo}.
		\item Esempio:
		\begin{sqlcode}
			SELECT Dept, AVG(Wage) FROM EMPLOYEE
			GROUP BY Dept
			HAVING AVG(Wage) > 50000;
		\end{sqlcode}
	\end{itemize}
	
	\subsubsection{NULLs e Raggruppamento}
	\begin{itemize}
		\item I valori \texttt{NULL} in una colonna di raggruppamento formano un gruppo a sé stante.
	\end{itemize}
	
	\subsection{Modifica dei Dati}
	\subsubsection{\texttt{INSERT}}
	\begin{itemize}
		\item
		\begin{sqlcode}
			INSERT INTO table_name [(colonna1, colonna2, ...)]
			VALUES (valore1, valore2, ...);
		\end{sqlcode}
		\item Aggiunge una nuova riga. Se la lista colonne è omessa, fornire valori per tutte le colonne nell'ordine definito.
		\item
		\begin{sqlcode}
			INSERT INTO table_name [(colonna1, ...)]
			SELECT query_che_restituisce_righe_compatibili;
		\end{sqlcode}
	\end{itemize}
	
	\subsubsection{\texttt{UPDATE}}
	\begin{itemize}
		\item
		\begin{sqlcode}
			UPDATE table_name
			SET colonna1 = valore1, colonna2 = valore2, ...
			[WHERE condizione];
		\end{sqlcode}
		\item Modifica righe che soddisfano la \texttt{condizione}. \textbf{ATTENZIONE:} Senza \texttt{WHERE}, aggiorna tutte le righe!
		\item Il valore può essere un'espressione, \texttt{NULL}, \texttt{DEFAULT}, o una subquery scalare.
		\begin{sqlcode}
			UPDATE PEOPLE SET Income = Income * 1.1 WHERE Age < 30;
		\end{sqlcode}
	\end{itemize}
	
	\subsubsection{\texttt{DELETE}}
	\begin{itemize}
		\item
		\begin{sqlcode}
			DELETE FROM table_name [WHERE condizione];
		\end{sqlcode}
		\item Cancella righe che soddisfano la \texttt{condizione}. \textbf{ATTENZIONE:} Senza \texttt{WHERE}, cancella tutte le righe!
		\item Può innescare azioni referenziali.
	\end{itemize}
	
	\section{Concetti Chiave da Ricordare}
	\begin{enumerate}
		\item \textbf{SQL è Dichiarativo:} Dici \textit{cosa} vuoi, non \textit{come} ottenerlo.
		\item \textbf{Integrità dei Dati:} I vincoli sono fondamentali.
		\item \textbf{NULL è Speciale:} Rappresenta assenza di valore. Va trattato con \texttt{IS NULL} / \texttt{IS NOT NULL}.
		\item \textbf{JOINs sono Potenti:} Cuore delle query relazionali. Comprendere \texttt{INNER} vs \texttt{OUTER} JOINs è cruciale.
		\item \textbf{Aggregazione e Raggruppamento:} \texttt{GROUP BY}, funzioni aggregate e \texttt{HAVING} permettono calcoli sui dati.
		\item \textbf{Subquery:} Offrono flessibilità per query complesse.
	\end{enumerate}
	
\end{document}