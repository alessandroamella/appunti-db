\documentclass{article}

% --- Encoding e lingua ---
\usepackage[utf8]{inputenc}
\usepackage[italian]{babel}

% --- Margini e layout ---
\usepackage{geometry}
\geometry{a4paper, margin=1in}

% --- Font sans-serif ---
\usepackage[scaled]{helvet}
\renewcommand{\familydefault}{\sfdefault}
\usepackage[T1]{fontenc}

% --- Matematica ---
\usepackage{amsmath}
\usepackage{amssymb}

% --- Liste personalizzate ---
\usepackage{enumitem}


\usepackage{tikz}
\usetikzlibrary{shapes.geometric, positioning, calc}

% --- Hyperlink ---
\usepackage{hyperref}
\hypersetup{
	colorlinks=true,
	linkcolor=white,     % Cambiato per visibilità su sfondo nero
	filecolor=magenta,
	urlcolor=blue,
	pdftitle={Appunti sulla Modellazione Concettuale dei Dati},
	pdfpagemode=FullScreen,
}

% --- Colori e sfondo nero ---
\usepackage{xcolor}
\pagecolor{black}
\color{white}

% --- Evidenziazione del codice (richiede -shell-escape) ---
% Compilare con: pdflatex -shell-escape nomefile.tex
\usepackage{minted}
\setminted{
	frame=lines,     % Cornice attorno al codice
	framesep=2mm,
	fontsize=\small,
	breaklines=true, % A capo automatico per linee lunghe
	style=monokai    % Stile di highlighting
}
\newminted{sql}{frame=lines, framesep=2mm, fontsize=\small, breaklines=true, style=monokai}
\newminted{prisma}{frame=lines, framesep=2mm, fontsize=\small, breaklines=true, style=monokai}


% --- Titolo ---
\title{Appunti sulla Modellazione Concettuale dei Dati}
\author{Basato sulle slide del Prof. Danilo Montesi}
\date{\today}

\begin{document}
	
	\maketitle
	\tableofcontents
	\newpage
	
	\section{Perché la Modellazione Concettuale?}
	
	Partire direttamente a definire tabelle SQL (modello logico) è difficile e rischioso. I problemi principali sono:
	\begin{itemize}
		\item Ci si perde nei dettagli troppo presto.
		\item Il modello relazionale (tabelle, colonne, tipi) è troppo \textit{rigido} per le fasi iniziali di brainstorming e analisi dei requisiti.
	\end{itemize}
	
	La soluzione è il \textbf{Modello Concettuale} (ad esempio, il diagramma Entità-Relazione - ERD):
	\begin{itemize}
		\item Permette di ragionare sulla \textit{realtà di interesse} in modo \textbf{indipendente dall'implementazione} specifica (quale DBMS useremo, come saranno le tabelle, ecc.).
		\item Aiuta a definire le \textbf{classi di oggetti} (entità) e le loro \textbf{relazioni}.
		\item Fornisce una \textbf{rappresentazione visuale} chiara, utile per la documentazione e la comunicazione con gli stakeholder (anche non tecnici).
	\end{itemize}
	
	\paragraph{Esempio Pratico:} Immagina di dover creare un sistema per una biblioteca. Invece di pensare subito a \texttt{CREATE TABLE Libri (...)}, con il modello concettuale pensi: ``Ok, ho bisogno di \texttt{Libri}, \texttt{Utenti}, e una relazione che dice \textit{Un Utente prende in prestito un Libro}''. Questo è più astratto e flessibile.
	
	\section{Il Ciclo di Vita del Design del Database}
	
	Il design del database è una fase cruciale nello sviluppo di Sistemi Informativi (SI). Le fasi principali del design sono:
	
	\subsection{Design Concettuale}
	\begin{itemize}
		\item \textbf{Input:} Requisiti del database (cosa deve fare il sistema?).
		\item \textbf{Output:} \textbf{Schema Concettuale} (es. un diagramma Entità-Relazione - ERD).
		\item \textbf{Focus:} Il ``COSA?'' – quali informazioni ci servono e come sono collegate, ad alto livello.
		\item \textit{Esempio Pratico:} ``Abbiamo Entità \texttt{Studente} e \texttt{Corso}. Uno \texttt{Studente} si \texttt{IscriveA} un \texttt{Corso}.''
	\end{itemize}
	
		
	\subsection{Design Logico}
	\begin{itemize}
		\item \textbf{Input:} Schema Concettuale.
		\item \textbf{Output:} \textbf{Schema Logico} (es. definizione di tabelle per un DB relazionale, o collezioni per un DB NoSQL come MongoDB).
		\item \textbf{Focus:} Il ``COME?'' – come traduciamo il modello concettuale in un modello supportato da un tipo di DBMS (es. relazionale, a documenti, a grafo). È indipendente dal DBMS specifico, ma non dal \textit{tipo} di DBMS.
		\item \textit{Esempio Pratico:}\\
		\begin{center}
			\begin{tikzpicture}[
				entity/.style={rectangle, draw=white, fill=black!90, text=gray!10, minimum width=3cm, minimum height=1.2cm, font=\bfseries},
				relationship/.style={diamond, draw=white, fill=black!90, text=gray!10, aspect=2, minimum width=2.8cm, minimum height=1.2cm, font=\bfseries},
				line/.style={draw=white, thick},
				node distance=0.5cm and 1.5cm
				]
				
				% Top row
				\node[entity] (studente) {Studente};
				\node[relationship, right=of studente] (esame) {Esame};
				\node[entity, right=of esame] (lezione) {Lezione};
				
				\draw[line] (studente) -- (esame);
				\draw[line] (esame) -- (lezione);
				
				% Bottom row
				\node[entity, below=of studente] (impiegato) {Impiegato};
				\node[relationship, right=of impiegato] (residenza) {Residenza};
				\node[entity, right=of residenza] (città) {Città};
				
				\draw[line] (impiegato) -- (residenza);
				\draw[line] (residenza) -- (città);
				
			\end{tikzpicture}
		\end{center}
		
		
		\item Dallo schema concettuale sopra, con un po' d'immaginazione sulle releazioni, con SQL potrebbe essere:
		\begin{minted}{sql}
-- Table: Studente
CREATE TABLE Studente (
	id INT PRIMARY KEY,
	nome VARCHAR(100)
);

-- Table: Lezione
CREATE TABLE Lezione (
	id INT PRIMARY KEY,
	titolo VARCHAR(100)
);

-- Join Table: Esame (between Studente and Lezione)
CREATE TABLE Esame (
	studente_id INT,
	lezione_id INT,
	data DATE,
	voto INT,
	PRIMARY KEY (studente_id, lezione_id),
	FOREIGN KEY (studente_id) REFERENCES Studente(id),
	FOREIGN KEY (lezione_id) REFERENCES Lezione(id)
);

-- Table: Impiegato
CREATE TABLE Impiegato (
	id INT PRIMARY KEY,
	nome VARCHAR(100)
);

-- Table: Città
CREATE TABLE Città (
	id INT PRIMARY
)
		\end{minted}
	\end{itemize}
	
	\subsection{Design Fisico}
	\begin{itemize}
		\item \textbf{Input:} Schema Logico.
		\item \textbf{Output:} \textbf{Schema Fisico} (definizioni specifiche per il DBMS scelto: indici, partizionamento, filegroup, ecc.).
		\item \textbf{Focus:} Ottimizzazione delle performance e dello storage.
		\item \textit{Esempio Pratico:} ``Sulla tabella \texttt{Studenti}, creiamo un indice sulla colonna \texttt{Cognome} per velocizzare le ricerche.''
	\end{itemize}
	
	\section{Modelli di Dati: Costrutti, Schemi e Istanze}
	\begin{itemize}
		\item \textbf{Modello di Dati:} Una collezione di ``costrutti'' (come i tipi di dato in programmazione) per categorizzare i dati e descrivere le operazioni su di essi.
		\begin{itemize}
			\item Esempio: il modello relazionale usa il costrutto \texttt{relazione} (tabella) per insiemi uniformi di \texttt{tuple} (righe).
		\end{itemize}
		\item \textbf{Schema:} La struttura invariante nel tempo dei dati (aspetto \textit{intensionale}).
		\begin{itemize}
			\item SQL: \texttt{CREATE TABLE Users (id INT, name VARCHAR(255));}
			\item Prisma: \texttt{model User \{ id Int @id; name String; \}}
		\end{itemize}
		\item \textbf{Istanza:} I valori attuali dei dati in un certo momento, che cambiano nel tempo (aspetto \textit{estensionale}).
		\begin{itemize}
			\item SQL: Le righe effettive nella tabella \texttt{Users}: \texttt{(1, 'Alice'), (2, 'Bob')}.
			\item MongoDB: I documenti effettivi nella collezione \texttt{users}.
		\end{itemize}
	\end{itemize}
	
	\section{Il Modello Entità-Relazione (ER Model)}
	È il modello concettuale più usato. Ecco i suoi costrutti principali:
	
	\subsection{Entità (Entity)}
	\begin{itemize}
		\item Rappresenta una classe di ``oggetti'' (cose, persone, luoghi) del mondo reale che hanno proprietà comuni e un'esistenza autonoma.
		\item \textbf{Esempi:} \texttt{Studente}, \texttt{Prodotto}, \texttt{Dipartimento}.
		\item \textbf{Rappresentazione Grafica:} Rettangolo.
		\item \textbf{Convenzioni:} Nomi singolari, significativi.
		\item \textit{Paragone Pratico:} Simile a una classe in OOP, un \texttt{model} in Prisma, o una collezione in MongoDB.
	\end{itemize}

	\begin{figure}[h]
		\centering
		\begin{tikzpicture}[
			every node/.style={font=\small},
			entity/.style={rectangle, draw=white, fill=black!80, text=white, minimum width=2cm, minimum height=1cm},
			attribute/.style={ellipse, draw=white, fill=black!60, text=white, minimum width=1.5cm, minimum height=0.8cm},
			key/.style={attribute, draw=white, ultra thick}
		]
			\node[entity] (studente) {Studente};
			\node[attribute] (matricola) [above=1cm of studente] {Matricola};
			\node[attribute] (nome) [above right=0.5cm and 1cm of studente] {Nome};
			\node[attribute] (cognome) [right=1cm of studente] {Cognome};
			\node[attribute] (data) [below right=0.5cm and 1cm of studente] {DataNascita};
			
			\draw[white] (studente) -- (matricola);
			\draw[white] (studente) -- (nome);
			\draw[white] (studente) -- (cognome);
			\draw[white] (studente) -- (data);
			
			\draw[white, ultra thick] (studente) -- (matricola);
		\end{tikzpicture}
		\caption{Esempio di entità Studente con i suoi attributi. La linea spessa indica l'identificatore (Matricola).}
	\end{figure}
	
	\subsection{Relazione (Relationship)}
	\begin{itemize}
		\item Un legame, un'associazione logica tra due o più tipi di entità.
		\item \textbf{Esempi:} \texttt{Studente} \textit{Frequenta} \texttt{Corso}; \texttt{Impiegato} \textit{LavoraIn} \texttt{Dipartimento}.
		\item \textbf{Rappresentazione Grafica:} Rombo.
		\item \textbf{Convenzioni:} Nomi singolari (se possibile, nomi invece di verbi).
		\item \textbf{Tipi:}
		\begin{itemize}
			\item \textbf{Binarie:} Coinvolgono due entità.
			\item \textbf{N-arie:} Coinvolgono più di due entità (es. \texttt{Fornitore} \textit{Fornisce} \texttt{Prodotto} a un \texttt{Dipartimento}). Spesso si cerca di scomporle in binarie.
			\item \textbf{Ricorsive:} Un'entità è in relazione con se stessa (es. \texttt{Impiegato} \textit{Supervisiona} \texttt{Impiegato}).
			\begin{itemize}
				\item \textit{Paragone Pratico (Ricorsiva):} In SQL, una tabella \texttt{Impiegati} con una colonna \texttt{ID\_Manager} che è una foreign key a \texttt{Impiegati.ID}.
			\end{itemize}
		\end{itemize}
		\item \textbf{Ruoli:} Utili nelle relazioni ricorsive per chiarire il significato (es. \texttt{Presidente} -(Precedente/Successivo)-> \texttt{Successione}).
	\end{itemize}

	\begin{figure}[h]
		\centering
		\begin{tikzpicture}[
			every node/.style={font=\small},
			entity/.style={rectangle, draw=white, fill=black!80, text=white, minimum width=2cm, minimum height=1cm},
			relationship/.style={diamond, draw=white, fill=black!70, text=white, minimum width=1.5cm, minimum height=1cm}
		]
			\node[entity] (studente) {Studente};
			\node[relationship] (frequenta) [right=2cm of studente] {Frequenta};
			\node[entity] (corso) [right=2cm of frequenta] {Corso};
			
			\draw[white] (studente) -- node[above, text=white] {(0,N)} (frequenta);
			\draw[white] (frequenta) -- node[above, text=white] {(0,N)} (corso);
		\end{tikzpicture}
		\caption{Esempio di relazione molti-a-molti tra Studente e Corso.}
	\end{figure}
	
	\subsection{Promozione di Relazioni a Entità}
	\textbf{Quando?}
	\begin{itemize}
		\item Se una relazione ha attributi propri (es. la relazione \texttt{Iscrizione} tra \texttt{Studente} e \texttt{Corso} ha attributi come \texttt{DataIscrizione} e \texttt{VotoEsame}).
		\item Se uno studente può sostenere lo \textit{stesso} esame più volte (es. per migliorare il voto). La semplice relazione \texttt{Studente-Esame-Corso} non cattura i tentativi multipli.
	\end{itemize}
	\textbf{Come?} La relazione diventa un'entità ``associativa''.
	\begin{itemize}
		\item \textit{Esempio Pratico:} La relazione \texttt{Studente}-\texttt{Iscrizione}-\texttt{Corso} diventa: Entità \texttt{Studente} --- Relazione \texttt{HaSostenuto} --- Entità \texttt{IstanzaEsame} --- Relazione \texttt{Riguarda} --- Entità \texttt{Corso}. L'entità \texttt{IstanzaEsame} avrà attributi come \texttt{Data}, \texttt{Voto}.
		\item \textit{SQL:} Questo si traduce in una ``join table'' o ``tabella associativa'':
		\begin{minted}{sql}
CREATE TABLE EsamiSostenuti (
	ID_Studente INT,
	ID_Corso INT,
	Data DATE,
	Voto INT,
	PRIMARY KEY (ID_Studente, ID_Corso, Data) -- Data inclusa per tentativi multipli
);
		\end{minted}
	\end{itemize}

	\begin{figure}[h]
		\centering
		\begin{tikzpicture}[
			every node/.style={font=\small},
			entity/.style={rectangle, draw=white, fill=black!80, text=white, minimum width=2cm, minimum height=1cm},
			relationship/.style={diamond, draw=white, fill=black!70, text=white, minimum width=1.5cm, minimum height=1cm},
			weak/.style={rectangle, draw=white, fill=black!60, text=white, minimum width=2cm, minimum height=1cm, double}
		]
			\node[entity] (studente) {Studente};
			\node[weak] (esame) [right=2cm of studente] {IstanzaEsame};
			\node[entity] (corso) [right=2cm of esame] {Corso};
			
			\draw[white] (studente) -- node[above, text=white] {(0,N)} (esame);
			\draw[white] (esame) -- node[above, text=white] {(1,1)} (corso);
			
			\node[draw=white, fill=black!50, text=white, ellipse, minimum width=1cm, minimum height=0.5cm] (data) [above=0.5cm of esame] {Data};
			\node[draw=white, fill=black!50, text=white, ellipse, minimum width=1cm, minimum height=0.5cm] (voto) [below=0.5cm of esame] {Voto};
			
			\draw[white] (esame) -- (data);
			\draw[white] (esame) -- (voto);
		\end{tikzpicture}
		\caption{Esempio di promozione della relazione Esame a entità debole (doppio bordo) con attributi propri.}
	\end{figure}
	
	\subsection{Attributi (Attribute)}
	\begin{itemize}
		\item Una proprietà o caratteristica di un'entità o di una relazione.
		\item Collega ogni istanza dell'entità/relazione a un valore da un ``dominio'' (insieme di valori possibili).
		\item \textbf{Esempi:} \texttt{Nome} dell'entità \texttt{Studente}; \texttt{Data} della relazione \texttt{Esame}.
		\item \textbf{Rappresentazione Grafica:} Ovale.
		\item \textbf{Tipi:}
		\begin{itemize}
			\item \textbf{Semplici:} Atomici (es. \texttt{Età}).
			\item \textbf{Composti:} Possono essere scomposti in sotto-attributi (es. \texttt{Indirizzo} composto da \texttt{Via}, \texttt{NumeroCivico}, \texttt{Città}).
			\begin{itemize}
				\item \textit{Paragone Pratico (Composto):} In MongoDB è naturale: \texttt{address: \{ street: "...", city: "..." \}}. In SQL, spesso si ``appiattiscono'' in colonne separate (\texttt{Via}, \texttt{NumeroCivico}, \texttt{Citta}) o, se complesso, si mette in una tabella separata.
			\end{itemize}
		\end{itemize}
	\end{itemize}
	
	\subsection{Cardinalità (Cardinality)}
	Specifica il numero minimo e massimo di istanze di un'entità che possono partecipare a una relazione, o il numero di valori che un attributo può assumere.
	\begin{itemize}
		\item \textbf{Notazione comune:} \texttt{(min, max)}
		\begin{itemize}
			\item \texttt{min = 0}: partecipazione opzionale.
			\item \texttt{min = 1} (o più): partecipazione obbligatoria.
			\item \texttt{max = 1}: al massimo una.
			\item \texttt{max = N} (o \texttt{*}): molte.
		\end{itemize}
	\end{itemize}
	
	\subsubsection{Cardinalità delle Relazioni}
	\begin{itemize}
		\item \textbf{Esempio:} \texttt{Impiegato (1,1)} --- \texttt{LavoraPer} --- \texttt{(0,N) Dipartimento}
		\begin{itemize}
			\item Un \texttt{Impiegato} deve lavorare per \textbf{esattamente un} \texttt{Dipartimento}.
			\item Un \texttt{Dipartimento} può avere \textbf{da zero a molti} \texttt{Impiegati}.
		\end{itemize}
		\item \textbf{Tipi comuni (basati su max):}
		\begin{itemize}
			\item \textbf{Uno-a-Uno (1:1):} Es. \texttt{Persona (0,1)} --- \texttt{Possiede} --- \texttt{(0,1) Pacemaker}.
			\item \textbf{Uno-a-Molti (1:N):} Es. \texttt{Cliente (1,1)} --- \texttt{Effettua} --- \texttt{(0,N) Ordine}.
			\item \textbf{Molti-a-Molti (M:N):} Es. \texttt{Studente (0,N)} --- \texttt{Frequenta} --- \texttt{(0,N) Corso}.
			\begin{itemize}
				\item \textit{Paragone Pratico (M:N):} In SQL, le relazioni M:N si implementano sempre con una tabella associativa intermedia. Prisma gestisce questo in modo più astratto.
			\end{itemize}
		\end{itemize}
	\end{itemize}

	\begin{figure}[h]
		\centering
		\begin{tikzpicture}[
			every node/.style={font=\small},
			entity/.style={rectangle, draw=white, fill=black!80, text=white, minimum width=2cm, minimum height=1cm},
			relationship/.style={diamond, draw=white, fill=black!70, text=white, minimum width=1.5cm, minimum height=1cm}
		]
			% 1:1 relationship
			\node[entity] (persona) {Persona};
			\node[relationship] (possiede) [right=2cm of persona] {Possiede};
			\node[entity] (pacemaker) [right=2cm of possiede] {Pacemaker};
			
			\draw[white] (persona) -- node[above, text=white] {(0,1)} (possiede);
			\draw[white] (possiede) -- node[above, text=white] {(0,1)} (pacemaker);
			
			% 1:N relationship
			\node[entity] (cliente) [below=2cm of persona] {Cliente};
			\node[relationship] (effettua) [right=2cm of cliente] {Effettua};
			\node[entity] (ordine) [right=2cm of effettua] {Ordine};
			
			\draw[white] (cliente) -- node[above, text=white] {(1,1)} (effettua);
			\draw[white] (effettua) -- node[above, text=white] {(0,N)} (ordine);
		\end{tikzpicture}
		\caption{Esempi di relazioni uno-a-uno (Persona-Pacemaker) e uno-a-molti (Cliente-Ordine).}
	\end{figure}
	
	\subsubsection{Cardinalità degli Attributi}
	\begin{itemize}
		\item \texttt{(0,1)}: Attributo opzionale (può essere \texttt{NULL}). Es. \texttt{NumeroTelefonoSecondario}.
		\item \texttt{(1,1)}: Attributo obbligatorio, singolo valore (default). Es. \texttt{CodiceFiscale}.
		\item \texttt{(0,N)} o \texttt{(1,N)}: Attributo multivalore (un'entità può avere più valori per quell'attributo). Es. \texttt{NumeroTelefono} (una persona può avere più numeri).
		\begin{itemize}
			\item \textit{Paragone Pratico (Multivalore):} In SQL, si usa una tabella separata: \texttt{Persona(ID\_Persona)}, \texttt{NumeriTelefono(ID\_Persona\_FK, Numero)}. In MongoDB, si usa un array: \texttt{telefoni: ["123", "456"]}.
		\end{itemize}
	\end{itemize}

	\begin{figure}[h]
		\centering
		\begin{tikzpicture}[
			every node/.style={font=\small},
			entity/.style={rectangle, draw=white, fill=black!80, text=white, minimum width=2cm, minimum height=1cm},
			attribute/.style={ellipse, draw=white, fill=black!60, text=white, minimum width=1.5cm, minimum height=0.8cm},
			multivalue/.style={ellipse, draw=white, fill=black!50, text=white, minimum width=1.5cm, minimum height=0.8cm, double}
		]
			\node[entity] (persona) {Persona};
			
			% Single value attributes
			\node[attribute] (cf) [above=1cm of persona] {CF};
			\node[attribute] (nome) [above right=0.5cm and 1cm of persona] {Nome};
			
			% Optional attribute
			\node[attribute] (tel2) [right=1cm of persona] {Tel2};
			
			% Multivalue attribute
			\node[multivalue] (tel) [below right=0.5cm and 1cm of persona] {Telefoni};
			
			\draw[white] (persona) -- (cf);
			\draw[white] (persona) -- (nome);
			\draw[white] (persona) -- (tel2);
			\draw[white] (persona) -- (tel);
			
			% Cardinality labels
			\node[text=white] at ($(cf)!0.5!(persona)$) [left] {(1,1)};
			\node[text=white] at ($(tel2)!0.5!(persona)$) [right] {(0,1)};
			\node[text=white] at ($(tel)!0.5!(persona)$) [right] {(0,N)};
		\end{tikzpicture}
		\caption{Esempi di attributi con diverse cardinalità: obbligatorio (CF), opzionale (Tel2), e multivalore (Telefoni).}
	\end{figure}
	
	\subsection{Identificatori (Chiavi - Keys)}
	\begin{itemize}
		\item Un attributo o un insieme di attributi che identificano univocamente ogni istanza di un'entità.
		\item \textbf{Rappresentazione Grafica:} Attributo sottolineato.
		\item \textbf{Tipi:}
		\begin{itemize}
			\item \textbf{Identificatore Interno:} Formato da attributi della stessa entità.
			\begin{itemize}
				\item Es. \texttt{CodiceFiscale} per l'entità \texttt{Persona}.
				\item \textit{Paragone Pratico:} \texttt{PRIMARY KEY} in SQL; \texttt{\_id} in MongoDB; \texttt{@id} in Prisma.
			\end{itemize}
			\item \textbf{Identificatore Esterno:} Formato da attributi dell'entità più l'identificatore di un'entità esterna a cui è collegata tramite una relazione con cardinalità \texttt{(1,1)} dal lato dell'entità da identificare. Usato per ``entità deboli'' che non possono esistere o essere identificate senza l'entità ``forte''.
			\begin{itemize}
				\item Es. \texttt{NumeroRiga} (attributo di \texttt{RigaOrdine}) + \texttt{ID\_Ordine} (dall'entità \texttt{Ordine}) identifica univocamente una \texttt{RigaOrdine}. \texttt{RigaOrdine} è un'entità debole rispetto a \texttt{Ordine}.
			\end{itemize}
		\end{itemize}
		\item Ogni entità deve avere almeno un identificatore.
		\item Le relazioni di solito non hanno identificatori (se ne hanno bisogno, si promuovono a entità).
	\end{itemize}

	\begin{figure}[h]
		\centering
		\begin{tikzpicture}[
			every node/.style={font=\small},
			entity/.style={rectangle, draw=white, fill=black!80, text=white, minimum width=2cm, minimum height=1cm},
			weak/.style={rectangle, draw=white, fill=black!60, text=white, minimum width=2cm, minimum height=1cm, double},
			attribute/.style={ellipse, draw=white, fill=black!60, text=white, minimum width=1.5cm, minimum height=0.8cm},
			key/.style={attribute, draw=white, ultra thick}
		]
			% Strong entity with internal key
			\node[entity] (ordine) {Ordine};
			\node[key] (id) [above=1cm of ordine] {ID\_Ordine};
			\draw[white, ultra thick] (ordine) -- (id);
			
			% Weak entity with external key
			\node[weak] (riga) [right=3cm of ordine] {RigaOrdine};
			\node[attribute] (num) [above=1cm of riga] {NumeroRiga};
			\node[attribute] (qty) [right=1cm of riga] {Quantità};
			
			\draw[white] (riga) -- (num);
			\draw[white] (riga) -- (qty);
			\draw[white] (ordine) -- node[above, text=white] {(1,1)} (riga);
			
			% Dashed line for identifying relationship
			\draw[white, dashed] (id) -- (num);
		\end{tikzpicture}
		\caption{Esempio di entità forte (Ordine) con identificatore interno e entità debole (RigaOrdine) con identificatore esterno.}
	\end{figure}
	
	\subsection{Generalizzazione/Specializzazione (Inheritance)}
	\begin{itemize}
		\item Una relazione tra un'entità genitore (superclasse, es. \texttt{Veicolo}) e una o più entità figlie (sottoclassi, es. \texttt{Automobile}, \texttt{Motocicletta}).
		\item Le figlie sono ``tipi di'' genitore: ereditano attributi e relazioni del genitore e possono averne di propri.
		\item \textbf{Rappresentazione Grafica:} Freccia (triangolo vuoto) dalle figlie al genitore.
		\item \textbf{Proprietà:}
		\begin{itemize}
			\item \textbf{Ereditarietà:} Le proprietà del genitore sono implicitamente presenti nelle figlie.
			\item \textbf{Copertura (Total/Partial):}
			\begin{itemize}
				\item \textbf{Totale:} Ogni istanza del genitore DEVE essere un'istanza di (almeno) una delle figlie. (Es. \texttt{Persona} -> \texttt{Maschio}, \texttt{Femmina}).
				\item \textbf{Parziale:} Un'istanza del genitore PUÒ essere un'istanza di una figlia (o solo del tipo genitore). (Es. \texttt{Veicolo} -> \texttt{Automobile}, \texttt{Motocicletta}).
			\end{itemize}
			\item \textbf{Disgiunzione (Disjoint/Overlapping):}
			\begin{itemize}
				\item \textbf{Disgiunta:} Un'istanza del genitore può essere al massimo un tipo di figlia. (Es. \texttt{Persona} è \texttt{Maschio} O \texttt{Femmina}).
				\item \textbf{Sovrapposta:} Un'istanza del genitore può essere più tipi di figlia contemporaneamente (raro e più complesso da modellare).
			\end{itemize}
			\item Di solito ci si concentra su generalizzazioni \textbf{Disgiunte (Totali o Parziali)}.
		\end{itemize}
		\item \textit{Paragone Pratico:}
		\begin{itemize}
			\item OOP: \texttt{class Veicolo \{\}}, \texttt{class Automobile extends Veicolo \{\}}.
			\item SQL: Ci sono diversi pattern per implementare l'ereditarietà.
			\item Prisma: Può essere modellato con campi discriminatori o modelli separati con relazioni.
		\end{itemize}
	\end{itemize}

	\begin{figure}[h]
		\centering
		\begin{tikzpicture}[
			every node/.style={font=\small},
			entity/.style={rectangle, draw=white, fill=black!80, text=white, minimum width=2cm, minimum height=1cm},
			subclass/.style={rectangle, draw=white, fill=black!70, text=white, minimum width=2cm, minimum height=1cm},
			attribute/.style={ellipse, draw=white, fill=black!60, text=white, minimum width=1.5cm, minimum height=0.8cm},
			generalization/.style={white, fill=white, draw=white, line width=1pt, shape=isosceles triangle, shape border rotate=90, isosceles triangle stretches=true, minimum height=0.6cm, minimum width=0.8cm}
		]
			% Parent class
			\node[entity] (veicolo) {Veicolo};
			\node[attribute] (targa) [above=1cm of veicolo] {Targa};
			\node[attribute] (anno) [right=1cm of veicolo] {Anno};
			
			% Subclasses
			\node[subclass] (auto) [below=2.5cm of veicolo, xshift=-2cm] {Automobile};
			\node[subclass] (moto) [below=2.5cm of veicolo, xshift=2cm] {Motocicletta};
			
			% Subclass attributes
			\node[attribute] (porte) [below=0.5cm of auto] {NumPorte};
			\node[attribute] (cilindrata) [below=0.5cm of moto] {Cilindrata};
			
			% Attributes connections
			\draw[white] (veicolo) -- (targa);
			\draw[white] (veicolo) -- (anno);
			\draw[white] (auto) -- (porte);
			\draw[white] (moto) -- (cilindrata);
			
			% Generalization triangle
			\node[generalization] (gen) at ($(veicolo.south) + (0,-0.5)$) {};
			
			% Lines from triangle to subclasses
			\draw[white] (gen.south) -- ($(gen.south) + (0,-0.2)$) -| (auto);
			\draw[white] (gen.south) -- ($(gen.south) + (0,-0.2)$) -| (moto);
			
			% Line from superclass to triangle
			\draw[white] (veicolo) -- (gen);
		\end{tikzpicture}
		\caption{Esempio di generalizzazione/specializzazione: Veicolo come superclasse e Automobile/Motocicletta come sottoclassi con simbolo di ereditarietà (triangolo vuoto).}
	\end{figure}
	
	\section{Documentazione}
	\begin{itemize}
		\item \textbf{Dizionario dei Dati:} Descrive in dettaglio ogni entità, relazione e attributo.
		\item \textbf{Vincoli Non Esprimibili:} Alcuni vincoli di business non possono essere rappresentati graficamente nell'ERD (es. ``Lo stipendio di un impiegato non può superare quello del suo manager''). Vanno documentati a parte.
		\begin{itemize}
			\item \textit{Paragone Pratico:} Questi vincoli si implementano spesso con \texttt{CHECK constraints} in SQL, \texttt{triggers}, o a livello applicativo.
		\end{itemize}
	\end{itemize}
	
	\section{UML (Unified Modeling Language) come Alternativa}
	\begin{itemize}
		\item UML è un linguaggio di modellazione più ampio, usato per vari aspetti dello sviluppo software.
		\item Per la modellazione dei dati, si usano principalmente i \textbf{Diagrammi delle Classi (Class Diagrams)}.
		\item Molti concetti ER hanno un equivalente in UML:
		\begin{itemize}
			\item \textbf{Entità} -> \textbf{Classe}
			\item \textbf{Relazione} -> \textbf{Associazione}
			\item \textbf{Relazione con attributi} -> \textbf{Classe di Associazione}
			\item \textbf{Cardinalità:} \texttt{1}, \texttt{0..1}, \texttt{*}, \texttt{1..*}
			\item \textbf{Identificatori:} \texttt{\{id\}} accanto all'attributo.
			\item \textbf{Generalizzazione/Specializzazione:} Freccia con triangolo vuoto verso la superclasse.
			\item \textbf{Concetti specifici UML:} Aggregazione (rombo vuoto), Composizione (rombo pieno).
		\end{itemize}
	\end{itemize}
	
\end{document}
