\section{Perché la Modellazione Concettuale?}

Partire direttamente a definire tabelle SQL (modello logico) è difficile e rischioso. I problemi principali sono:
\begin{itemize}
	\item Ci si perde nei dettagli troppo presto.
	\item Il modello relazionale (tabelle, colonne, tipi) è troppo \textit{rigido} per le fasi iniziali di brainstorming e analisi dei requisiti.
\end{itemize}

La soluzione è il \textbf{Modello Concettuale} (ad esempio, il diagramma Entità-Relazione - ERD):
\begin{itemize}
	\item Permette di ragionare sulla \textit{realtà di interesse} in modo \textbf{indipendente dall'implementazione} specifica (quale DBMS useremo, come saranno le tabelle, ecc.).
	\item Aiuta a definire le \textbf{classi di oggetti} (entità) e le loro \textbf{relazioni}.
	\item Fornisce una \textbf{rappresentazione visuale} chiara, utile per la documentazione e la comunicazione con gli stakeholder (anche non tecnici).
\end{itemize}

\paragraph{Esempio Pratico:} Immagina di dover creare un sistema per una biblioteca. Invece di pensare subito a \texttt{CREATE TABLE Libri (...)}, con il modello concettuale pensi: ``Ok, ho bisogno di \texttt{Libri}, \texttt{Utenti}, e una relazione che dice \textit{Un Utente prende in prestito un Libro}''. Questo è più astratto e flessibile.

\section{Il Ciclo di Vita del Design del Database}

Il design del database è una fase cruciale nello sviluppo di Sistemi Informativi (SI). Le fasi principali del design sono:

\subsection{Design Concettuale}
\begin{itemize}
	\item \textbf{Input:} Requisiti del database (cosa deve fare il sistema?).
	\item \textbf{Output:} \textbf{Schema Concettuale} (es. un diagramma Entità-Relazione - ERD).
	\item \textbf{Focus:} Il ``COSA?'' – quali informazioni ci servono e come sono collegate, ad alto livello.
	\item \textit{Esempio Pratico:} ``Abbiamo Entità \texttt{Studente} e \texttt{Corso}. Uno \texttt{Studente} si \texttt{IscriveA} un \texttt{Corso}.''
\end{itemize}

	
\subsection{Design Logico}
\begin{itemize}
	\item \textbf{Input:} Schema Concettuale.
	\item \textbf{Output:} \textbf{Schema Logico} (es. definizione di tabelle per un DB relazionale, o collezioni per un DB NoSQL come MongoDB).
	\item \textbf{Focus:} Il ``COME?'' – come traduciamo il modello concettuale in un modello supportato da un tipo di DBMS (es. relazionale, a documenti, a grafo). È indipendente dal DBMS specifico, ma non dal \textit{tipo} di DBMS.
	\item \textit{Esempio Pratico:}\\
	\begin{center}
		\begin{tikzpicture}[
			entity/.style={rectangle, draw=white, fill=black!90, text=gray!10, minimum width=3cm, minimum height=1.2cm, font=\bfseries},
			relationship/.style={diamond, draw=white, fill=black!90, text=gray!10, aspect=2, minimum width=2.8cm, minimum height=1.2cm, font=\bfseries},
			line/.style={draw=white, thick},
			node distance=0.5cm and 1.5cm
			]
			
			% Top row
			\node[entity] (studente) {Studente};
			\node[relationship, right=of studente] (esame) {Esame};
			\node[entity, right=of esame] (lezione) {Lezione};
			
			\draw[line] (studente) -- (esame);
			\draw[line] (esame) -- (lezione);
			
			% Bottom row
			\node[entity, below=of studente] (impiegato) {Impiegato};
			\node[relationship, right=of impiegato] (residenza) {Residenza};
			\node[entity, right=of residenza] (città) {Città};
			
			\draw[line] (impiegato) -- (residenza);
			\draw[line] (residenza) -- (città);
			
		\end{tikzpicture}
	\end{center}
	
	
	\item Dallo schema concettuale sopra, con un po' d'immaginazione sulle releazioni, con SQL potrebbe essere:
	\begin{minted}{sql}
-- Table: Studente
CREATE TABLE Studente (
id INT PRIMARY KEY,
nome VARCHAR(100)
);

-- Table: Lezione
CREATE TABLE Lezione (
id INT PRIMARY KEY,
titolo VARCHAR(100)
);

-- Join Table: Esame (between Studente and Lezione)
CREATE TABLE Esame (
studente_id INT,
lezione_id INT,
data DATE,
voto INT,
PRIMARY KEY (studente_id, lezione_id),
FOREIGN KEY (studente_id) REFERENCES Studente(id),
FOREIGN KEY (lezione_id) REFERENCES Lezione(id)
);

-- Table: Impiegato
CREATE TABLE Impiegato (
id INT PRIMARY KEY,
nome VARCHAR(100)
);

-- Table: Città
CREATE TABLE Città (
id INT PRIMARY
)
	\end{minted}
\end{itemize}

\subsection{Design Fisico}
\begin{itemize}
	\item \textbf{Input:} Schema Logico.
	\item \textbf{Output:} \textbf{Schema Fisico} (definizioni specifiche per il DBMS scelto: indici, partizionamento, filegroup, ecc.).
	\item \textbf{Focus:} Ottimizzazione delle performance e dello storage.
	\item \textit{Esempio Pratico:} ``Sulla tabella \texttt{Studenti}, creiamo un indice sulla colonna \texttt{Cognome} per velocizzare le ricerche.''
\end{itemize}

\section{Modelli di Dati: Costrutti, Schemi e Istanze}
\begin{itemize}
	\item \textbf{Modello di Dati:} Una collezione di ``costrutti'' (come i tipi di dato in programmazione) per categorizzare i dati e descrivere le operazioni su di essi.
	\begin{itemize}
		\item Esempio: il modello relazionale usa il costrutto \texttt{relazione} (tabella) per insiemi uniformi di \texttt{tuple} (righe).
	\end{itemize}
	\item \textbf{Schema:} La struttura invariante nel tempo dei dati (aspetto \textit{intensionale}).
	\begin{itemize}
		\item SQL: \texttt{CREATE TABLE Users (id INT, name VARCHAR(255));}
		\item Prisma: \texttt{model User \{ id Int @id; name String; \}}
	\end{itemize}
	\item \textbf{Istanza:} I valori attuali dei dati in un certo momento, che cambiano nel tempo (aspetto \textit{estensionale}).
	\begin{itemize}
		\item SQL: Le righe effettive nella tabella \texttt{Users}: \texttt{(1, 'Alice'), (2, 'Bob')}.
		\item MongoDB: I documenti effettivi nella collezione \texttt{users}.
	\end{itemize}
\end{itemize}

\section{Il Modello Entità-Relazione (ER Model)}
È il modello concettuale più usato. Ecco i suoi costrutti principali:

\subsection{Entità (Entity)}
\begin{itemize}
	\item Rappresenta una classe di ``oggetti'' (cose, persone, luoghi) del mondo reale che hanno proprietà comuni e un'esistenza autonoma.
	\item \textbf{Esempi:} \texttt{Studente}, \texttt{Prodotto}, \texttt{Dipartimento}.
	\item \textbf{Rappresentazione Grafica:} Rettangolo.
	\item \textbf{Convenzioni:} Nomi singolari, significativi.
	\item \textit{Paragone Pratico:} Simile a una classe in OOP, un \texttt{model} in Prisma, o una collezione in MongoDB.
\end{itemize}

\begin{figure}[h]
	\centering
	\begin{tikzpicture}[
		every node/.style={font=\small},
		entity/.style={rectangle, draw=white, fill=black!80, text=white, minimum width=2cm, minimum height=1cm},
		attribute/.style={ellipse, draw=white, fill=black!60, text=white, minimum width=1.5cm, minimum height=0.8cm},
		key/.style={attribute, draw=white, ultra thick}
	]
		\node[entity] (studente) {Studente};
		\node[attribute] (matricola) [above=1cm of studente] {Matricola};
		\node[attribute] (nome) [above right=0.5cm and 1cm of studente] {Nome};
		\node[attribute] (cognome) [right=1cm of studente] {Cognome};
		\node[attribute] (data) [below right=0.5cm and 1cm of studente] {DataNascita};
		
		\draw[white] (studente) -- (matricola);
		\draw[white] (studente) -- (nome);
		\draw[white] (studente) -- (cognome);
		\draw[white] (studente) -- (data);
		
		\draw[white, ultra thick] (studente) -- (matricola);
	\end{tikzpicture}
	\caption{Esempio di entità Studente con i suoi attributi. La linea spessa indica l'identificatore (Matricola).}
\end{figure}

\subsection{Relazione (Relationship)}
\begin{itemize}
	\item Un legame, un'associazione logica tra due o più tipi di entità.
	\item \textbf{Esempi:} \texttt{Studente} \textit{Frequenta} \texttt{Corso}; \texttt{Impiegato} \textit{LavoraIn} \texttt{Dipartimento}.
	\item \textbf{Rappresentazione Grafica:} Rombo.
	\item \textbf{Convenzioni:} Nomi singolari (se possibile, nomi invece di verbi).
	\item \textbf{Tipi:}
	\begin{itemize}
		\item \textbf{Binarie:} Coinvolgono due entità.
		\item \textbf{N-arie:} Coinvolgono più di due entità (es. \texttt{Fornitore} \textit{Fornisce} \texttt{Prodotto} a un \texttt{Dipartimento}). Spesso si cerca di scomporle in binarie.
		\item \textbf{Ricorsive:} Un'entità è in relazione con se stessa (es. \texttt{Impiegato} \textit{Supervisiona} \texttt{Impiegato}).
		\begin{itemize}
			\item \textit{Paragone Pratico (Ricorsiva):} In SQL, una tabella \texttt{Impiegati} con una colonna \texttt{ID\_Manager} che è una foreign key a \texttt{Impiegati.ID}.
		\end{itemize}
	\end{itemize}
	\item \textbf{Ruoli:} Utili nelle relazioni ricorsive per chiarire il significato (es. \texttt{Presidente} -(Precedente/Successivo)-> \texttt{Successione}).
\end{itemize}

\begin{figure}[h]
	\centering
	\begin{tikzpicture}[
		every node/.style={font=\small},
		entity/.style={rectangle, draw=white, fill=black!80, text=white, minimum width=2cm, minimum height=1cm},
		relationship/.style={diamond, draw=white, fill=black!70, text=white, minimum width=1.5cm, minimum height=1cm}
	]
		\node[entity] (studente) {Studente};
		\node[relationship] (frequenta) [right=2cm of studente] {Frequenta};
		\node[entity] (corso) [right=2cm of frequenta] {Corso};
		
		\draw[white] (studente) -- node[above, text=white] {(0,N)} (frequenta);
		\draw[white] (frequenta) -- node[above, text=white] {(0,N)} (corso);
	\end{tikzpicture}
	\caption{Esempio di relazione molti-a-molti tra Studente e Corso.}
\end{figure}

\subsection{Promozione di Relazioni a Entità}
\textbf{Quando?}
\begin{itemize}
	\item Se una relazione ha attributi propri (es. la relazione \texttt{Iscrizione} tra \texttt{Studente} e \texttt{Corso} ha attributi come \texttt{DataIscrizione} e \texttt{VotoEsame}).
	\item Se uno studente può sostenere lo \textit{stesso} esame più volte (es. per migliorare il voto). La semplice relazione \texttt{Studente-Esame-Corso} non cattura i tentativi multipli.
\end{itemize}
\textbf{Come?} La relazione diventa un'entità ``associativa''.
\begin{itemize}
	\item \textit{Esempio Pratico:} La relazione \texttt{Studente}-\texttt{Iscrizione}-\texttt{Corso} diventa: Entità \texttt{Studente} --- Relazione \texttt{HaSostenuto} --- Entità \texttt{IstanzaEsame} --- Relazione \texttt{Riguarda} --- Entità \texttt{Corso}. L'entità \texttt{IstanzaEsame} avrà attributi come \texttt{Data}, \texttt{Voto}.
	\item \textit{SQL:} Questo si traduce in una ``join table'' o ``tabella associativa'':
	\begin{minted}{sql}
CREATE TABLE EsamiSostenuti (
ID_Studente INT,
ID_Corso INT,
Data DATE,
Voto INT,
PRIMARY KEY (ID_Studente, ID_Corso, Data) -- Data inclusa per tentativi multipli
);
	\end{minted}
\end{itemize}

\begin{figure}[h]
	\centering
	\begin{tikzpicture}[
		every node/.style={font=\small},
		entity/.style={rectangle, draw=white, fill=black!80, text=white, minimum width=2cm, minimum height=1cm},
		relationship/.style={diamond, draw=white, fill=black!70, text=white, minimum width=1.5cm, minimum height=1cm},
		% Nuova definizione per l'entità debole con doppio bordo visibile su sfondo nero
		weak/.style={
			rectangle,
			draw=white,
			fill=black!60, 
			text=white, 
			minimum width=2cm, 
			minimum height=1cm,
			path picture={
				\draw[white] 
				([shift={(0.05cm,0.05cm)}]path picture bounding box.south west) 
				rectangle 
				([shift={(-0.05cm,-0.05cm)}]path picture bounding box.north east);
			}
		}
	]
		\node[entity] (studente) {Studente};
		\node[weak] (esame) [right=2cm of studente] {IstanzaEsame};
		\node[entity] (corso) [right=2cm of esame] {Corso};
		
		\draw[white] (studente) -- node[above, text=white] {(0,N)} (esame);
		\draw[white] (esame) -- node[above, text=white] {(1,1)} (corso);
		
		\node[draw=white, fill=black!50, text=white, ellipse, minimum width=1cm, minimum height=0.5cm] (data) [above=0.5cm of esame] {Data};
		\node[draw=white, fill=black!50, text=white, ellipse, minimum width=1cm, minimum height=0.5cm] (voto) [below=0.5cm of esame] {Voto};
		
		\draw[white] (esame) -- (data);
		\draw[white] (esame) -- (voto);
	\end{tikzpicture}
	\caption{Esempio di promozione della relazione Esame a entità debole (doppio bordo) con attributi propri.}
\end{figure}

\subsection{Attributi (Attribute)}
\begin{itemize}
	\item Una proprietà o caratteristica di un'entità o di una relazione.
	\item Collega ogni istanza dell'entità/relazione a un valore da un ``dominio'' (insieme di valori possibili).
	\item \textbf{Esempi:} \texttt{Nome} dell'entità \texttt{Studente}; \texttt{Data} della relazione \texttt{Esame}.
	\item \textbf{Rappresentazione Grafica:} Ovale.
	\item \textbf{Tipi:}
	\begin{itemize}
		\item \textbf{Semplici:} Atomici (es. \texttt{Età}).
		\item \textbf{Composti:} Possono essere scomposti in sotto-attributi (es. \texttt{Indirizzo} composto da \texttt{Via}, \texttt{NumeroCivico}, \texttt{Città}).
		\begin{itemize}
			\item \textit{Paragone Pratico (Composto):} In MongoDB è naturale: \texttt{address: \{ street: "...", city: "..." \}}. In SQL, spesso si ``appiattiscono'' in colonne separate (\texttt{Via}, \texttt{NumeroCivico}, \texttt{Citta}) o, se complesso, si mette in una tabella separata.
		\end{itemize}
	\end{itemize}
\end{itemize}

\subsection{Cardinalità (Cardinality)}
Specifica il numero minimo e massimo di istanze di un'entità che possono partecipare a una relazione, o il numero di valori che un attributo può assumere.
\begin{itemize}
	\item \textbf{Notazione comune:} \texttt{(min, max)}
	\begin{itemize}
		\item \texttt{min = 0}: partecipazione opzionale.
		\item \texttt{min = 1} (o più): partecipazione obbligatoria.
		\item \texttt{max = 1}: al massimo una.
		\item \texttt{max = N} (o \texttt{*}): molte.
	\end{itemize}
\end{itemize}

\subsubsection{Cardinalità delle Relazioni}
\begin{itemize}
	\item \textbf{Esempio:} \texttt{Impiegato (1,1)} --- \texttt{LavoraPer} --- \texttt{(0,N) Dipartimento}
	\begin{itemize}
		\item Un \texttt{Impiegato} deve lavorare per \textbf{esattamente un} \texttt{Dipartimento}.
		\item Un \texttt{Dipartimento} può avere \textbf{da zero a molti} \texttt{Impiegati}.
	\end{itemize}
	\item \textbf{Tipi comuni (basati su max):}
	\begin{itemize}
		\item \textbf{Uno-a-Uno (1:1):} Es. \texttt{Persona (0,1)} --- \texttt{Possiede} --- \texttt{(0,1) Pacemaker}.
		\item \textbf{Uno-a-Molti (1:N):} Es. \texttt{Cliente (1,1)} --- \texttt{Effettua} --- \texttt{(0,N) Ordine}.
		\item \textbf{Molti-a-Molti (M:N):} Es. \texttt{Studente (0,N)} --- \texttt{Frequenta} --- \texttt{(0,N) Corso}.
		\begin{itemize}
			\item \textit{Paragone Pratico (M:N):} In SQL, le relazioni M:N si implementano sempre con una tabella associativa intermedia. Prisma gestisce questo in modo più astratto.
		\end{itemize}
	\end{itemize}
\end{itemize}

\begin{figure}[h]
	\centering
	\begin{tikzpicture}[
		every node/.style={font=\small},
		entity/.style={rectangle, draw=white, fill=black!80, text=white, minimum width=2cm, minimum height=1cm},
		relationship/.style={diamond, draw=white, fill=black!70, text=white, minimum width=1.5cm, minimum height=1cm}
	]
		% 1:1 relationship
		\node[entity] (persona) {Persona};
		\node[relationship] (possiede) [right=2cm of persona] {Possiede};
		\node[entity] (pacemaker) [right=2cm of possiede] {Pacemaker};
		
		\draw[white] (persona) -- node[above, text=white] {(0,1)} (possiede);
		\draw[white] (possiede) -- node[above, text=white] {(0,1)} (pacemaker);
		
		% 1:N relationship
		\node[entity] (cliente) [below=2cm of persona] {Cliente};
		\node[relationship] (effettua) [right=2cm of cliente] {Effettua};
		\node[entity] (ordine) [right=2cm of effettua] {Ordine};
		
		\draw[white] (cliente) -- node[above, text=white] {(1,1)} (effettua);
		\draw[white] (effettua) -- node[above, text=white] {(0,N)} (ordine);
	\end{tikzpicture}
	\caption{Esempi di relazioni uno-a-uno (Persona-Pacemaker) e uno-a-molti (Cliente-Ordine).}
\end{figure}

\subsubsection{Cardinalità degli Attributi}
\begin{itemize}
	\item \texttt{(0,1)}: Attributo opzionale (può essere \texttt{NULL}). Es. \texttt{NumeroTelefonoSecondario}.
	\item \texttt{(1,1)}: Attributo obbligatorio, singolo valore (default). Es. \texttt{CodiceFiscale}.
	\item \texttt{(0,N)} o \texttt{(1,N)}: Attributo multivalore (un'entità può avere più valori per quell'attributo). Es. \texttt{NumeriTelefono} (una persona può avere più numeri).
	\begin{itemize}
		\item \textit{Paragone Pratico (Multivalore):} In SQL, si usa una tabella separata: \texttt{Persona(ID\_Persona)}, \texttt{NumeriTelefono(ID\_Persona\_FK, Numero)}. In MongoDB, si usa un array: \texttt{telefoni: ["123", "456"]}.
	\end{itemize}
\end{itemize}

\begin{figure}[h]
	\centering
	\begin{tikzpicture}[
		every node/.style={font=\small},
		entity/.style={rectangle, draw=white, fill=black!80, text=white, minimum width=2cm, minimum height=1cm},
		attribute/.style={ellipse, draw=white, fill=black!60, text=white, minimum width=1.5cm, minimum height=0.8cm},
		multivalue/.style={
			ellipse,
			draw=white,
			fill=black!50,
			text=white,
			minimum width=1.5cm,
			minimum height=0.8cm,
			path picture={
				\draw[white] 
				(path picture bounding box.center) 
				ellipse 
				({0.5*\pgfkeysvalueof{/pgf/minimum width}-0.05cm} and {0.5*\pgfkeysvalueof{/pgf/minimum height}-0.05cm});
			}
		}
	]
	
		\node[entity] (persona) {Persona};
		
		% Single value attributes
		\node[attribute] (cf) [above=1cm of persona] {CF};
		\node[attribute] (nome) [above right=0.5cm and 1cm of persona] {Nome};
		
		% Optional attribute
		\node[attribute] (tel2) [right=1cm of persona] {Tel2};
		
		% Multivalue attribute
		\node[multivalue] (tel) [below right=0.5cm and 1cm of persona] {Telefoni};
		
		\draw[white] (persona) -- (cf);
		\draw[white] (persona) -- (nome);
		\draw[white] (persona) -- (tel2);
		\draw[white] (persona) -- (tel);
		
		% Cardinality labels
		\node[text=white] at ($(cf)!0.5!(persona)$) [left] {(1,1)};
		\node[text=white] at ($(tel2)!0.5!(persona)$) [right] {(0,1)};
		\node[text=white] at ($(tel)!0.5!(persona)$) [right] {(0,N)};
	\end{tikzpicture}
	\caption{Esempi di attributi con diverse cardinalità: obbligatorio (CF), opzionale (Tel2), e multivalore (Telefoni).}
\end{figure}

\subsection{Identificatori (Chiavi - Keys)}
\begin{itemize}
	\item Un attributo o un insieme di attributi che identificano univocamente ogni istanza di un'entità.
	\item \textbf{Rappresentazione Grafica:} Attributo sottolineato.
	\item \textbf{Tipi:}
	\begin{itemize}
		\item \textbf{Identificatore Interno:} Formato da attributi della stessa entità.
		\begin{itemize}
			\item Es. \texttt{codiceFiscale} per l'entità \texttt{Persona}.
			\item \textit{Paragone Pratico:} \texttt{PRIMARY KEY} in SQL; \texttt{\_id} in MongoDB; \texttt{@id} in Prisma.
		\end{itemize}
		\item \textbf{Identificatore Esterno:} Formato da attributi dell'entità più l'identificatore di un'entità esterna a cui è collegata tramite una relazione con cardinalità \texttt{(1,1)} dal lato dell'entità da identificare. Usato per ``entità deboli'' che non possono esistere o essere identificate senza l'entità ``forte''.
		\begin{itemize}
			\item Es. \texttt{lineId} (attributo di \texttt{OrderItem}) + \texttt{orderId} (dall'entità \texttt{Order}) identifica univocamente un \texttt{OrderItem}. \texttt{OrderItem} è un'entità debole rispetto a \texttt{Order}.
		\end{itemize}
	\end{itemize}
	\item Ogni entità deve avere almeno un identificatore.
	\item Le relazioni di solito non hanno identificatori (se ne hanno bisogno, si promuovono a entità).
\end{itemize}

\begin{figure}[h]
	\centering
	\begin{tikzpicture}[
		every node/.style={font=\small},
		entity/.style={rectangle, draw=white, fill=black!80, text=white, minimum width=2cm, minimum height=1cm},
		weak/.style={
			rectangle, 
			draw=white, 
			fill=black!60, 
			text=white, 
			minimum width=2cm, 
			minimum height=1cm,
			path picture={
				\draw[white] 
				([shift={(0.05cm,0.05cm)}]path picture bounding box.south west) 
				rectangle 
				([shift={(-0.05cm,-0.05cm)}]path picture bounding box.north east);
			}
		},
		attribute/.style={ellipse, draw=white, fill=black!60, text=white, minimum width=1.5cm, minimum height=0.8cm},
		key/.style={attribute, draw=white, ultra thick}
	]
		% Strong entity with internal key
		\node[entity] (ordine) {Order};
		\node[key] (id) [above=1cm of ordine] {orderId};
		\draw[white, ultra thick] (ordine) -- (id);
		
		% Weak entity with external key
		\node[weak] (riga) [right=3cm of ordine] {OrderItem};
		\node[attribute] (num) [above=1cm of riga] {lineId};
		\node[attribute] (qty) [right=1cm of riga] {quantity};
		
		\draw[white] (riga) -- (num);
		\draw[white] (riga) -- (qty);
		\draw[white] (ordine) -- node[above, text=white] {(1,1)} (riga);
		
		% Dashed line for identifying relationship
		\draw[white, dashed] (id) -- (num);
	\end{tikzpicture}
	\caption{Esempio di entità forte (Order) con identificatore interno e entità debole (OrderItem) con identificatore esterno.}
\end{figure}

\begin{minted}{rust}
// Esempio con Prisma
model Order {
	orderId    Int         @id @default(autoincrement())
	// altri campi dell'ordine
	orderItems OrderItem[]
}

model OrderItem {
	orderId     Int
	lineId      Int
	quantity    Int
	order       Order  @relation(fields: [orderId], references: [orderId])

	@@id([orderId, lineId]) // Chiave primaria composta
}
\end{minted}

\begin{minted}{sql}
-- Definizione SQL dello schema
CREATE TABLE "Order" (
orderId    SERIAL PRIMARY KEY,
// altri campi dell'ordine
);

CREATE TABLE OrderItem (
orderId     INT,
lineId      INT,
quantity    INT,
PRIMARY KEY (orderId, lineId),
FOREIGN KEY (orderId) REFERENCES "Order"(orderId) ON DELETE CASCADE
);
\end{minted}

\subsection{Generalizzazione/Specializzazione (Inheritance)}
\begin{itemize}
	\item Una relazione tra un'entità genitore (superclasse, es. \texttt{Veicolo}) e una o più entità figlie (sottoclassi, es. \texttt{Automobile}, \texttt{Motocicletta}).
	\item Le figlie sono ``tipi di'' genitore: ereditano attributi e relazioni del genitore e possono averne di propri.
	\item \textbf{Rappresentazione Grafica:} Freccia (triangolo vuoto) dalle figlie al genitore.
	\item \textbf{Proprietà:}
	\begin{itemize}
		\item \textbf{Ereditarietà:} Le proprietà del genitore sono implicitamente presenti nelle figlie.
		\item \textbf{Copertura (Total/Partial):}
		\begin{itemize}
			\item \textbf{Totale:} Ogni istanza del genitore DEVE essere un'istanza di (almeno) una delle figlie. (Es. \texttt{Persona} -> \texttt{Maschio}, \texttt{Femmina}).
			\item \textbf{Parziale:} Un'istanza del genitore PUÒ essere un'istanza di una figlia (o solo del tipo genitore). (Es. \texttt{Veicolo} -> \texttt{Automobile}, \texttt{Motocicletta}).
		\end{itemize}
		\item \textbf{Disgiunzione (Disjoint/Overlapping):}
		\begin{itemize}
			\item \textbf{Disgiunta:} Un'istanza del genitore può essere al massimo un tipo di figlia. (Es. \texttt{Persona} è \texttt{Maschio} O \texttt{Femmina}).
			\item \textbf{Sovrapposta:} Un'istanza del genitore può essere più tipi di figlia contemporaneamente (raro e più complesso da modellare).
		\end{itemize}
		\item Di solito ci si concentra su generalizzazioni \textbf{Disgiunte (Totali o Parziali)}.
	\end{itemize}
	\item \textit{Paragone Pratico:}
	\begin{itemize}
		\item OOP: \texttt{class Veicolo \{\}}, \texttt{class Automobile extends Veicolo \{\}}.
		\item SQL: Ci sono diversi pattern per implementare l'ereditarietà.
		\item Prisma: Può essere modellato con campi discriminatori o modelli separati con relazioni.
	\end{itemize}
\end{itemize}

\begin{figure}[H]
	\centering
	\begin{tikzpicture}[
		every node/.style={font=\small},
		entity/.style={rectangle, draw=white, fill=black!80, text=white, minimum width=2cm, minimum height=1cm},
		subclass/.style={rectangle, draw=white, fill=black!70, text=white, minimum width=2cm, minimum height=1cm},
		attribute/.style={ellipse, draw=white, fill=black!60, text=white, minimum width=1.5cm, minimum height=0.8cm},
		generalization/.style={white, fill=white, draw=white, line width=1pt, shape=isosceles triangle, shape border rotate=90, isosceles triangle stretches=true, minimum height=0.6cm, minimum width=0.8cm}
	]
		% Parent class
		\node[entity] (veicolo) {Veicolo};
		\node[attribute] (targa) [above=1cm of veicolo] {Targa};
		\node[attribute] (anno) [right=1cm of veicolo] {Anno};
		
		% Subclasses
		\node[subclass] (auto) [below=2.5cm of veicolo, xshift=-2cm] {Automobile};
		\node[subclass] (moto) [below=2.5cm of veicolo, xshift=2cm] {Motocicletta};
		
		% Subclass attributes
		\node[attribute] (porte) [below=0.5cm of auto] {NumPorte};
		\node[attribute] (cilindrata) [below=0.5cm of moto] {Cilindrata};
		
		% Attributes connections
		\draw[white] (veicolo) -- (targa);
		\draw[white] (veicolo) -- (anno);
		\draw[white] (auto) -- (porte);
		\draw[white] (moto) -- (cilindrata);
		
		% Generalization triangle
		\node[generalization] (gen) at ($(veicolo.south) + (0,-0.5)$) {};
		
		% Lines from triangle to subclasses
		\draw[white] (gen.south) -- ($(gen.south) + (0,-0.2)$) -| (auto);
		\draw[white] (gen.south) -- ($(gen.south) + (0,-0.2)$) -| (moto);
		
		% Line from superclass to triangle
		\draw[white] (veicolo) -- (gen);
	\end{tikzpicture}
	\caption{Esempio di generalizzazione/specializzazione: Veicolo come superclasse e Automobile/Motocicletta come sottoclassi con simbolo di ereditarietà (triangolo vuoto).}
\end{figure}

\section{Documentazione}
\begin{itemize}
	\item \textbf{Dizionario dei Dati:} Descrive in dettaglio ogni entità, relazione e attributo.
	\item \textbf{Vincoli Non Esprimibili:} Alcuni vincoli non possono essere rappresentati graficamente nell'ERD (es. ``Lo stipendio di un impiegato non può superare quello del suo manager''). Vanno documentati a parte.
	\begin{itemize}
		\item \textit{Paragone Pratico:} Questi vincoli si implementano spesso con \texttt{CHECK constraints} in SQL, \texttt{triggers}, o a livello applicativo.
	\end{itemize}
\end{itemize}

\section{UML (Unified Modeling Language) come Alternativa}
\begin{itemize}
	\item UML è un linguaggio di modellazione più ampio, usato per vari aspetti dello sviluppo software.
	\item Per la modellazione dei dati, si usano principalmente i \textbf{Diagrammi delle Classi (Class Diagrams)}.
	\item Molti concetti ER hanno un equivalente in UML:
	\begin{itemize}
		\item \textbf{Entità} -> \textbf{Classe}
		\item \textbf{Relazione} -> \textbf{Associazione}
		\item \textbf{Relazione con attributi} -> \textbf{Classe di Associazione}
		\item \textbf{Cardinalità:} \texttt{1}, \texttt{0..1}, \texttt{*}, \texttt{1..*}
		\item \textbf{Identificatori:} \texttt{\{id\}} accanto all'attributo.
		\item \textbf{Generalizzazione/Specializzazione:} Freccia con triangolo vuoto verso la superclasse.
		\item \textbf{Concetti specifici UML:} Aggregazione (rombo vuoto), Composizione (rombo pieno).
	\end{itemize}
\end{itemize}


% --- Assicurati che questi pacchetti siano nel tuo preambolo ---
% \usepackage{graphicx}
% \usepackage{float} % Per l'opzione [H] delle figure
% \usepackage[section]{placeins} % Opzionale, per un controllo più stretto del float
% -------------------------------------------------------------

% --- Continua dalla sezione precedente ---

\section{Modellazione Concettuale con UML (Unified Modeling Language)}

UML è un linguaggio di modellazione standardizzato e ampiamente utilizzato per specificare, visualizzare, costruire e documentare gli artefatti di un sistema software. Sebbene l'ERD sia specifico per i database, UML offre un approccio più generale e può essere utilizzato anche per la modellazione concettuale dei dati, principalmente attraverso i \textbf{Diagrammi delle Classi (Class Diagrams)}.

\subsection{Classi (Classes)}
In UML, un'entità del modello ER corrisponde a una \textbf{Classe}. Una classe è rappresentata come un rettangolo, tipicamente diviso in tre sezioni:
\begin{enumerate}
		\item \textbf{Nome della Classe:} In alto, in grassetto.
		\item \textbf{Attributi (Attributes):} Al centro, elencano le proprietà della classe.
		\item \textbf{Operazioni (Operations/Methods):} In basso (spesso omessa nella modellazione concettuale dei dati puri, poiché ci si concentra sulla struttura).
\end{enumerate}

\begin{figure}[H]
		\centering
		\includegraphics[width=0.8\textwidth]{uml_images/fig_classes_employee_project.png}
		\caption{Esempio di Classi UML: Employee e Project.}
		\label{fig:uml_classes_employee_project}
\end{figure}

\subsection{Associazioni (Associations)}
Le relazioni del modello ER sono chiamate \textbf{Associazioni} in UML. Un'associazione rappresenta una relazione semantica tra due o più classi. È disegnata come una linea continua che connette le classi.
\begin{itemize}
		\item \textbf{Nome dell'Associazione (opzionale):} Può essere scritto vicino alla linea, spesso con una freccetta che indica la direzione di lettura (se il nome è un verbo).
		\item \textbf{Ruoli (opzionale):} Nomi posti alle estremità della linea di associazione per chiarire il ruolo che una classe gioca nell'associazione.
		\item \textbf{Molteplicità (Multiplicity):} Equivalente alla cardinalità ER, indica quante istanze di una classe possono essere collegate a un'istanza dell'altra classe. Notazioni comuni:
		\begin{itemize}
				\item \texttt{1} (esattamente uno)
				\item \texttt{0..1} (zero o uno)
				\item \texttt{*} (zero o molti)
				\item \texttt{1..*} (uno o molti)
				\item \texttt{m..n} (da m a n)
		\end{itemize}
\end{itemize}

\begin{figure}[H]
		\centering
		\includegraphics[width=0.9\textwidth]{uml_images/fig_associations_examples.png}
		\caption{Esempi di Associazioni UML con nomi, ruoli e molteplicità.}
		\label{fig:uml_associations_examples}
\end{figure}

\subsection{Classe di Associazione (Association Class)}
Quando un'associazione stessa ha attributi o operazioni, può essere modellata come una \textbf{Classe di Associazione}. È rappresentata come una classe normale collegata da una linea tratteggiata all'associazione che descrive.
\textit{Esempio:} L'associazione "Sostiene Esame" tra \texttt{Student} e \texttt{Course} può avere attributi come \texttt{Date} e \texttt{Degree}. \texttt{Exam} diventa una classe di associazione.

\begin{figure}[H]
		\centering
		\includegraphics[width=0.6\textwidth]{uml_images/fig_association_class_exam.png}
		\caption{Esempio di Classe di Associazione UML: Exam.}
		\label{fig:uml_association_class_exam}
\end{figure}

\subsection{Associazione N-aria (N-ary Association) e Reificazione}
Un'associazione può coinvolgere più di due classi (ternaria, quaternaria, ecc.). Graficamente, si usa un rombo (come in ER) a cui sono collegate le classi. Se l'associazione n-aria ha attributi, una classe di associazione viene collegata al rombo.
Le associazioni n-arie (n > 2) sono spesso complesse da gestire e interpretare. Una pratica comune è la \textbf{reificazione} (o "promozione a classe"): l'associazione n-aria viene trasformata in una nuova classe regolare, che viene poi collegata alle classi originariamente coinvolte tramite associazioni binarie.

\begin{figure}[H]
		\centering
		\includegraphics[width=0.45\textwidth]{uml_images/fig_ternary_association.png}\hfill
		\includegraphics[width=0.45\textwidth]{uml_images/fig_reification.png}
		\caption{Associazione Ternaria con classe di associazione (a, sinistra) e sua Reificazione (b, destra) in UML.}
		\label{fig:uml_nary_reification}
\end{figure}

\subsection{Aggregazione e Composizione (Aggregation and Composition)}
Sono tipi speciali di associazione che rappresentano relazioni "parte-di" (whole-part).
\begin{itemize}
		\item \textbf{Aggregazione (Aggregation):} Rappresenta una relazione "ha-un" debole. Le parti possono esistere indipendentemente dal tutto. È indicata da un \textbf{rombo vuoto} dal lato del "tutto" (aggregato).
		\textit{Esempio:} Un \texttt{Team} è composto da \texttt{Technician}. Un tecnico può esistere anche se il team viene sciolto, o può appartenere a più team (a seconda della molteplicità).
		\item \textbf{Composizione (Composition):} Rappresenta una relazione "ha-un" forte. Le parti dipendono esistenzialmente dal tutto; se il tutto viene distrutto, anche le parti lo sono. È indicata da un \textbf{rombo pieno} dal lato del "tutto" (composito). La molteplicità dal lato del composito verso la parte è solitamente \texttt{1} o \texttt{0..1} (una parte appartiene a un solo tutto).
		\textit{Esempio:} Un'automobile (\texttt{Car}) è composta da un motore (\texttt{Engine}). Se l'automobile viene rottamata, anche il suo motore specifico (come parte di quell'auto) cessa di esistere in quel contesto. Una \texttt{Agency} è parte di una \texttt{Firm}.
\end{itemize}

\begin{figure}[H]
		\centering
		\includegraphics[width=0.7\textwidth]{uml_images/fig_aggregation_composition.png}
		\caption{Esempio di Aggregazione (Team-Technician) e Composizione (Firm-Agency) in UML.}
		\label{fig:uml_aggregation_composition}
\end{figure}

\subsection{Identificatori (Identifiers)}
In UML, gli attributi che compongono l'identificatore (chiave primaria) di una classe possono essere contrassegnati con la proprietà \texttt{\{id\}} o talvolta sottolineati (anche se \texttt{\{id\}} è più comune in UML2).

\begin{figure}[H]
		\centering
		\includegraphics[width=0.8\textwidth]{uml_images/fig_identifiers.png}
		\caption{Esempio di Identificatori UML con \texttt{\{id\}}.}
		\label{fig:uml_identifiers}
\end{figure}

\subsection{Identificatore Esterno (External Identifier) e Associazioni Qualificate}
Un concetto simile all'identificatore esterno ER si ha quando l'identità di una classe (la "debole") dipende parzialmente da un'altra classe attraverso un'associazione. Questo può essere indicato con:
\begin{itemize}
		\item Uno \textbf{stereotipo} sull'associazione, come \texttt{<<identifying>>} (come suggerito nelle slide del prof., anche se non standard UML stretto per questo specifico caso).
		\item Una \textbf{Associazione Qualificata}: un piccolo rettangolo (il qualificatore) è attaccato alla classe "forte", contenente un attributo della classe "debole" che, insieme all'istanza della classe forte, identifica univocamente l'istanza della classe debole.
		\item La molteplicità dal lato della classe "debole" verso la "forte" è spesso \texttt{1} in un'associazione identificante.
\end{itemize}
Nelle slide del Prof. Montesi (slide 100), si usa lo stereotipo \texttt{<<identifiying>>} (probabile typo per \texttt{<<identifying>>}) sull'associazione e \texttt{\{id\}} sugli attributi che compongono la chiave, inclusi quelli della classe "debole" e quelli ereditati implicitamente tramite l'associazione identificante.

\begin{figure}[H]
		\centering
		\includegraphics[width=0.9\textwidth]{uml_images/fig_identifying_association.png}
		\caption{Esempio di associazione identificante con stereotipo (come da slide 100).}
		\label{fig:uml_identifying_association}
\end{figure}
\textit{Nota:} L'attributo \texttt{Number} di \texttt{Student} è \texttt{\{id\}} nel contesto dell'associazione con \texttt{University}. La vera chiave univoca di \texttt{Student} sarebbe una combinazione di \texttt{Student.Number} e l'identificatore di \texttt{University}.

\subsection{Generalizzazione (Generalization)}
Corrisponde alla generalizzazione/specializzazione del modello ER e rappresenta una relazione "è-un-tipo-di" (is-a-kind-of) tra una classe più generale (superclasse) e una classe più specifica (sottoclasse).
\begin{itemize}
		\item \textbf{Rappresentazione Grafica:} Una linea continua con una \textbf{grande freccia triangolare vuota} che punta dalla sottoclasse alla superclasse.
		\item \textbf{Ereditarietà:} La sottoclasse eredita attributi, operazioni e associazioni della superclasse.
		\item \textbf{Vincoli:} Simili a ER, si possono specificare vincoli come:
		\begin{itemize}
				\item \texttt{\{complete, disjoint\}} o \texttt{\{total, disjoint\}}: Ogni istanza della superclasse è esattamente una delle sottoclassi.
				\item \texttt{\{incomplete, disjoint\}} o \texttt{\{partial, disjoint\}}: Un'istanza della superclasse può essere una delle sottoclassi o nessuna di esse (solo la superclasse), ma non più di una.
				\item \texttt{\{overlapping\}}: Una sottoclasse può essere istanza di più sottoclassi (più raro).
		\end{itemize}
		Questi vincoli sono scritti vicino alla freccia di generalizzazione.
\end{itemize}

\begin{figure}[H]
		\centering
		\includegraphics[width=\textwidth]{uml_images/fig_generalization.png}
		\caption{Esempio di Generalizzazione UML (basato su slide 101).}
		\label{fig:uml_generalization}
\end{figure}

\subsection{Esempio Complessivo: Schema Concettuale in UML}
La slide 102 del Prof. Montesi mostra un diagramma ER tradotto in UML Class Diagram.
Notiamo:
\begin{itemize}
		\item Le entità diventano Classi (\texttt{Employee}, \texttt{Dept}, \texttt{Project}, \texttt{Office}).
		\item Le relazioni diventano Associazioni (\texttt{Management}, \texttt{Affiliation}, \texttt{Attendance}).
		\item \texttt{Affiliation} è una classe di associazione perché ha l'attributo \texttt{Date}.
		\item \texttt{Composition} tra \texttt{Dept} e \texttt{Office} è una composizione forte (rombo pieno) e identificante (\texttt{<<identifying>>}).
		\item Le cardinalità ER sono tradotte in molteplicità UML.
		\item Gli identificatori sono marcati con \texttt{\{id\}}.
		\item Una nota è attaccata alla classe \texttt{Project}.
\end{itemize}

\begin{figure}[H]
\centering
\includegraphics[width=\textwidth]{uml_images/fig_overall_example_slide102.png}
\caption{Diagramma concettuale UML basato sulla slide 102.}
\label{fig:uml_overall_example_slide102}
\end{figure}

\textit{Conclusione:} UML offre un set ricco di costrutti per la modellazione concettuale dei dati, con una notazione leggermente diversa ma concettualmente simile a ER. La scelta tra ER e UML Class Diagrams spesso dipende dalle convenzioni del team o dalla necessità di integrare il modello dati con altri modelli UML del sistema.

