%%%%%%%%%%%%%%%%%%%%%%%%%%%%%%%%%%%%%%%%%%%%%%%%%%%%%%%%%%%%%%%%%%%%%%%%%%%%%%%
% PREAMBOLO COMUNE PER APPUNTI (Stile Scuro)
%
% Questo file contiene tutte le impostazioni e i pacchetti comuni.
% NON contiene \begin{document} o \end{document}.
%
% Istruzioni per la compilazione del file principale:
% pdflatex -shell-escape nomefile_principale.tex
%%%%%%%%%%%%%%%%%%%%%%%%%%%%%%%%%%%%%%%%%%%%%%%%%%%%%%%%%%%%%%%%%%%%%%%%%%%%%%%

\documentclass{article}

% --- Encoding e lingua ---
\usepackage[utf8]{inputenc}
\usepackage[italian]{babel}

% --- Margini e layout ---
\usepackage{geometry}
\geometry{a4paper, margin=1in}

% --- Font sans-serif (come Helvetica) ---
\usepackage[scaled]{helvet}
\renewcommand{\familydefault}{\sfdefault}
\usepackage[T1]{fontenc}

% --- Matematica ---
\usepackage{amsmath}
\usepackage{amssymb}

% --- Liste personalizzate ---
\usepackage{enumitem}
% \setlist{nosep}

% --- Immagini e Grafica ---
\usepackage{float}
% \usepackage{graphicx}
\usepackage{tikz}
\usetikzlibrary{shapes.geometric, positioning, calc}

% --- Tabelle Avanzate ---
\usepackage{array}
\usepackage{booktabs}
\usepackage{longtable}

% --- Hyperlink e Metadati PDF ---
\usepackage{hyperref}
\hypersetup{
    colorlinks=true,
    linkcolor=white,
    filecolor=magenta,
    urlcolor=cyan,
    citecolor=green,
    % pdftitle, pdfauthor, ecc. verranno impostati nel file principale
    pdfpagemode=FullScreen,
    bookmarksopen=true,
    bookmarksnumbered=true
}

% --- Colori e Sfondo Nero ---
\usepackage{xcolor}
\pagecolor{black}
\color{white}

% --- Evidenziazione del Codice ---
\usepackage{minted}
\setminted{
    frame=lines,
    framesep=2mm,
    fontsize=\small,
    breaklines=true,
    style=monokai,
    bgcolor=black!80
}
\usemintedstyle{monokai}

% --- Comandi Personalizzati (Opzionali) ---
% \newcommand{\Rel}[1]{\textit{#1}}
% ... (altri comandi se necessario) ...

% NESSUN \title, \author, \date, \begin{document} o \end{document} QUI


% --- Hyperlink ---
\usepackage{hyperref}
\hypersetup{
	colorlinks=true,
	linkcolor=white,
	filecolor=magenta,
	urlcolor=cyan,
	pdftitle={Architettura, Transazioni, Concorrenza e Ripristino nei DBMS Relazionali},
	pdfauthor={Alessandro Amella},
	pdfpagemode=FullScreen,
}

% --- Titolo ---
\title{Architettura, Transazioni, Concorrenza e Ripristino nei DBMS Relazionali\\
  \large Appunti basati sulle dispense del Dott. Magistrale Flavio Bertini}
\author{Alessandro Amella, Gemini e Claude}
\date{\today}

\begin{document}

\maketitle
\tableofcontents
\newpage

\section{Architettura di un DBMS}

\subsection{Cos'è un DataBase Management System (DBMS)?}
Un DBMS è un software progettato per \textbf{creare e gestire database}. Il suo scopo principale è fornire un modo \textbf{efficiente e persistente} per creare, recuperare, aggiornare e gestire enormi quantità di dati.
\begin{itemize}
    \item \textbf{Efficienza:} Deve rispondere rapidamente alle interrogazioni.
    \item \textbf{Persistenza:} I dati devono sopravvivere allo spegnimento del sistema o a crash.
    \item \textbf{Gestione di grandi quantità di dati:} Ottimizzato per moli di dati che non entrerebbero in memoria RAM.
\end{itemize}

\textit{Esempio pratico:} Pensa a Prisma (per database SQL) o Mongoose (per MongoDB). Quando esegui un comando come:
\begin{minted}{javascript}
// Con Prisma
const users = await prisma.user.findMany();

// Con Mongoose
const users = await User.find();
\end{minted}
è il DBMS sottostante (es. PostgreSQL, MySQL, MongoDB Server) che esegue il lavoro pesante di trovare i dati sul disco, filtrarli e restituirli.

\subsection{Implementare un DBMS Relazionale}
Se volessimo costruire un DBMS relazionale da zero, dovrebbe essere in grado di:
\begin{enumerate}
    \item \textbf{Memorizzare relazioni come tabelle:} Ad esempio, tabelle $R(A,B,C)$, $S(A,B)$, $T(A,B)$. Ogni riga è una tupla (record), ogni colonna un attributo.
    \item \textbf{Eseguire query SQL:} Come:
    \begin{minted}{sql}
SELECT A, B FROM R WHERE B = 2000;
    \end{minted}
    o query più complesse con JOIN:
    \begin{minted}{sql}
SELECT R.A, T.B FROM R, S, T WHERE R.B = S.A AND S.B = T.A;
    \end{minted}
\end{enumerate}

\subsection{Funzioni Chiave di un DBMS}
Un DBMS deve gestire efficientemente diverse operazioni e scenari:
\begin{itemize}
    \item Ricerca e aggiornamento tuple.
    \item Ordine di esecuzione delle query SQL (ottimizzazione).
    \item Accessi ripetuti alle stesse locazioni su disco (caching).
    \item Operazioni concorrenti di lettura e scrittura (Controllo della Concorrenza).
    \item Controllo degli accessi (permessi).
    \item Disaster recovery (Ripristino dopo guasti).
    \item API (Application Programming Interfaces) per l'interazione con le applicazioni.
\end{itemize}

\subsection{Architettura di un DBMS Relazionale (Componenti Principali)}
L'architettura di un DBMS è complessa e modulare. I componenti principali possono essere raggruppati in tre macro-aree:

\begin{enumerate}
    \item \textbf{Query Processor (Processore delle Query):} Responsabile della trasformazione e dell'esecuzione delle query.
    \begin{itemize}
        \item \textbf{DDL Compiler:} Interpreta comandi DDL (Data Definition Language, es. \texttt{CREATE TABLE}, \texttt{ALTER TABLE}) e aggiorna i metadati (lo schema del database).
        \item \textbf{Query Compiler:}
        \begin{itemize}
            \item \textit{Parsing:} Analisi sintattica della query SQL. Costruisce un albero di parsing.
            \item \textit{Preprocessing (Validazione Semantica):} Verifica che tabelle e colonne esistano, tipi di dati compatibili. Trasforma l'albero in un albero di operazioni algebriche (algebra relazionale).
            \item \textit{Optimization:} Trasforma la query originale nella sequenza di operazioni più efficiente. Decide l'ordine dei join, l'uso degli indici, ecc.
        \end{itemize}
        \item \textbf{Execution Engine (Motore di Esecuzione):} Esegue il piano di query scelto dall'ottimizzatore, interagendo con gli altri componenti.
    \end{itemize}

    \item \textbf{Resource Manager (Gestore delle Risorse):} Gestisce l'accesso ai dati su disco e in memoria.
    \begin{itemize}
        \item \textbf{Storage Manager (Gestore della Memoria Secondaria):} Tiene traccia della posizione dei file sul disco. Recupera blocchi di dati dal disco.
        \item \textbf{Buffer Manager (Gestore del Buffer):} Gestisce una porzione della memoria RAM (il buffer pool) per memorizzare temporaneamente le pagine di dati lette dal disco. Se una pagina richiesta è già nel buffer (cache hit), l'accesso è veloce. Altrimenti (cache miss), la pagina viene caricata dal disco.
        \item \textbf{Index/File/Record Manager:} Conosce lo schema del database e le strutture dati (come indici B-Tree) per un accesso efficiente ai dati.
    \end{itemize}

    \item \textbf{Transaction Manager (Gestore delle Transazioni):} Assicura le proprietà ACID delle transazioni.
    \begin{itemize}
        \item \textbf{Transaction Manager (componente specifico):} Supervisiona le transazioni.
        \item \textbf{Logging and Recovery Manager:}
        \begin{itemize}
            \item \textit{Logging:} Registra tutte le modifiche al database in un file di log (Write-Ahead Logging - WAL).
            \item \textit{Recovery:} In caso di crash, usa il log per ripristinare il database a uno stato consistente.
        \end{itemize}
        \item \textbf{Concurrency Control Manager:} Gestisce l'accesso concorrente ai dati da parte di più transazioni, usando meccanismi come i lock. Mantiene una \textbf{Lock Table}.
    \end{itemize}
\end{enumerate}
Gli utenti e le applicazioni interagiscono con il DBMS, così come l'amministratore per le operazioni DDL.

\subsection{Il Percorso di una Query (Esempio Semplificato)}
Consideriamo la query:
\begin{minted}{sql}
SELECT A, B FROM R1, R2 WHERE C = D AND C = 'c';
\end{minted}

\begin{enumerate}
    \item \textbf{Query Compiler:}
    \begin{itemize}
        \item Parsing e validazione.
        \item Trasformazione in algebra relazionale (forma logica), ad esempio:\\
        $\pi_{A,B} ( \sigma_{C=D \land C='c'} (R1 \bowtie R2) )$
        \item Ottimizzazione (piano di esecuzione fisico), ad esempio, potrebbe decidere di eseguire:\\
        $\pi_{A,B}(\sigma_{C=D}( (\sigma_{C='c'}(R1)) \bowtie R2 ))$\\
        (prima si filtra $R1$, poi si fa il join con $R2$, poi si applica l'altro filtro e infine la proiezione).
    \end{itemize}
    
    \item \textbf{Execution Engine:}
    \begin{itemize}
        \item Richiede al Resource Manager i blocchi di $R1$ necessari per calcolare $\sigma_{C='c'}(R1)$.
        \item Il \textbf{Resource Manager} (tramite Storage Manager e Buffer Manager) carica i blocchi di $R1$ dal disco al buffer.
        \item L'Execution Engine (o l'Index/File/Record Manager) applica il filtro $C='c'$ sui dati di $R1$ nel buffer. Il risultato intermedio resta nel buffer.
        \item Richiede i blocchi di $R2$. Vengono caricati nel buffer.
        \item Esegue il join tra il risultato intermedio di $R1$ e $R2$, applica il filtro $C=D$ e la proiezione $\pi_{A,B}$, operando sui dati nel buffer.
        \item Restituisce il risultato finale.
    \end{itemize}
\end{enumerate}
Durante tutte le operazioni di modifica, il Transaction Manager è coinvolto per il logging e il controllo della concorrenza.

\section{Transazioni e Controllo della Concorrenza}

\subsection{Transazioni}
Le operazioni su un database raramente coinvolgono una singola istruzione SQL. Spesso, una singola "azione logica" richiede una serie di query.
\textit{Esempio:} Un trasferimento bancario da conto A123 a B456 di 100€.
\begin{minted}{sql}
START TRANSACTION;
UPDATE Conto SET Saldo = Saldo - 100 WHERE NumeroConto = 'A123';
UPDATE Conto SET Saldo = Saldo + 100 WHERE NumeroConto = 'B456';
COMMIT; -- oppure ROLLBACK in caso di errore
\end{minted}
Queste due operazioni devono avvenire insieme. O entrambe hanno successo, o nessuna delle due. Questo gruppo di operazioni è una \textbf{transazione}. Una transazione è l'esecuzione di un programma utente (una o più operazioni di lettura/scrittura) vista dal DBMS come un'unità atomica.

\textbf{Problemi principali da risolvere:}
\begin{enumerate}
    \item Esecuzione Concorrente di Transazioni.
    \item Crash Recovery (Ripristino dopo guasti).
\end{enumerate}

\subsection{Motivazioni per la Concorrenza}
\begin{itemize}
    \item Gli accessi al disco sono lenti; la CPU non dovrebbe attendere passivamente.
    \item Per migliorare le prestazioni, più transazioni vengono eseguite in modo \textbf{concorrente} (i loro passi si mescolano, \textit{interleaving}), mantenendo la CPU occupata.
    \item \textbf{Problema:} L'utente non deve percepire questa "mescolanza". Ogni transazione deve apparire come se fosse eseguita da sola, in isolamento.
\end{itemize}

\subsection{Motivazioni per il Crash Recovery}
\begin{itemize}
    \item Una transazione potrebbe non terminare come previsto a causa di un evento imprevisto (errore software, crash hardware).
    \item \textbf{Problema:} Il DBMS deve assicurare che le altre transazioni non siano affette dall'esecuzione incorretta e che il database rimanga in uno \textbf{stato consistente}.
\end{itemize}

\subsection{Proprietà ACID delle Transazioni}
Per garantire un'esecuzione sicura e concorrente, ogni transazione deve avere le seguenti proprietà:
\begin{itemize}
    \item \textbf{A - Atomicity (Atomicità):} Una transazione è un'unità atomica. O tutte le sue operazioni vengono eseguite, o nessuna. Non ci deve essere uno stato intermedio parziale persistente. In caso di fallimento, le modifiche parziali vengono annullate (rollback).
    \item \textbf{C - Consistency (Consistenza):} Una transazione porta il database da uno stato consistente a un altro stato consistente. Le \textit{regole di integrità} del database (es. chiavi primarie, foreign key, vincoli \texttt{CHECK}) devono essere soddisfatte alla fine della transazione.
    \item \textbf{I - Isolation (Isolamento):} Ogni transazione deve apparire come se fosse eseguita in isolamento, senza interferenze da altre transazioni concorrenti. Il risultato finale dell'esecuzione concorrente deve essere equivalente a quello che si otterrebbe eseguendole una dopo l'altra in una qualche sequenza (serializzazione).
    \item \textbf{D - Durability (Durabilità):} Una volta che una transazione è stata "commessa" (\texttt{COMMIT}), le sue modifiche sono permanenti e devono sopravvivere a fallimenti successivi.
\end{itemize}

\subsection{Schedule (Pianificazione)}
\begin{itemize}
    \item Una transazione è una sequenza di operazioni di lettura (read) e scrittura (write) su oggetti del database.
    \item Ogni transazione specifica la sua azione finale: \texttt{commit} (successo) o \texttt{abort} (fallimento, con annullamento delle azioni).
    \item Uno \textbf{schedule} è una sequenza di azioni (read, write, commit, abort) prese da un insieme di transazioni, preservando l'ordine delle azioni \textit{all'interno} di ciascuna transazione.
    \item \textbf{Schedule Seriale:} Le transazioni sono eseguite una dopo l'altra, senza interleaving.
\end{itemize}

\subsection{Schedule Serializzabile}
Uno schedule è \textbf{serializzabile} se il suo effetto su un'istanza consistente del database è \textbf{identico} a quello di un qualche schedule seriale completo. Questo è l'obiettivo del controllo di concorrenza: permettere l'interleaving per le performance, garantendo la correttezza.

\subsection{Anomalie dell'Esecuzione Interleaved (Conflitti)}
L'interleaving non controllato può portare a stati inconsistenti. Tipi di conflitti (secondo le slide):
\begin{itemize}
    \item \textbf{Write-Read Conflict (Dirty Read):} $T_2$ legge un oggetto scritto da $T_1$, ma $T_1$ non ha ancora fatto \texttt{commit}. Se $T_1$ fa \texttt{abort}, $T_2$ ha letto dati "sporchi".
    \item \textbf{Read-Write Conflict (Unrepeatable Read):} $T_2$ scrive un oggetto letto da $T_1$, prima che $T_1$ faccia \texttt{commit}. Se $T_1$ legge di nuovo lo stesso oggetto, ottiene un valore diverso.
    \item \textbf{Write-Write Conflict (Lost Update):} $T_2$ scrive un oggetto anch'esso scritto da $T_1$. Una delle due scritture potrebbe andare persa.
    \item \textbf{Phantom Anomalies (Anomalie Fantasma):} Una transazione esegue una query che restituisce un insieme di righe. Un'altra transazione inserisce o cancella righe che soddisfano i criteri della query. Se la prima transazione riesegue la query, ottiene un insieme diverso di righe ("fantasmi" appaiono o scompaiono).
\end{itemize}

\subsection{Livelli di Isolamento delle Transazioni}
SQL definisce diversi livelli di isolamento, che offrono un compromesso tra correttezza e performance.
\begin{itemize}
    \item \textbf{READ UNCOMMITTED:} Permette Dirty Reads, Unrepeatable Reads, Phantoms. Massime performance, minima consistenza.
    \item \textbf{READ COMMITTED:} Previene Dirty Reads. Permette Unrepeatable Reads, Phantoms. La transazione legge solo dati committati.
    \item \textbf{REPEATABLE READS:} Previene Dirty Reads, Unrepeatable Reads. Permette Phantoms. Garantisce che rileggendo un dato si ottenga lo stesso valore, ma nuove righe ("fantasmi") possono apparire.
    \item \textbf{SERIALIZABLE:} Previene Dirty Reads, Unrepeatable Reads, Phantoms. Massima consistenza.
\end{itemize}

\textbf{Tabella Riepilogativa (da slide 26):}
\begin{center}
\begin{tabular}{|l|c|c|c|c|}
\hline
\textbf{Level} & \textbf{Dirty read (Write-Read)} & \textbf{Lost update (Read-Write)} & \textbf{Unrepeatable read (Write-Write)} & \textbf{Phantom} \\ \hline
READ UNCOMMITTED   & may occur & may occur         & may occur               & may occur \\ \hline
READ COMMITTED     & don't occur& may occur         & may occur               & may occur \\ \hline
REPEATABLE READS   & don't occur& don't occur       & don't occur             & may occur \\ \hline
SERIALIZABLE       & don't occur& don't occur       & don't occur             & don't occur \\ \hline
\end{tabular}
\end{center}
\textit{Nota: Le definizioni dei conflitti tra parentesi nella tabella sopra sono riportate come dalle slide. Potrebbero esserci delle interpretazioni diverse della terminologia standard dei conflitti "Lost Update" e "Unrepeatable Read" rispetto ai tipi di interazione Read/Write indicati.}

\subsection{Approcci al Controllo della Concorrenza}
\begin{itemize}
    \item \textbf{Restrittivo (Pessimistico):} Previene i conflitti prima che accadano, tipicamente usando \textbf{lock}.
    \begin{itemize}
        \item \textbf{Strict Two-Phase Locking (Strict 2PL):}
        \begin{enumerate}
            \item Fase di crescita: La transazione acquisisce lock (condivisi S per lettura, esclusivi X per scrittura).
            \item Fase di decrescita: La transazione rilascia i lock. In 2PL "stretto", tutti i lock vengono rilasciati solo al \texttt{commit} o \texttt{abort}. Una volta rilasciato un lock, non se ne possono acquisire altri.
        \end{enumerate}
        Garantisce la serializzabilità. Il Lock Manager gestisce i lock in una Lock Table.
    \end{itemize}
    \item \textbf{Ottimistico:} Assume conflitti rari. Le transazioni procedono senza lock. Prima del \texttt{commit}, una fase di \textbf{validazione} verifica conflitti. Se presenti, la transazione abortisce e riparte.
    Fasi: Read (in area privata), Validation, Write (se validazione OK).
    \item \textbf{Timestamping:} Ad ogni transazione è assegnato un timestamp. Il DBMS usa i timestamp per imporre un ordine seriale. Se un'operazione viola l'ordine, la transazione abortisce.
\end{itemize}

\subsection{Deadlock (Stallo)}
Con il locking pessimistico, può verificarsi un deadlock: $T_1$ ha lock su $A$ e aspetta $B$, $T_2$ ha lock su $B$ e aspetta $A$.
\begin{itemize}
    \item \textbf{Prevenzione Deadlock:} Assegnare priorità (es. timestamp) e usare politiche come Wait-Die o Wound-Wait.
    \item \textbf{Rilevamento e Risoluzione Deadlock:} Permettere i deadlock, rilevarli (es. con un waits-for graph) e risolverli abortendo una transazione ("vittima"). Oppure, usare un timeout.
\end{itemize}

\subsection{Crash Recovery}
Il Logging and Recovery Manager deve assicurare Atomicità e Durabilità.
\textbf{ARIES (Advanced Recovery and Integrated Extraction System):} Algoritmo di recovery comune.
\begin{itemize}
    \item \textbf{Principi ARIES:}
    \begin{enumerate}
        \item \textbf{Write-Ahead Logging (WAL):} Modifiche scritte nel log \textit{prima} della scrittura della pagina dati su disco. Record di log con info per Undo/Redo.
        \item \textbf{Repeating History During Redo:} Al riavvio dopo crash, ARIES riapplica modifiche dal log (da ultimo checkpoint) per riportare DB allo stato del crash (incluse transazioni non committate).
        \item \textbf{Logging Changes During Undo:} L'annullamento di un'operazione è loggato con un \textbf{Compensation Log Record (CLR)}. I CLR non vengono mai annullati.
    \end{enumerate}
    \item \textbf{Log File:} Traccia tutte le azioni. Record con \textbf{Log Sequence Number (LSN)} univoco.
    Record di Log: \texttt{<LSN, TransactionID, PageID, RedoInfo, UndoInfo, PreviousLSN\_for\_this\_TX>}.
    \item \textbf{Strutture Dati per Logging (in memoria):}
    \begin{itemize}
        \item \textbf{Dirty Page Table (DPT):} Pagine modificate in buffer ma non su disco.
        \item \textbf{Transaction Table:} Transazioni attive, stato, ultimo LSN.
    \end{itemize}
    \item \textbf{Checkpoint (in ARIES è "Fuzzy"):} Snapshot dello stato per ridurre tempo di recovery.
    \begin{enumerate}
        \item Scrive record \texttt{begin\_checkpoint}.
        \item Scrive record \texttt{end\_checkpoint} con DPT e Transaction Table.
        \item Scrive LSN del \texttt{begin\_checkpoint} in un Master Record su disco.
    \end{enumerate}
    Non interrompe le operazioni normali e non forza la scrittura di tutte le dirty pages.
    \item \textbf{Fasi di Recovery ARIES dopo un Crash:}
    \begin{enumerate}
        \item \textbf{Analysis Phase:} Scansiona log da ultimo checkpoint. Ricostruisce DPT e Transaction Table al momento del crash. Identifica transazioni "perdenti" (attive al crash).
        \item \textbf{Redo Phase:} Scansiona log in avanti. Riapplica modifiche (Redo) per le pagine sporche se necessario, per portare il DB allo stato esatto del crash.
        \item \textbf{Undo Phase:} Scansiona log all'indietro. Annulla operazioni delle transazioni "perdenti", scrivendo CLR per ogni operazione annullata.
    \end{enumerate}
\end{itemize}

\end{document}