\documentclass{article}

% --- Encoding e lingua ---
\usepackage[utf8]{inputenc}
\usepackage[italian]{babel}

% --- Margini e layout ---
\usepackage{geometry}
\geometry{a4paper, margin=1in}

% --- Font sans-serif ---
\usepackage[scaled]{helvet}
\renewcommand{\familydefault}{\sfdefault}
\usepackage[T1]{fontenc}

% --- Matematica ---
\usepackage{amsmath}
\usepackage{amssymb}

% --- Liste personalizzate ---
\usepackage{enumitem}

% --- Immagini ---
\usepackage{float}
\usepackage{tikz}
\usetikzlibrary{shapes.geometric, positioning, calc}

% --- Hyperlink ---
\usepackage{hyperref}
\hypersetup{
	colorlinks=true,
	linkcolor=white,     % Cambiato per visibilità su sfondo nero
	filecolor=magenta,
	urlcolor=blue,
	pdftitle={Appunti sulla Progettazione Logica dei Database},
	pdfpagemode=FullScreen,
}

% --- Colori e sfondo nero ---
\usepackage{xcolor}
\pagecolor{black}
\color{white}

% --- Evidenziazione del codice (richiede -shell-escape) ---
% Compilare con: pdflatex -shell-escape nomefile.tex
\usepackage{minted}
\setminted{
	frame=lines,     % Cornice attorno al codice
	framesep=2mm,
	fontsize=\small,
	breaklines=true, % A capo automatico per linee lunghe
	style=monokai    % Stile di highlighting
}


% --- Titolo ---
\title{Appunti sulla Progettazione Logica dei Database}
\author{Basato sulle slide del Prof. Danilo Montesi}
\date{\today}

\begin{document}
	
	\maketitle
	\tableofcontents
	\newpage
	
	\section{Introduzione alla Progettazione Logica dei Database}
	L'obiettivo principale della \textbf{progettazione logica} è "tradurre" lo schema concettuale (spesso un diagramma E-R) in uno schema logico (ad esempio, per un database relazionale come SQL Server, MySQL, PostgreSQL) che rappresenti gli stessi dati in modo \textbf{corretto ed efficiente}.
	
	\subsection{Input della Progettazione Logica}
	\begin{itemize}
		\item \textbf{Schema Concettuale:} Il diagramma E-R che descrive la realtà di interesse.
		\item \textbf{Informazioni sul Carico Applicativo (Workload):} Quali operazioni verranno eseguite più frequentemente? Quanti dati ci aspettiamo? (Fondamentale per l'efficienza).
		\item \textbf{Modello Logico Scelto:} Ad es. relazionale, a oggetti, a grafo. Ci concentreremo sul relazionale.
	\end{itemize}
	
	\subsection{Output della Progettazione Logica}
	\begin{itemize}
		\item \textbf{Schema Logico:} Ad esempio, un insieme di istruzioni \texttt{CREATE TABLE} per SQL.
		\item \textbf{Documentazione Associata:} Spiegazioni delle scelte fatte.
	\end{itemize}
	
	\subsection{Non è una semplice traduzione!}
	\begin{itemize}
		\item Alcuni aspetti dello schema concettuale potrebbero non essere rappresentabili direttamente nel modello logico scelto (es. generalizzazioni nel modello relazionale puro).
		\item Bisogna considerare le \textbf{prestazioni (efficienza)}.
	\end{itemize}
	
	\section{Fasi della Trasformazione Logica}
	\begin{enumerate}
		\item \textbf{Ristrutturazione dello Schema E-R (E-R Schema Restructuring):}
		\begin{itemize}
			\item Si modifica lo schema E-R iniziale tenendo conto del carico applicativo e del modello logico.
			\item \textbf{Perché?}
			\begin{itemize}
				\item Semplificare la successiva traduzione.
				\item Ottimizzare le prestazioni.
			\end{itemize}
			\item \textbf{Nota Bene:} Uno schema E-R ristrutturato \textit{non è più} uno schema concettuale puro, perché inizia a includere considerazioni di implementazione.
		\end{itemize}
		\item \textbf{Traduzione nel Modello Logico:}
		\begin{itemize}
			\item Si converte lo schema E-R ristrutturato nello schema del modello scelto (es. tabelle relazionali).
		\end{itemize}
	\end{enumerate}
	
	\section{Analisi delle Prestazioni (Approssimata)}
	A livello concettuale/logico iniziale, non possiamo valutare le prestazioni con precisione assoluta, ma usiamo degli "indicatori":
	\begin{itemize}
		\item \textbf{Spazio (Space):} Numero di istanze (righe nelle tabelle) attese.
		\begin{itemize}
			\item \textit{Esempio Pratico:} Se hai una tabella \texttt{Utenti} e prevedi 1 milione di utenti, questo è un indicatore di spazio.
		\end{itemize}
		\item \textbf{Tempo (Time):} Numero di istanze (entità e relazioni) "visitate" (lette/scritte) durante un'operazione.
		\begin{itemize}
			\item \textit{Esempio Pratico:} Per mostrare il profilo di un utente con i suoi ultimi 10 post e i commenti a quei post, quante righe da diverse tabelle (\texttt{Utenti}, \texttt{Post}, \texttt{Commenti}) devi leggere?
		\end{itemize}
	\end{itemize}
	La \textbf{Tabella delle Dimensioni (Size Table)} stima il numero di istanze per ogni entità (E) e relazione (R).
	La \textbf{Tabella degli Accessi (Access Table)}, per un'operazione specifica, elenca:
	\begin{itemize}
		\item Quali entità/relazioni vengono accedute.
		\item Quante volte (numero accessi).
		\item Tipo di accesso (Lettura/Scrittura).
		\item Ordine di accesso.
	\end{itemize}
	Questo aiuta a confrontare diverse opzioni di ristrutturazione.
	
	\section{Attività di Ristrutturazione dello Schema E-R}
	Sono 4 attività principali:
	
	\subsection{Analisi delle Ridondanze}
	\begin{itemize}
		\item Una \textbf{ridondanza} è un'informazione rilevante che può essere derivata da altre informazioni già presenti.
		\item Bisogna decidere se: mantenere, rimuovere o \textit{creare nuove} ridondanze.
		\item \textbf{Pro della Ridondanza:}
		\begin{itemize}
			\item Semplifica le query (meno join, calcoli al volo).
			\item \textit{Esempio Pratico:} In una tabella \texttt{Ordini}, potresti avere una colonna \texttt{TotaleOrdine}. Questo è ridondante se puoi calcolarlo sommando \texttt{Prezzo * Quantità} da una tabella \texttt{RigheOrdine} collegata. Averlo precalcolato velocizza la lettura del totale ordine.
		\end{itemize}
		\item \textbf{Contro della Ridondanza:}
		\begin{itemize}
			\item Gli aggiornamenti richiedono più tempo (devi aggiornare il dato in più posti e mantenerlo consistente).
			\item Aumenta lo spazio di archiviazione.
			\item \textit{Esempio Pratico (continuazione):} Se modifichi una riga in \texttt{RigheOrdine}, devi ricalcolare e aggiornare \texttt{TotaleOrdine} nella tabella \texttt{Ordini}. Se non lo fai, i dati diventano inconsistenti.
		\end{itemize}
		\item \textbf{Tipi di Ridondanze comuni in E-R:}
		\begin{itemize}
			\item \textbf{Attributi derivabili:}
			\begin{itemize}
				\item Da altri attributi nella stessa entità/relazione (es. \texttt{Fattura} con \texttt{ValoreNetto}, \texttt{IVA}, \texttt{ValoreLordo}. \texttt{ValoreLordo} è derivabile).
				\item Da attributi di altre entità/relazioni (es. \texttt{Acquisto} con attributo \texttt{Totale}, derivabile da $\sum(\texttt{Composizione.Quantita} \times \texttt{Prodotto.Prezzo})$).
			\end{itemize}
			\item \textbf{Relazioni derivabili:} Spesso cicli di relazioni (es. se hai \texttt{Studente - Frequenta - Lezione - TenutaDa - Docente}, una relazione diretta \texttt{Studente - InsegnatoDa - Docente} potrebbe essere ridondante).
			\item \textbf{Attributi derivabili da conteggio:} (es. \texttt{Citta} con attributo \texttt{NumeroAbitanti}, derivabile da \texttt{COUNT(*)} delle persone che risiedono in quella città).
		\end{itemize}
		\item \textbf{Decisione sulla Ridondanza:} Si basa sull'analisi costi/benefici, considerando la frequenza delle operazioni.
		\begin{itemize}
			\item Se un dato derivato è letto molto frequentemente e scritto/aggiornato raramente, mantenerlo ridondante può essere vantaggioso.
			\item Se è aggiornato spesso, la ridondanza può diventare problematica.
		\end{itemize}
	\end{itemize}
	
	\subsection{Eliminazione delle Generalizzazioni (Gerarchie)}
	\begin{itemize}
		\item Il modello relazionale puro non supporta direttamente le generalizzazioni (ereditarietà). Vanno trasformate.
		\item \textbf{Tre Soluzioni Possibili} (esempio: E0 genitore, E1 ed E2 figli):
		\begin{enumerate}
			\item \textbf{Embedding dei Figli nel Genitore (Roll-up / Accorpamento verso l'alto):}
			\begin{itemize}
				\item Si elimina E1 ed E2. L'entità E0 eredita tutti gli attributi di E1 ed E2.
				\item Si aggiunge un attributo "Tipo" (o "Kind") a E0 per distinguere se un'istanza era originariamente E1 o E2.
				\item Gli attributi specifici di E1 o E2 diventano \texttt{NULL}abili in E0 se un'istanza non è di quel tipo.
				\item Le relazioni dei figli vengono "spostate" sul genitore.
				\item \textit{Esempio Pratico:} \texttt{Veicolo} (genitore), \texttt{Auto} (figlio), \texttt{Moto} (figlio). Diventa un'unica tabella \texttt{VEICOLI(targa, marca, modello, tipoVeicolo, numPorte\_auto, cilindrata\_moto, ...)}. \texttt{numPorte\_auto} sarà \texttt{NULL} per le moto.
				\item \textbf{Quando usarla:} Se le operazioni accedono frequentemente al genitore e ai figli contemporaneamente.
			\end{itemize}
			\item \textbf{Embedding del Genitore nei Figli (Roll-down / Accorpamento verso il basso):}
			\begin{itemize}
				\item Si elimina E0. Le entità E1 ed E2 ereditano tutti gli attributi di E0.
				\item Le relazioni di E0 vengono replicate per E1 ed E2.
				\item \textit{Esempio Pratico:} Tabelle separate \texttt{AUTO(targa, marca, modello\_veicolo, numPorte, ...)} e \texttt{MOTO(targa, marca, modello\_veicolo, cilindrata, ...)}. \texttt{marca} e \texttt{modello\_veicolo} sono duplicati.
				\item \textbf{Quando usarla:} Se le operazioni accedono ai figli indipendentemente l'uno dall'altro.
			\end{itemize}
			\item \textbf{Sostituzione della Generalizzazione con Relazioni (Una tabella per classe):}
			\begin{itemize}
				\item Si mantengono E0, E1, E2 come entità separate.
				\item Si creano relazioni 1-a-1 tra E0 ed E1, e tra E0 ed E2. La chiave primaria di E1 (e E2) sarà anche chiave esterna verso E0.
				\item \textit{Esempio Pratico:} Tabella \texttt{VEICOLI(targa\_PK, marca, modello)}. Tabella \texttt{AUTO(targa\_FK\_PK, numPorte)}. Tabella \texttt{MOTO(targa\_FK\_PK, cilindrata)}. Per avere tutti i dati di un'auto, fai un \texttt{JOIN}.
				\item \textbf{Quando usarla:} Se i figli sono acceduti indipendentemente dal padre.
			\end{itemize}
		\end{enumerate}
		\item \textbf{Soluzioni Ibride:} Si possono combinare queste strategie, specialmente con gerarchie a più livelli.
	\end{itemize}
	
	\subsection{Partizionamento/Raggruppamento di Entità e Relazioni}
	L'obiettivo è ridurre il numero di accessi.
	\begin{itemize}
		\item \textbf{Partizionamento Verticale di Entità:}
		\begin{itemize}
			\item Se un'entità ha molti attributi e alcuni sono usati frequentemente insieme, mentre altri raramente, si può dividere l'entità in due (o più) entità collegate da una relazione 1-a-1.
			\item \textit{Esempio Pratico:} \texttt{Impiegato(Codice, Cognome, Indirizzo, DataNascita, Livello, Salario, Tasse)} può diventare \texttt{DatiPersonali(Codice\_PK, Cognome, Indirizzo, DataNascita)} e \texttt{DatiAziendali(Codice\_PK\_FK, Livello, Salario, Tasse)}.
		\end{itemize}
		\item \textbf{Ristrutturazione di Attributi Multi-Valore:}
		\begin{itemize}
			\item Un attributo multi-valore (es. \texttt{Ufficio} con \texttt{Telefono(1,N)}) viene trasformato in una nuova entità (\texttt{Telefono}) e una relazione 1-a-N (\texttt{Possiede}).
			\item \textit{Esempio Pratico:} Se un \texttt{Prodotto} può avere più \texttt{Tag}, crei una tabella \texttt{Prodotti}, una tabella \texttt{Tag} e una tabella di join \texttt{ProdottoTag}.
		\end{itemize}
		\item \textbf{Raggruppamento di Entità (Denormalizzazione):}
		\begin{itemize}
			\item Se un'entità A ha una relazione 1-a-1 (o N-a-1) con un'entità B, e sono \textit{sempre} accedute insieme, gli attributi di B possono essere "assorbiti" in A.
			\item \textit{Esempio Pratico:} \texttt{Persona(0,1) --- ViveIn --- (1,1)Appartamento}. Si possono spostare \texttt{NumAppartamento} e \texttt{IndirizzoAppartamento} nell'entità \texttt{Persona}, rendendoli \texttt{NULL}abili.
		\end{itemize}
		\item \textbf{Partizionamento Orizzontale di Relazioni:}
		\begin{itemize}
			\item Si dividono le istanze di una relazione in più relazioni distinte se hanno pattern di accesso diversi.
			\item \textit{Esempio Pratico:} Relazione \texttt{ParteDi} tra \texttt{Giocatore} e \texttt{Squadra} può essere divisa in \texttt{MilitaAttualmenteIn} e \texttt{HaMilitatoInPassato}.
		\end{itemize}
	\end{itemize}
	
	\subsection{Identificazione delle Chiavi Primarie}
	\begin{itemize}
		\item Operazione \textbf{obbligatoria} per la traduzione in un modello relazionale.
		\item \textbf{Criteri di Scelta:}
		\begin{enumerate}
			\item \textbf{Informazione Obbligatoria:} L'attributo deve essere \texttt{NOT NULL}.
			\item \textbf{Semplicità:} Preferire chiavi singole a chiavi composite.
			\item \textbf{Usata nelle Operazioni più Frequenti/Rilevanti.}
		\end{enumerate}
		\item \textbf{Nuovi Attributi (Surrogate Keys):} Se nessuna combinazione di attributi esistenti è una buona chiave primaria, si introducono nuovi attributi "artificiali".
		\begin{itemize}
			\item \textit{Esempio Pratico:} \texttt{id INT AUTO\_INCREMENT PRIMARY KEY} in SQL, \texttt{ObjectId()} in MongoDB.
		\end{itemize}
	\end{itemize}
	
	\section{Traduzione nel Modello Relazionale (Regole Generali)}
	\begin{itemize}
		\item \textbf{Entità:} Diventano tabelle. Gli attributi dell'entità diventano colonne. La chiave primaria identificata diventa la \texttt{PRIMARY KEY}.
		\item \textbf{Relazioni:}
		\begin{itemize}
			\item \textbf{Molti-a-Molti (M:N):}
			\begin{itemize}
				\item Diventano una \textbf{nuova tabella} (tabella di associazione).
				\item Colonne: chiavi primarie delle entità coinvolte (come \texttt{FOREIGN KEY}, formano la \texttt{PRIMARY KEY} della nuova tabella) + attributi propri della relazione.
				\item \textit{Esempio:} \texttt{IMPIEGATO(0,N) --- ISCRIZIONE(DataInizio) --- (1,N)PROGETTO}
				\begin{minted}{sql}
IMPIEGATO(Numero_PK, Cognome, Salario)
PROGETTO(Codice_PK, Nome, Budget)
ISCRIZIONE(NumeroImpiegato_FK_PK, CodiceProgetto_FK_PK, DataInizio)
				\end{minted}
				\item \textbf{Vincoli di Integrità Referenziale:} Vanno definiti.
				\item \textbf{Nomi Espressivi per FK:} Meglio \texttt{IDImpiegato}, \texttt{IDProgetto}.
				\item \textbf{Cardinalità Minima:} La traduzione base M:N *non* cattura la cardinalità minima. Richiede logica applicativa o \texttt{TRIGGER}/\texttt{CHECK} complessi.
			\end{itemize}
			
			\item \textbf{Relazioni Ricorsive (M:N sulla stessa entità):}
			\begin{itemize}
				\item Simile a M:N, si crea una nuova tabella con due chiavi esterne che puntano alla stessa tabella originale.
				\item \textit{Esempio:} \texttt{PRODOTTO --- CompostoDa(NumeroPezzi) --- PRODOTTO}
				\begin{minted}{sql}
PRODOTTO(Codice_PK, Nome, Costo)
COMPOSTODA(CodiceProdottoComposto_FK_PK, CodiceComponente_FK_PK, NumeroPezzi)
				\end{minted}
			\end{itemize}
			
			\item \textbf{Relazioni N-arie (coinvolgono 3 o più entità):}
			\begin{itemize}
				\item Diventano una nuova tabella con le chiavi primarie di tutte le entità coinvolte (come FK) + attributi propri.
				\item \textit{Esempio:} \texttt{FORNITORE --- FORNITURA(NumeroPezzi) --- PRODOTTO --- A REPARTO}
				\begin{minted}{sql}
FORNITURA(ID_Fornitore_FK_PK, ID_Prodotto_FK_PK, ID_Reparto_FK_PK, NumeroPezzi)
				\end{minted}
			\end{itemize}
			
			\item \textbf{Uno-a-Molti (1:N):}
			\begin{itemize}
				\item La chiave primaria dell'entità sul lato "1" viene \textbf{propagata} come \texttt{FOREIGN KEY} nell'entità sul lato "N".
				\item Gli attributi della relazione vengono aggiunti alla tabella sul lato "N".
				\item \textit{Esempio:} \texttt{GIOCATORE(1,1) --- CONTRATTO(DataIngaggio) --- (0,N)SQUADRA}
				\begin{minted}{sql}
SQUADRA(Nome_PK, Citta, ColoriSociali)
GIOCATORE(CF_PK, Cognome, DataNascita, Ruolo, NomeSquadra_FK, DataIngaggio)
				\end{minted}
				\item \textbf{Cardinalità Minima (0 sul lato N):} Se opzionale, la \texttt{FOREIGN KEY} permette \texttt{NULL}.
				\item \textbf{Cardinalità Minima (1 sul lato N):} Se obbligatoria, la \texttt{FOREIGN KEY} è \texttt{NOT NULL}.
			\end{itemize}
			
			\item \textbf{Entità con Identificatore Esterno (Entità Debole):}
			\begin{itemize}
				\item L'entità debole diventa una tabella la cui chiave primaria è composta dalla PK dell'entità forte + identificatore parziale.
				\item \textit{Esempio:} \texttt{STUDENTE(Matricola)} identificato da \texttt{UNIVERSITA(Nome)}.
				\begin{minted}{sql}
UNIVERSITA(Nome_PK, Citta, Indirizzo)
STUDENTE(NomeUniversita_FK_PKpart, Matricola_PKpart, Cognome, AnnoCorso)
				\end{minted}
				La PK di \texttt{STUDENTE} è \texttt{(NomeUniversita, Matricola)}.
			\end{itemize}
			
			\item \textbf{Uno-a-Uno (1:1):}
			\begin{itemize}
				\item Si sceglie una delle due tabelle e si aggiunge la PK dell'altra come \texttt{FOREIGN KEY} e \texttt{UNIQUE KEY}. Gli attributi della relazione vanno nella tabella scelta.
				\item \textit{Esempio:} \texttt{CAPO(1,1) --- SUPERVISIONE(DataInizio) --- (1,1)DIPARTIMENTO}
				\begin{minted}{sql}
-- Opzione 1:
DIPARTIMENTO(Nome_PK, Ufficio, Telefono)
CAPO(Codice_PK, Cognome, Salario, NomeDipartimento_FK_UNIQUE, DataInizioSupervisione)

-- Opzione 2:
CAPO(Codice_PK, Cognome, Salario)
DIPARTIMENTO(Nome_PK, Ufficio, Telefono, CodiceCapo_FK_UNIQUE, DataInizioSupervisione)
				\end{minted}
				\item \textbf{Cardinalità (0,1) - Opzionale:}
				\begin{itemize}
					\item Se una partecipazione è opzionale, la \texttt{FOREIGN KEY} è \texttt{NULL}abile (ma sempre \texttt{UNIQUE} se presente).
					\item Se entrambe sono \texttt{(0,1)}: la soluzione più pulita è una tabella separata per la relazione \texttt{SUPERVISIONE(CodiceCapo\_FK\_PK, NomeDipartimento\_FK\_PK, DataInizio)}.
				\end{itemize}
			\end{itemize}
		\end{itemize}
	\end{itemize}
	
	\section{Attenzione Finale}
	Piccole differenze nelle cardinalità e nelle scelte degli identificatori nello schema E-R possono portare a significati e schemi logici molto diversi. È fondamentale essere precisi.
	
	\section{Strumenti (Tools)}
	Esistono software CASE (Computer-Aided Software Engineering) che supportano la modellazione, come:
	\begin{itemize}
		\item ERwin/ERX
		\item IBM Rational Rose
	\end{itemize}
	Questi strumenti aiutano a disegnare diagrammi E-R e a generare lo schema DDL.
	
\end{document}
