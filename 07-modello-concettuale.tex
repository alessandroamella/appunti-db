%%%%%%%%%%%%%%%%%%%%%%%%%%%%%%%%%%%%%%%%%%%%%%%%%%%%%%%%%%%%%%%%%%%%%%%%%%%%%%%
% PREAMBOLO COMUNE PER APPUNTI (Stile Scuro)
%
% Questo file contiene tutte le impostazioni e i pacchetti comuni.
% NON contiene \begin{document} o \end{document}.
%
% Istruzioni per la compilazione del file principale:
% pdflatex -shell-escape nomefile_principale.tex
%%%%%%%%%%%%%%%%%%%%%%%%%%%%%%%%%%%%%%%%%%%%%%%%%%%%%%%%%%%%%%%%%%%%%%%%%%%%%%%

\documentclass{article}

% --- Encoding e lingua ---
\usepackage[utf8]{inputenc}
\usepackage[italian]{babel}

% --- Margini e layout ---
\usepackage{geometry}
\geometry{a4paper, margin=1in}

% --- Font sans-serif (come Helvetica) ---
\usepackage[scaled]{helvet}
\renewcommand{\familydefault}{\sfdefault}
\usepackage[T1]{fontenc}

% --- Matematica ---
\usepackage{amsmath}
\usepackage{amssymb}

% --- Liste personalizzate ---
\usepackage{enumitem}
% \setlist{nosep}

% --- Immagini e Grafica ---
\usepackage{float}
% \usepackage{graphicx}
\usepackage{tikz}
\usetikzlibrary{shapes.geometric, positioning, calc}

% --- Tabelle Avanzate ---
\usepackage{array}
\usepackage{booktabs}
\usepackage{longtable}

% --- Hyperlink e Metadati PDF ---
\usepackage{hyperref}
\hypersetup{
    colorlinks=true,
    linkcolor=white,
    filecolor=magenta,
    urlcolor=cyan,
    citecolor=green,
    % pdftitle, pdfauthor, ecc. verranno impostati nel file principale
    pdfpagemode=FullScreen,
    bookmarksopen=true,
    bookmarksnumbered=true
}

% --- Colori e Sfondo Nero ---
\usepackage{xcolor}
\pagecolor{black}
\color{white}

% --- Evidenziazione del Codice ---
\usepackage{minted}
\setminted{
    frame=lines,
    framesep=2mm,
    fontsize=\small,
    breaklines=true,
    style=monokai,
    bgcolor=black!80
}
\usemintedstyle{monokai}

% --- Comandi Personalizzati (Opzionali) ---
% \newcommand{\Rel}[1]{\textit{#1}}
% ... (altri comandi se necessario) ...

% NESSUN \title, \author, \date, \begin{document} o \end{document} QUI


% --- Hyperlink ---
\usepackage{hyperref}
\hypersetup{
	colorlinks=true,
	linkcolor=white,
	filecolor=magenta,
	urlcolor=cyan,
	pdftitle={Modellazione Concettuale dei Dati},
	pdfauthor={Alessandro Amella},
	pdfpagemode=FullScreen,
}


% --- Titolo ---
\title{Appunti sulla Progettazione Concettuale di Basi di Dati\\
  \large Appunti basati sulle dispense del Prof. Danilo Montesi}
\author{Alessandro Amella, Gemini e Claude}
\date{\today}

\begin{document}
	
	\maketitle
	\tableofcontents
	\newpage
	
	\section{Introduzione alla Progettazione Concettuale}
	L'obiettivo della progettazione concettuale è creare un modello dei dati che sia \textbf{indipendente} da qualsiasi specifico Database Management System (DBMS) o tecnologia. È la fase in cui tradiamo i requisiti del mondo reale in una struttura formale.
	
	\section{Il Processo di Progettazione di un Database}
	Il design di un database si articola in diverse fasi:
	\begin{enumerate}
		\item \textbf{Raccolta dei Requisiti del DB (\texttt{DB requirements})}: Cosa deve fare il database? Quali dati deve memorizzare?
		\item \textbf{Progettazione Concettuale (\texttt{Conceptual Design})}:
		\begin{itemize}
			\item È la fase di \textbf{ANALISI} ("\texttt{WHAT?}").
			\item Si traduce i requisiti in un modello concettuale (es. Diagramma Entità-Relazione o E-R).
			\item L'output è lo \textbf{Schema Concettuale (\texttt{Conceptual Schema})}. Questo schema è una descrizione astratta della struttura del database, focalizzata sulle entità, gli attributi e le relazioni tra di esse, senza preoccuparsi di come verranno implementate.
		\end{itemize}
		\item \textbf{Progettazione Logica (\texttt{Logical Design})}:
		\begin{itemize}
			\item È la fase di \textbf{PROGETTAZIONE} ("\texttt{HOW?}").
			\item Si traduce lo schema concettuale in un modello logico, specifico per un tipo di DBMS (es. relazionale, NoSQL).
			\item L'output è lo \textbf{Schema Logico (\texttt{Logical Schema})} (es. tabelle SQL con chiavi primarie/esterne, tipi di dato).
		\end{itemize}
		\item \textbf{Progettazione Fisica (\texttt{Physical Design})}:
		\begin{itemize}
			\item Si specificano i dettagli di implementazione fisica (es. indici, partizionamento).
			\item L'output è lo \textbf{Schema Fisico (\texttt{Physical Schema})}.
		\end{itemize}
	\end{enumerate}
	\textbf{Noi ci concentriamo sulla Progettazione Concettuale.}
	
	\section{Attività della Progettazione Concettuale e Modellazione dei Dati}
	Queste attività sono interconnesse e spesso iterative:
	\begin{itemize}
		\item \textbf{Elicitazione dei Requisiti (\texttt{Requirements' elicitation})}: Raccogliere le informazioni.
		\item \textbf{Analisi dei Requisiti (\texttt{Requirements' analysis})}: Capire e chiarire le informazioni.
		\item \textbf{Costruzione dello Schema Concettuale (\texttt{Building the conceptual schema})}: Disegnare il modello E-R.
		\item \textbf{Costruzione del Glossario (\texttt{Building the glossary})}: Definire i termini chiave.
	\end{itemize}
	
	\section{Raccolta dei Requisiti}
	\subsection{Fonti dei Requisiti}
	\begin{itemize}
		\item \textbf{Utenti e Clienti}:
		\begin{itemize}
			\item Interviste.
			\item Documentazione specifica (\textit{ad hoc}).
		\end{itemize}
		\item \textbf{Documentazione Esistente}:
		\begin{itemize}
			\item Leggi e regolamenti di settore.
			\item Regolamenti interni, processi aziendali.
			\item Soluzioni preesistenti.
		\end{itemize}
		\item \textbf{Moduli (\texttt{Forms})}: I moduli cartacei o digitali esistenti sono una miniera d'oro di informazioni sui dati.
	\end{itemize}
	
	\subsection{Acquisizione e Analisi}
	\begin{itemize}
		\item È un'attività \textbf{difficile e non standardizzata}.
		\item Spesso si parte da requisiti iniziali che necessitano di \textbf{raffinamento} attraverso ulteriori acquisizioni.
	\end{itemize}
	
	\subsection{Acquisizione tramite Interviste}
	\begin{itemize}
		\item \textbf{Diverse tipologie di utenti forniscono informazioni diverse}.
		\item I manager di alto livello hanno una visione più ampia ma meno dettagliata.
		\item Le interviste portano a raffinamenti successivi.
	\end{itemize}
	
	\subsection{Interagire con gli Utenti: Consigli}
	\begin{itemize}
		\item \textbf{Controlli di comprensione e coerenza frequenti}: "Quindi, se ho capito bene, ogni studente può iscriversi a più corsi, e ogni corso può avere più studenti?"
		\item \textbf{Esempi di casi d'uso (\texttt{Use cases})}: Molto utili, specialmente casi generici e casi limite. "Cosa succede se uno studente si iscrive e poi si ritira? E se un corso non ha iscritti?"
		\item \textbf{Chiedere definizioni e classificazioni}: "Cosa intende esattamente per 'studente attivo'?"
		\item \textbf{Chiedere di evidenziare aspetti essenziali vs. periferici}: "È fondamentale tracciare lo storico degli indirizzi dello studente, o basta l'indirizzo attuale?"
	\end{itemize}
	
	\section{Documentazione Descrittiva e Gestione dei Termini}
	\subsection{Regole per la Documentazione Descrittiva}
	\begin{itemize}
		\item \textbf{Scegliere il giusto livello di astrazione}.
		\item \textbf{Struttura delle frasi standard}: Semplifica l'analisi.
		\item \textbf{Dividere frasi troppo lunghe/complesse}.
		\item \textbf{Distinguere frasi sui "dati" da frasi sulle "funzioni"}:
		\begin{itemize}
			\item Dati: "Uno studente \textit{ha} un nome, un cognome e una matricola."
			\item Funzioni: "Il sistema \textit{deve permettere} di iscrivere uno studente a un corso."
		\end{itemize}
	\end{itemize}
	
	\subsection{Regole Generali per Termini e Concetti}
	\begin{itemize}
		\item \textbf{Costruire un glossario dei termini}: Cruciale per evitare ambiguità.
		\begin{itemize}
			\item \textit{Esempio Pratico}: Se nel tuo team Node.js uno chiama un campo \texttt{customerId} e un altro \texttt{client\_id}, il glossario chiarisce che \texttt{Customer} e \texttt{Client} sono sinonimi e si userà \texttt{customerId}.
		\end{itemize}
		\item \textbf{Omonimi e sinonimi devono essere unificati}: Un solo termine per un concetto.
		\item \textbf{Chiarire esplicitamente le relazioni tra i termini}.
		\item \textbf{Ordinare le frasi per concetti}: Raggruppare requisiti simili.
	\end{itemize}
	
	\section{Esempi di Requisiti}
	\subsection{Esempio Database Bibliografico}
	\begin{itemize}
		\item Automatizzare riferimenti bibliografici.
		\item ID di 7 caratteri (iniziali autori, anno, carattere disambiguazione).
		\item Riferimenti possono essere \textit{monografie} (editore, data, luogo) o \textit{articoli di rivista} (nome rivista, volume, numero, pagine, anno).
		\item Per entrambi: nomi degli autori.
	\end{itemize}
	
	\subsection{Esempio Azienda di Formazione}
	Questo esempio sarà usato più avanti per la progettazione.
	\begin{itemize}
		\item Gestire corsi, lezioni, insegnanti.
		\item \textbf{Studenti ($\sim$5000)}: ID, codice fiscale, cognome, età, sesso, luogo nascita, nome datori di lavoro (passati e presenti con date), indirizzo, telefono, corsi frequentati ($\sim$200) con voto finale.
		\item \textbf{Workshop/Corsi}: Tracciare workshop frequentati, dove e quando si tengono le lezioni. I corsi hanno codice, titolo, edizioni (con data inizio/fine, numero partecipanti).
		\item \textbf{Studenti Freelance}: Area di interesse, titolo onorifico.
		\item \textbf{Insegnanti ($\sim$300)}: Cognome, età, luogo nascita, nome corso insegnato, set corsi insegnati (passati/futuri), storico telefonate. Possono essere dipendenti interni o collaboratori esterni.
	\end{itemize}
	
	\section{Il Glossario}
	Il glossario è fondamentale. Ecco un esempio basato sull'azienda di formazione:
	
	\begin{longtable}{|p{0.2\textwidth}|p{0.4\textwidth}|p{0.15\textwidth}|p{0.15\textwidth}|}
		\hline
		\textbf{Termine} & \textbf{Descrizione} & \textbf{Sinonimo} & \textbf{Correlato a} \\
		\hline
		\endfirsthead
		\hline
		\textbf{Termine} & \textbf{Descrizione} & \textbf{Sinonimo} & \textbf{Correlato a} \\
		\hline
		\endhead
		\hline
		\endfoot
		\hline
		\endlastfoot
		Partecipante & Chi prende parte ai corsi & Studente & Corso, Società \\
		\hline
		Docente & L'insegnante dei corsi. Potrebbe essere un dipendente interno. & Insegnante & Corso \\
		\hline
		Corso & Corso interno. Può avere diverse edizioni. & Workshop & Docente \\
		\hline
		Azienda & Luogo di lavoro attuale (o passato) del partecipante. & Luogo & Partecipante \\
	\end{longtable}
	
	\textit{Esempio Pratico}: Nel tuo schema Prisma o MongoDB:
	\begin{itemize}
		\item \texttt{Participant} potrebbe diventare \texttt{model Student \{\}} o una collection \texttt{students}.
		\item Il glossario ti aiuta a decidere se \texttt{Lecturer} e \texttt{Teacher} sono la stessa cosa e come chiamare l'entità/collection (\texttt{model Teacher \{\}}).
		\item \texttt{Course} e \texttt{Workshop} sono sinonimi per la stessa entità/collection.
	\end{itemize}
	
	\section{Strutturare i Requisiti}
	Dopo la raccolta, i requisiti vanno organizzati in gruppi omogenei di frasi. L'esempio dell'azienda di formazione viene strutturato:
	\begin{itemize}
		\item \textbf{Frasi generali}: "L'azienda richiede un DB per corsi, lezioni, insegnanti."
		\item \textbf{Frasi sui partecipanti}: Dettagli sugli studenti (ID, CF, nome, età, datori lavoro, corsi frequentati, ecc.).
		\item \textbf{Frasi specifiche sui partecipanti}: Dettagli per freelance (area interesse) o dipendenti di organizzazioni (livello gerarchico).
		\item \textbf{Frasi sul datore di lavoro}: Dettagli sui datori di lavoro dei partecipanti (nome, indirizzo, telefono).
		\item \textbf{Frasi sui corsi}: Dettagli sui corsi (codice, titolo, edizioni, date, n° partecipanti, aule, orari).
		\item \textbf{Frasi sui docenti}: Dettagli sugli insegnanti (cognome, età, corsi insegnati, tipo contratto, ecc.).
	\end{itemize}
	Questa strutturazione aiuta a identificare le future entità e le loro proprietà.
	
	\section{Dai Requisiti agli Schemi Concettuali (E-R)}
	Come tradurre i termini identificati nei costrutti del modello Entità-Relazione (E-R)?
	\begin{itemize}
		\item \textbf{Entità (\texttt{Entity})}:
		\begin{itemize}
			\item Se il termine ha \textbf{proprietà rilevanti} e descrive \textbf{oggetti autonomi}.
			\item \textit{Esempio}: \texttt{Studente}, \texttt{Corso}, \texttt{Docente}.
			\item \textit{Prisma/SQL}: Diventeranno tabelle (\texttt{model Student \{\}}, \texttt{CREATE TABLE Student (...)}).
			\item \textit{MongoDB}: Diventeranno collections (\texttt{db.students}).
		\end{itemize}
		\item \textbf{Attributo (\texttt{Attribute})}:
		\begin{itemize}
			\item Se è un termine \textbf{semplice senza ulteriori specificazioni} (proprietà di un'entità).
			\item \textit{Esempio}: \texttt{Nome} dello Studente, \texttt{Titolo} del Corso.
			\item \textit{Prisma/SQL}: Diventeranno colonne nelle tabelle (\texttt{name: String}, \texttt{title: String}).
			\item \textit{MongoDB}: Diventeranno campi nei documenti (\texttt{\{ name: "Mario", title: "Database 101" \}}).
		\end{itemize}
		\item \textbf{Relazione (\texttt{Relationship})}:
		\begin{itemize}
			\item Quando un termine \textbf{collega altri termini} (entità).
			\item \textit{Esempio}: "Studente \textit{si iscrive a} Corso".
			\item \textit{Prisma/SQL}: Spesso implementate con chiavi esterne e tabelle di join.
			\begin{minted}{graphql}
model Student {
	id          Int      @id @default(autoincrement())
	// ... altri attributi
	enrollments Enrollment[]
}
model Course {
	id          Int      @id @default(autoincrement())
	// ... altri attributi
	enrollments Enrollment[]
}
model Enrollment { // Tabella di join
	studentId Int
	courseId  Int
	student   Student @relation(fields: [studentId], references: [id])
	course    Course  @relation(fields: [courseId], references: [id])
	enrollmentDate DateTime
	@@id([studentId, courseId])
}
			\end{minted}
			\item \textit{MongoDB}: Spesso implementate con DBRefs, array di ID, o embedding (se la relazione è 1-a-pochi e i dati sono strettamente legati).
		\end{itemize}
		\item \textbf{Generalizzazione (\texttt{Generalization / ISA Relationship})}:
		\begin{itemize}
			\item Quando un termine è un \textbf{caso più generale di un altro}.
			\item \textit{Esempio}: \texttt{Persona} è una generalizzazione di \texttt{Studente} e \texttt{Docente}. Sia studenti che docenti sono persone e condividono attributi comuni (nome, cognome, CF) ma hanno anche attributi specifici.
			\item \textit{Prisma/SQL}: Ci sono varie strategie:
			\begin{enumerate}
				\item Tabella unica con un campo "tipo" (es. \texttt{Person} con \texttt{personType: "Student" | "Teacher"}).
				\item Tabelle separate per le specializzazioni che referenziano una tabella base comune.
				\item Tabelle separate che duplicano gli attributi comuni (meno ideale per la consistenza).
			\end{enumerate}
			\item \textit{MongoDB}: Spesso si usa un campo \texttt{type} in una singola collection \texttt{people}, oppure collections separate se le differenze sono marcate.
		\end{itemize}
	\end{itemize}
	
	\section{Design Pattern E-R Comuni}
	Sono "best practices" per risolvere problemi comuni di modellazione.
	
	\subsection{Reificazione di Attributi in Entità}
	\begin{itemize}
		\item \textbf{Problema}: Un attributo di un'entità ha esso stesso delle proprietà o partecipa ad altre relazioni.
		\item \textbf{Esempio}: Inizialmente \texttt{Company} è un attributo (es. \texttt{companyName}) di \texttt{Employee}.
		\begin{itemize}
			\item Se \texttt{Company} deve avere un suo indirizzo, partita IVA, o essere collegata ad altri \texttt{Employee} o a \texttt{Projects}, allora \texttt{Company} va "reificata" (resa concreta) come un'entità separata.
			\item Si crea l'entità \texttt{Company} e una relazione \texttt{Job} (o \texttt{WorksFor}) tra \texttt{Employee} e \texttt{Company}.
		\end{itemize}
		\item \textbf{Cardinalità}: Un \texttt{Employee} (\texttt{1,1}) lavora per una \texttt{Company}. Una \texttt{Company} (\texttt{1,N}) può avere molti \texttt{Employee}.
		\begin{itemize}
			\item (\texttt{1,1}): Esattamente uno.
			\item (\texttt{1,N}): Da uno a molti.
			\item (\texttt{0,1}): Zero o uno.
			\item (\texttt{0,N}): Da zero a molti.
		\end{itemize}
		\item \textit{Esempio Pratico}:
		\begin{itemize}
			\item \textit{Prima}: \texttt{model Employee \{ id Int @id; name String; companyName String; \}}
			\item \textit{Dopo}:
			\begin{minted}{graphql}
model Employee {
	id        Int     @id
	name      String
	companyId Int
	company   Company @relation(fields: [companyId], references: [id])
}
model Company {
	id        Int     @id
	name      String
	address   String? // Company ha i suoi attributi
	employees Employee[]
}
			\end{minted}
		\end{itemize}
	\end{itemize}
	
	\subsection{Relazioni "Part-of" (Composizione e Aggregazione)}
	\begin{itemize}
		\item Relazioni (\texttt{1,N}) che rappresentano "parte di".
		\item \textbf{Composizione (\texttt{Composition})}: Forte dipendenza. La parte non può esistere senza il tutto.
		\begin{itemize}
			\item Esempio: \texttt{Cinema} (1) è composto da \texttt{Hall} (N). Ogni \texttt{Hall} (\texttt{1,1}) appartiene a un solo \texttt{Cinema}. Se il cinema viene distrutto, le sale non esistono più.
		\end{itemize}
		\item \textbf{Aggregazione (\texttt{Aggregation})}: Debole dipendenza. La parte può esistere indipendentemente dal tutto.
		\begin{itemize}
			\item Esempio: \texttt{Team} (1) è composto da \texttt{Expert} (N). Un \texttt{Expert} (\texttt{0,1}) può appartenere a un \texttt{Team} (o a nessuno). Se il team si scioglie, gli esperti esistono ancora.
		\end{itemize}
	\end{itemize}
	
	\subsection{"Instance-of"}
	\begin{itemize}
		\item \textbf{Problema}: Distinguere una rappresentazione astratta/modello da una sua istanza concreta.
		\item \textbf{Esempi}:
		\begin{itemize}
			\item \texttt{Flight} (astratto: rotta, orario generico) vs. \texttt{ScheduledFlight} (istanza: volo specifico di un giorno con data, aereo assegnato).
			\item \texttt{Tournament} (astratto: nome del torneo) vs. \texttt{Edition} (istanza: edizione 2024 del torneo, con date specifiche).
		\end{itemize}
		\item \textit{Esempio Pratico}:
		\begin{itemize}
			\item \texttt{ProductTemplate} (nome, descrizione generica) vs \texttt{ProductInstance} (SKU specifico, colore, taglia, data di produzione).
			\item \texttt{CourseDefinition} (codice, nome, crediti) vs \texttt{CourseOffering} (anno accademico, semestre, docente, aula).
		\end{itemize}
	\end{itemize}
	
	\subsection{Reificazione di Relazioni Binarie}
	\begin{itemize}
		\item \textbf{Problema}: Una relazione tra due entità ha essa stessa degli attributi o partecipa ad altre relazioni.
		\item \textbf{Esempio}: Studente - Esame - Lezione.
		\begin{itemize}
			\item Inizialmente, \texttt{Exam} potrebbe essere una relazione tra \texttt{Student} e \texttt{Lecture}.
			\item Se l'esame ha attributi come \texttt{Grade} (voto) e \texttt{Date}, allora \texttt{Exam} viene reificata come entità.
			\item Si creano due relazioni binarie: \texttt{Student-takes-Exam (S-E)} e \texttt{Exam-is\_for-Lecture (E-L)}.
		\end{itemize}
		\item \textit{Esempio Pratico (Prisma, vedi sopra per \texttt{Enrollment})}: Se la relazione "studente si iscrive a corso" ha una data di iscrizione, un voto, ecc., la tabella di join \texttt{Enrollment} diventa un'entità reificata.
	\end{itemize}
	
	\subsection{Reificazione di Relazioni Ricorsive}
	\begin{itemize}
		\item \textbf{Problema}: Una relazione tra istanze della stessa entità ha attributi.
		\item \textbf{Esempio}: \texttt{Team} gioca una \texttt{Match} contro un altro \texttt{Team}.
		\begin{itemize}
			\item Una relazione \texttt{PlaysAgainst} tra \texttt{Team} e \texttt{Team}.
			\item Se la partita (\texttt{Match}) ha attributi come \texttt{Date}, \texttt{Score}, allora \texttt{Match} viene reificata come entità.
			\item Si creano due relazioni: \texttt{Team\_Home-plays-Match} e \texttt{Team\_Visiting-plays-Match}.
		\end{itemize}
		\item \textit{Esempio Pratico}: Un \texttt{Employee} può essere \texttt{manager} di altri \texttt{Employee}. Se questa relazione di management ha una \texttt{startDate} o un \texttt{roleDescription}, si potrebbe reificare in un'entità \texttt{ManagementRelationship}.
	\end{itemize}
	
	\subsection{Reificazione di Attributi di Relazioni}
	\begin{itemize}
		\item \textbf{Problema}: Un attributo di una relazione molti-a-molti ha esso stesso delle proprietà.
		\item \textbf{Esempio}: \texttt{Player} (musicista) - \texttt{Plays} (suona) - \texttt{Orchestra}. La relazione \texttt{Plays} ha un attributo \texttt{Instrument}.
		\begin{itemize}
			\item Se \texttt{Instrument} (es. "Violino Stradivari Modello X") deve avere attributi propri (es. \texttt{Trademark}, \texttt{Type}, anno di fabbricazione) o essere suonato da più musicisti in diverse orchestre, allora \texttt{Instrument} va reificato.
			\item Si crea l'entità \texttt{Instrument} e la relazione \texttt{Plays} diventa ternaria (o si reifica \texttt{Plays} in un'entità che collega \texttt{Player}, \texttt{Orchestra}, \texttt{Instrument}).
		\end{itemize}
	\end{itemize}
	
	\subsection{Caso Specifico (Generalizzazione/ISA)}
	\begin{itemize}
		\item \textbf{Problema}: Una sottocategoria di un'entità ha caratteristiche o relazioni aggiuntive.
		\item \textbf{Esempio}: \texttt{Manager} è un caso specifico di \texttt{Employee}.
		\begin{itemize}
			\item Tutti i \texttt{Manager} sono \texttt{Employee}, ma solo i \texttt{Manager} gestiscono (\texttt{Manage}) dei \texttt{Project}.
			\item Non tutti gli \texttt{Employee} gestiscono progetti.
		\end{itemize}
		\item Si usa una freccia di generalizzazione (concettualmente) da \texttt{Manager} a \texttt{Employee}.
	\end{itemize}
	
	\subsection{Storicizzazione di un Concetto}
	\begin{itemize}
		\item \textbf{Problema}: Necessità di tracciare i cambiamenti di un concetto nel tempo.
		\item \textbf{Esempi}:
		\begin{itemize}
			\item \texttt{Anagraphic} (dati anagrafici) può avere una versione \texttt{Historic} e una \texttt{Current}. Si usano attributi come \texttt{StartDate}, \texttt{ExpiryDate}.
			\item \texttt{Software} può avere una versione \texttt{Legacy} e una \texttt{Updated}.
			\item \textbf{Impiego (\texttt{Employment})}: Si può modellare \texttt{CurrentEmployment} e \texttt{PastEmployment}.
			\begin{itemize}
				\item \textit{Opzione 1}: Due relazioni separate (\texttt{CurrentEmployment}, \texttt{PastEmployment}) tra \texttt{Employee} e \texttt{Company}.
				\item \textit{Opzione 2}: Reificare \texttt{Employment} come entità con \texttt{BeginDate}, \texttt{EndDate} e poi generalizzarla in \texttt{CurrentEmployment} e \texttt{PastEmployment} (o usare un attributo di stato).
			\end{itemize}
		\end{itemize}
		\item \textit{Esempio Pratico}: Per tracciare lo storico degli indirizzi di un cliente:
		\begin{minted}{graphql}
model Customer {
	id             Int       @id
	// ...
	addressHistory Address[]
}
model Address {
	id         Int       @id
	street     String
	city       String
	customerId Int
	customer   Customer  @relation(fields: [customerId], references: [id])
	startDate  DateTime
	endDate    DateTime? // Null se è l'indirizzo corrente
}
		\end{minted}
	\end{itemize}
	
	\subsection{Estensione di un Concetto (Generalizzazione/ISA)}
	\begin{itemize}
		\item \textbf{Problema}: Un concetto esistente viene esteso con nuove informazioni per casi specifici.
		\item \textbf{Esempio}: \texttt{Project} è un concetto generale. Un \texttt{AcceptedProject} (progetto accettato) richiede informazioni aggiuntive come \texttt{Founding} (finanziamento) e \texttt{StartDate}, che non sono necessarie per progetti in attesa o rifiutati.
		\item Si usa una generalizzazione (concettualmente) da \texttt{AcceptedProject} a \texttt{Project}.
	\end{itemize}
	
	\subsection{Relazioni Ternarie e Loro Reificazione}
	\begin{itemize}
		\item \textbf{Relazione Ternaria}: Coinvolge tre entità.
		\begin{itemize}
			\item Esempio: \texttt{Employee} lavora su un \texttt{Task} in un \texttt{Office}. La relazione \texttt{Work} collega queste tre.
			\item Le cardinalità indicano che un impiegato può lavorare su più task in più uffici, un ufficio può ospitare più impiegati su più task, e un task può essere svolto da più impiegati in più uffici.
		\end{itemize}
		\item \textbf{Reificazione di Relazione Ternaria (1)}:
		\begin{itemize}
			\item La relazione ternaria \texttt{Work} viene trasformata in un'entità \texttt{Work}.
			\item L'entità \texttt{Work} è collegata a \texttt{Employee}, \texttt{Office}, e \texttt{Task} tramite tre relazioni binarie (E-W, O-W, T-W).
			\item Questo è utile se l'evento "Work" ha attributi propri (es. \texttt{duration}, \texttt{status}).
		\end{itemize}
		\item \textbf{Reificazione di Relazione Ternaria (2) - Semplificata}:
		\begin{itemize}
			\item Se ci sono vincoli specifici (es. "un task può essere eseguito da un solo operatore e in un solo ufficio"), il modello può essere semplificato.
			\item Nell'esempio delle slide, \texttt{Task} diventa centrale, con una relazione (\texttt{1,1}) verso \texttt{Employee} (tramite O-S, probabilmente "Operator-for-Service") e (\texttt{1,1}) verso \texttt{Office} (tramite S-I, probabilmente "Service-at-Location").
			\item \textit{Nota}: Le etichette delle relazioni (O-S, S-I) sono un po' criptiche, ma il concetto è la semplificazione basata su vincoli.
		\end{itemize}
	\end{itemize}
	
	\section{Strategie di Progettazione dello Schema E-R}
	Come si affronta la creazione dello schema E-R?
	
	\subsection{Strategia Top-Down}
	\begin{enumerate}
		\item Si parte dai concetti più generali (entità principali).
		\item Si raffinano progressivamente aggiungendo dettagli:
		\begin{itemize}
			\item Identificare attributi.
			\item Identificare relazioni.
			\item Scomporre entità complesse.
			\item Introdurre generalizzazioni/specializzazioni.
		\end{itemize}
	\end{enumerate}
	\textit{Esempi}:
	\begin{itemize}
		\item \texttt{Exam} (iniziale) $\rightarrow$ \texttt{Student} - \texttt{Exam} (relazione) - \texttt{Lecture}.
		\item \texttt{Employee} (iniziale) $\rightarrow$ \texttt{Employee} con attributi \texttt{Surname}, \texttt{Age}, \texttt{Wage}.
		\item \texttt{People} (iniziale) $\rightarrow$ Generalizzazione in \texttt{Man} e \texttt{Woman}.
	\end{itemize}
	
	\subsection{Strategia Bottom-Up}
	\begin{enumerate}
		\item Si parte dai dettagli: attributi e concetti specifici.
		\item Si raggruppano per formare entità e relazioni.
		\item Si integrano i vari "pezzi" di schema per formare lo schema completo.
	\end{enumerate}
	\textit{Esempi}:
	\begin{itemize}
		\item Requisito su \texttt{Employee} $\rightarrow$ Entità \texttt{Employee}.
		\item Concetti \texttt{Student}, \texttt{Exam}, \texttt{Lecture} $\rightarrow$ Schema \texttt{Student-Exam-Lecture}.
		\item Entità \texttt{Man}, \texttt{Woman} $\rightarrow$ Generalizzazione \texttt{People}.
	\end{itemize}
	
	\subsection{Strategia Inside-Out}
	\begin{enumerate}
		\item Si identifica un concetto centrale e ben compreso.
		\item Si espande lo schema "verso l'esterno", aggiungendo concetti (entità, attributi, relazioni) direttamente collegati a quelli già identificati.
	\end{enumerate}
	\textit{Esempio (dalle slide)}:
	\begin{enumerate}
		\item Inizio: \texttt{Employee}.
		\item Aggiungo attributi a \texttt{Employee}: \texttt{Surname}, \texttt{Code}.
		\item Aggiungo \texttt{Dept} (Dipartimento) e le relazioni \texttt{Supervision} (Employee supervisiona Dept) e \texttt{Belonging} (Employee appartiene a Dept, con attributo \texttt{Date}).
		\item Aggiungo \texttt{Project} e la relazione \texttt{Enrollment} (Employee partecipa a Project).
		\item Aggiungo \texttt{Office} e la relazione \texttt{Composition} (Dept è composto da Office, con attributo \texttt{Addr} complesso).
	\end{enumerate}
	
	\section{Regola Pratica e Metodologia}
	\subsection{Regola Pratica: Usare uno Stile Misto!}
	\begin{enumerate}
		\item \textbf{Crea uno "schizzo" (\texttt{sketch})}: Identifica le entità più rilevanti.
		\item \textbf{Decomponi lo schema}: Dividi il problema se complesso.
		\item \textbf{Raffina (top-down), integra (bottom-up), espandi (inside-out)}.
	\end{enumerate}
	
	\subsection{Sketching dello Schema E-R}
	\begin{itemize}
		\item Parti dalle entità più rilevanti (più citate o esplicitamente indicate come tali).
		\item Crea un primo schema E-R di base.
	\end{itemize}
	
	\subsection{Metodologia "Best Practice"}
	\begin{enumerate}
		\item \textbf{Analisi dei Requisiti}:
		\begin{itemize}
			\item Analizza, risolvi ambiguità.
			\item Crea un glossario.
			\item Raggruppa requisiti simili.
		\end{itemize}
		\item \textbf{Caso Base (\texttt{Base case})}:
		\begin{itemize}
			\item Definisci uno schema "abbozzato" con i concetti più rilevanti.
		\end{itemize}
		\item \textbf{Caso Iterativo (\texttt{Iterative case})} (ripeti finché non va bene):
		\begin{itemize}
			\item Raffina i concetti base usando i requisiti.
			\item Aggiungi concetti per descrivere requisiti non ancora coperti.
		\end{itemize}
		\item \textbf{Analisi di Qualità (\texttt{Quality analysis})} (ripeti durante tutto il processo):
		\begin{itemize}
			\item Controlla la qualità dello schema e modificalo.
		\end{itemize}
	\end{enumerate}
	
	\section{Qualità dello Schema E-R}
	Misure di qualità per uno schema E-R:
	\begin{itemize}
		\item \textbf{Correttezza (\texttt{Correctness})}: Lo schema rappresenta accuratamente i requisiti? Usa i costrutti E-R in modo appropriato?
		\item \textbf{Completezza (\texttt{Completeness})}: Tutti i requisiti sono stati rappresentati nello schema? Tutti i dati necessari sono modellati?
		\item \textbf{Chiarezza (\texttt{Clarity})}: Lo schema è facile da capire? È ambiguo?
		\item \textbf{Minimalità (\texttt{Minimality})}: Ci sono elementi ridondanti (entità, attributi, relazioni non necessarie)? Si potrebbe rappresentare la stessa informazione in modo più semplice?
	\end{itemize}
	
	\section{Best Practice e Integrazione di Schemi}
	Per sistemi complessi, si può decomporre il problema:
	
	\subsection{Approccio 1}
	\begin{enumerate}
		\item Analisi dei Requisiti.
		\item Caso Base (schema "scheletro" generale).
		\item \textbf{Decomposizione}: Suddividi i requisiti complessi secondo lo schema scheletro.
		\item Caso Iterativo per ogni \textbf{sotto-schema}.
		\item \textbf{Integrazione}: Unisci i sotto-schemi in uno schema totale, usando lo schema scheletro come riferimento.
		\item Analisi di Qualità.
	\end{enumerate}
	
	\subsection{Approccio 2}
	\begin{enumerate}
		\item Analisi dei Requisiti.
		\item \textbf{Decomposizione}: Identifica aree di interesse e partiziona i requisiti (o acquisiscili separatamente per area).
		\item \textbf{Per ogni area}:
		\begin{itemize}
			\item Caso Base.
			\item Caso Iterativo.
		\end{itemize}
		\item \textbf{Integrazione}: Unisci gli schemi delle varie aree.
		\item Analisi di Qualità.
	\end{enumerate}
	
	\section{Esempio Finale: Azienda di Formazione}
	Questo è un'applicazione pratica della metodologia all'esempio dell'azienda di formazione.
	
	\subsection{Affermazione Generale}
	"Azienda di formazione richiede DB per corsi, lezioni, insegnanti."
	
	\subsection{Schema Abbozzato (\texttt{Sketched Schema})}
	\begin{itemize}
		\item Entità: \texttt{Participant}, \texttt{Lecture} (Lezione/Corso), \texttt{Lecturer} (Docente).
		\item Relazioni: \texttt{Presence} (Participant - Lecture), \texttt{Teaching} (Lecturer - Lecture).
	\end{itemize}
	
	\subsection{Raffinamento: Partecipanti e Datori di Lavoro}
	\begin{itemize}
		\item Requisiti: ID, CF, dati anagrafici, datori di lavoro (passati/presenti), corsi frequentati. Freelance vs. Dipendenti.
		\item \textbf{Schema Parziale (1)} (basato sulla slide 70):
		\begin{itemize}
			\item Entità \texttt{Participant} con attributi (\texttt{Tax}, \texttt{Code}, ...).
			\item Generalizzazione: \texttt{Participant} è generalizzazione di \texttt{Employee} (con attributi \texttt{Level}, \texttt{Position}) e \texttt{Freelance} (con \texttt{Title}, \texttt{Area}).
			\item Entità \texttt{Employer} (Datore di lavoro) con attributi (\texttt{Name}, ...).
			\item Relazioni: \texttt{CurrEmpl} (tra \texttt{Participant} e \texttt{Employer} per impiego attuale, \texttt{1-a-N}), \texttt{PastEmpl} (tra \texttt{Participant} e \texttt{Employer} per impieghi passati, \texttt{N-a-N}, reificata).
		\end{itemize}
	\end{itemize}
	
	\subsection{Raffinamento: Corsi}
	\begin{itemize}
		\item Requisiti: Corsi ($\sim$200) con codice, titolo, edizioni (con data inizio/fine, n° partecipanti), lezioni (giorno, aula, orario).
		\item \textbf{Schema Parziale (2)} (basato sulla slide 72):
		\begin{itemize}
			\item Pattern "Instance-of": \texttt{Lecture} (Corso generico) e \texttt{Edition} (Edizione specifica del corso).
			\begin{itemize}
				\item \texttt{Lecture} (Attributi: \texttt{Title}, \texttt{Code}).
				\item \texttt{Edition} (Attributi: \texttt{Start}, \texttt{End}, \texttt{\#Part.} - numero partecipanti).
				\item Relazione \texttt{KindOf} (\texttt{1,1}) tra \texttt{Edition} e \texttt{Lecture} (un'edizione è di un solo tipo di corso).
			\end{itemize}
			\item Pattern "Part-of": \texttt{Edition} è composta da \texttt{Lesson} (singola lezione).
			\begin{itemize}
				\item \texttt{Lesson} (Attributi: \texttt{Time}, \texttt{Room}, \texttt{Day}).
				\item Relazione \texttt{MadeOf} (\texttt{1,N}) tra \texttt{Edition} e \texttt{Lesson}.
			\end{itemize}
		\end{itemize}
	\end{itemize}
	
	\subsection{Raffinamento: Docenti}
	\begin{itemize}
		\item Requisiti: Docenti ($\sim$300) con dati anagrafici, corsi insegnati (passati/futuri), storico telefonate. Dipendenti interni vs. Esterni.
		\item \textbf{Schema Parziale (3)} (basato sulla slide 74):
		\begin{itemize}
			\item Entità \texttt{Lecturer} con attributi (\texttt{Tax}, \texttt{Surname}, \texttt{Age}, \texttt{Place of Birth}, \texttt{Phone} (multi-valore, \texttt{1,N})).
			\item Generalizzazione: \texttt{Lecturer} è generalizzazione di \texttt{Independent} (esterno) e \texttt{Home} (interno).
		\end{itemize}
	\end{itemize}
	
	\subsection{Integrazione dello Schema}
	Si parte dallo schema abbozzato e si integrano i raffinamenti.
	\begin{itemize}
		\item \textbf{Schema Intermedio (1)} (basato sulla slide 76): Integrazione di \texttt{Participant} e \texttt{Lecture}.
		\begin{itemize}
			\item Relazioni \texttt{PastPresence} (\texttt{0,N})-(\texttt{0,N}) e \texttt{CurrentPresence} (\texttt{0,1})-(\texttt{0,N}) tra \texttt{Participant} e \texttt{Lecture}. (Questo modella la frequenza ai corsi/lezioni).
		\end{itemize}
		\item \textbf{Schema Intermedio (2)} (basato sulla slide 77): Integrazione di \texttt{Lecturer}, \texttt{Lecture}, \texttt{Edition}.
		\begin{itemize}
			\item Relazioni \texttt{Past} (\texttt{0,1})-(\texttt{0,N}) e \texttt{Current} (\texttt{0,1})-(\texttt{0,N}) tra \texttt{Edition} e \texttt{Lecturer} (per insegnamento).
			\item Relazione \texttt{Duty} (\texttt{1,N})-(\texttt{0,N}) tra \texttt{Lecturer} e \texttt{Lecture} (per indicare i corsi che un docente \textit{può} insegnare o \textit{ha insegnato} in generale, separato dalle specifiche edizioni).
		\end{itemize}
	\end{itemize}
	
	\subsection{Schema Finale (Solo Entità e Relazioni)}
	La slide 78 mostra la struttura complessiva integrando tutti i pezzi, omettendo gli attributi per chiarezza.
	Si vedono chiaramente:
	\begin{itemize}
		\item \texttt{Participant} generalizzato in \texttt{Employee} e \texttt{Freelance}.
		\item \texttt{Employee} collegato a \texttt{Employer}.
		\item \texttt{Lecturer} generalizzato in \texttt{Independ.} e \texttt{Home}.
		\item Il nucleo \texttt{Lecture} $\rightarrow$ \texttt{Edition} $\rightarrow$ \texttt{Lesson}.
		\item Le relazioni che collegano \texttt{Participant} a \texttt{Lecture}/\texttt{Edition} (frequenza).
		\item Le relazioni che collegano \texttt{Lecturer} a \texttt{Lecture}/\texttt{Edition} (insegnamento).
	\end{itemize}
	Questo processo iterativo di sketching, raffinamento e integrazione, guidato dai requisiti e supportato da un glossario e da pattern di progettazione, porta a uno schema concettuale robusto.
	
\end{document}
