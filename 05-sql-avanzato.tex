\documentclass{article}

% --- Encoding e lingua ---
\usepackage[utf8]{inputenc}
\usepackage[italian]{babel}

% --- Margini e layout ---
\usepackage{geometry}
\geometry{a4paper, margin=1in}

% --- Font sans-serif ---
\usepackage[scaled]{helvet}
\renewcommand{\familydefault}{\sfdefault}

% --- Matematica ---
\usepackage{amsmath}
\usepackage{amssymb}

% --- Liste personalizzate ---
\usepackage{enumitem}

% --- Hyperlink ---
\usepackage{hyperref}
\hypersetup{
	colorlinks=true,
	linkcolor=white, % Cambiato per visibilità su sfondo nero
	filecolor=magenta,
	urlcolor=white,  % Cambiato per visibilità su sfondo nero
	citecolor=green, % Aggiunto per completezza
	pdftitle={Appunti su SQL Avanzato},
	pdfpagemode=FullScreen,
}

% --- Colori e sfondo nero ---
\usepackage{xcolor}
\pagecolor{black}
\color{white}

% --- Evidenziazione del codice (richiede -shell-escape) ---
% Compilare con: pdflatex -shell-escape nomefile.tex
\usepackage{minted}
\setminted{
	frame=lines,     % Cornice attorno al codice
	framesep=2mm,
	fontsize=\small,
	breaklines=true, % A capo automatico per linee lunghe
	style=monokai   % Stile di highlighting (assicurati che pygments lo supporti)
}
\usemintedstyle{monokai} % Applica lo stile globalmente per minted


% --- Titolo ---
\title{Appunti su SQL Avanzato}
\author{Basato sulle slide del Prof. Danilo Montesi}
\date{\today}

\begin{document}
	
	\maketitle
	\tableofcontents
	\newpage
	
	\section{Vincoli (Constraints)}
	
	\subsection{\texttt{CHECK}}
	\begin{itemize}
		\item \textbf{Concetto}: Specifica vincoli sui valori che una tupla (riga) può assumere. È una forma di validazione dei dati a livello di database.
		\item \textbf{Sintassi}: \texttt{CHECK ( Predicate )}
		\item \textbf{Esempi}:
		\begin{itemize}
			\item Semplice:
			\begin{minted}{sql}
Gender CHARACTER NOT NULL CHECK (Gender IN ('M', 'F'))
			\end{minted}
			\item Semplice:
			\begin{minted}{sql}
Salary INTEGER CHECK (Salary >= 0)
			\end{minted}
			\item Complesso (con subquery):
			\begin{minted}{sql}
-- Assicura che lo stipendio di un impiegato non superi
-- quello del suo supervisore.
-- Nota: le subquery nei CHECK non sono supportate da tutti i DBMS.
CHECK (Salary <= (SELECT Salary
FROM EMPLOYEE J
WHERE Supervisor = J.Number))
			\end{minted}
			\item Derivato:
			\begin{minted}{sql}
-- Assicura la coerenza per campi calcolati.
CHECK (Net = Salary - Withholding)
			\end{minted}
		\end{itemize}
		\item \textbf{Importanza}: Se un \texttt{INSERT} o \texttt{UPDATE} viola un vincolo \texttt{CHECK}, l'operazione fallisce, mantenendo l'integrità dei dati.
	\end{itemize}
	
	\subsection{\texttt{ASSERTION}}
	\begin{itemize}
		\item \textbf{Concetto}: Definisce vincoli a livello di schema, cioè che coinvolgono potenzialmente più tabelle o l'intero database, non solo una singola tupla.
		\item \textbf{Sintassi}: \texttt{CREATE ASSERTION NomeAsserzione CHECK ( Predicate )}
		\item \textbf{Esempio}:
		\begin{minted}{sql}
-- Questa asserzione garantisce che la tabella EMPLOYEE
-- non sia mai completamente vuota.
CREATE ASSERTION AtLeastOneEmployee
CHECK (1 <= (SELECT COUNT(*) FROM EMPLOYEE));
		\end{minted}
		\item \textit{Nota Pratica}: Anche qui, il supporto completo (specialmente con subquery complesse) varia tra i DBMS.
	\end{itemize}
	
	\section{Viste (Views)}
	\begin{itemize}
		\item \textbf{Concetto}: Una vista è una tabella virtuale il cui contenuto è definito da una query. Non memorizza dati fisicamente (generalmente), ma esegue la sua query sottostante ogni volta che viene interrogata.
		\item \textbf{Sintassi}: \texttt{CREATE VIEW NomeVista [(ListaAttributi)] AS SelectStatement}
		\item \textbf{Esempio}:
		\begin{minted}{sql}
CREATE VIEW ADMINEMPLOYEES (Name, Surname, Salary) AS
SELECT Name, Surname, Salary
FROM EMPLOYEE
WHERE Dept = 'Administration' AND Salary > 10;
		\end{minted}
		\item \textbf{Utilizzi}:
		\begin{itemize}
			\item \textbf{Semplificazione}: Nascondere la complessità di query complesse.
			\item \textbf{Sicurezza}: Limitare l'accesso a determinate colonne o righe di una tabella.
			\item \textbf{Indipendenza logica dei dati}: Se la struttura delle tabelle sottostanti cambia, la vista può essere modificata per mantenere la stessa interfaccia per gli utenti/applicazioni.
		\end{itemize}
	\end{itemize}
	
	\subsection{Aggiornamento delle Viste e \texttt{WITH CHECK OPTION}}
	\begin{itemize}
		\item Le viste possono essere aggiornabili (tramite \texttt{INSERT}, \texttt{UPDATE}, \texttt{DELETE}) se definite su una singola tabella e soddisfano certe condizioni.
		\item \texttt{WITH CHECK OPTION}: Se specificato, qualsiasi \texttt{INSERT} o \texttt{UPDATE} eseguito tramite la vista deve soddisfare la clausola \texttt{WHERE} della vista stessa.
		\begin{itemize}
			\item \textbf{Esempio}:
			\begin{minted}{sql}
CREATE VIEW POORADMINEMPLOYEES AS
SELECT *
FROM ADMINEMPLOYEES -- Supponiamo sia una vista o tabella
HERE Salary < 50
WITH CHECK OPTION;
			\end{minted}
			Se si tenta di fare \texttt{UPDATE POORADMINEMPLOYEES SET Salary = 60 WHERE Name = 'Ann'}, l'operazione fallirà.
			\item \texttt{LOCAL} vs \texttt{CASCADED} (per viste su viste):
			\begin{itemize}
				\item \texttt{LOCAL}: Il \texttt{CHECK OPTION} si applica solo alla definizione della vista corrente.
				\item \texttt{CASCADED}: Il \texttt{CHECK OPTION} si applica alla vista corrente E a tutte le viste sottostanti.
			\end{itemize}
		\end{itemize}
	\end{itemize}
	
	\subsection{Interrogare le Viste}
	\begin{itemize}
		\item Si interrogano come normali tabelle. Il DBMS sostituisce la vista con la sua definizione.
		\item \textbf{Utilità per query complesse}:
		\begin{itemize}
			\item \textbf{Problema}: "Calcolare la media del numero di uffici distinti per dipartimento". Una query come \texttt{SELECT AVG(COUNT(DISTINCT Office)) FROM EMPLOYEE GROUP BY Dept} è errata perché non si possono annidare funzioni aggregate direttamente.
			\item \textbf{Soluzione con Vista}:
			\begin{minted}{sql}
CREATE VIEW DEPTOFFICES (NameDept, OffNum) AS
SELECT Dept, COUNT(DISTINCT Office)
FROM EMPLOYEE
GROUP BY Dept;

SELECT AVG(OffNum) FROM DEPTOFFICES;
			\end{minted}
		\end{itemize}
	\end{itemize}
	
	\section{Query Ricorsive (\texttt{WITH RECURSIVE})}
	\begin{itemize}
		\item \textbf{Concetto}: Permettono di interrogare dati gerarchici o grafi. SQL:1999 ha introdotto le Common Table Expressions (CTE) ricorsive.
		\item \textbf{Sintassi Base}:
		\begin{minted}{sql}
WITH RECURSIVE NomeCTE (colonne) AS (
	-- Membro Ancora (non ricorsivo, caso base)
	SELECT ...
	UNION ALL
	-- Membro Ricorsivo (richiama NomeCTE)
	SELECT ... FROM NomeCTE JOIN ...
	)
SELECT * FROM NomeCTE;
		\end{minted}
		\item \textbf{Esempio (Trovare tutti gli antenati)}: Data una tabella \texttt{FATHERHOOD(Father, Child)}
		\begin{minted}{sql}
WITH RECURSIVE ANCESTORS (Ancestor, Descendant) AS (
	-- Caso base: padri diretti
	SELECT Father, Child FROM FATHERHOOD
	UNION ALL
	-- Passo ricorsivo: il padre di un antenato
	-- è anche un antenato
	SELECT FH.Father, A.Descendant
	FROM FATHERHOOD FH, ANCESTORS A
	WHERE FH.Child = A.Ancestor
)
SELECT * FROM ANCESTORS;
		\end{minted}
	\end{itemize}
	
	\section{Funzioni Scalari}
	Funzioni che operano su valori singoli e restituiscono un singolo valore per tupla.
	
	\subsection{Temporali}
	\begin{itemize}
		\item \texttt{CURRENT\_DATE()}: Data corrente.
		\item \texttt{EXTRACT(parte FROM espressione\_data)}: Estrae una parte da una data (es. \texttt{EXTRACT(YEAR FROM OrderDate)}).
		\item Esempio:
		\begin{minted}{sql}
SELECT EXTRACT(YEAR FROM OrderDate) AS OrderYear
FROM ORDERS
WHERE DATE(OrderDate) = CURRENT_DATE();
		\end{minted}
	\end{itemize}
	
	\subsection{Stringhe}
	\begin{itemize}
		\item \texttt{CHAR\_LENGTH(stringa)}: Lunghezza della stringa.
		\item \texttt{LOWER(stringa)}: Stringa in minuscolo.
	\end{itemize}
	
	\subsection{Casting}
	\begin{itemize}
		\item \texttt{CAST(espressione AS NuovoTipo)}: Converte un valore in un altro tipo di dato.
	\end{itemize}
	
	\subsection{Condizionali}
	\begin{itemize}
		\item \textbf{\texttt{COALESCE(expr1, expr2, ..., default)}}: Restituisce la prima espressione non-NULL nella lista.
		\begin{itemize}
			\item Esempio: \texttt{SELECT COALESCE(Mobile, PhoneHome, 'N/A') FROM EMPLOYEE;}
		\end{itemize}
		\item \textbf{\texttt{NULLIF(expr1, expr2)}}: Restituisce \texttt{NULL} se \texttt{expr1 = expr2}, altrimenti restituisce \texttt{expr1}.
		\begin{itemize}
			\item Esempio: \texttt{SELECT NULLIF(Dept, 'Unknown') FROM EMPLOYEE;}
		\end{itemize}
		\item \textbf{\texttt{CASE}}: Struttura if-then-else in SQL.
		\begin{itemize}
			\item \textbf{Sintassi "Searched"}:
			\begin{minted}{sql}
CASE
WHEN condizione1 THEN risultato1
WHEN condizione2 THEN risultato2
...
ELSE risultato_default
END
			\end{minted}
			\item \textbf{Esempio}: Calcolo tasse veicoli
			\begin{minted}{sql}
SELECT PlateNum,
(CASE Type
WHEN 'Car' THEN 2.58 * KWatt
WHEN 'Moto' THEN (22.00 + 1.00 * KWatt)
ELSE NULL
END) AS Tax
FROM VEHICLE
WHERE Year > 1975;
			\end{minted}
		\end{itemize}
	\end{itemize}
	
	\section{Sicurezza del Database}
	
	\subsection{Privilegi}
	\begin{itemize}
		\item SQL permette di concedere privilegi specifici (es. \texttt{SELECT}, \texttt{INSERT}, \texttt{UPDATE}, \texttt{DELETE}, \texttt{REFERENCES}, \texttt{USAGE}) agli utenti.
		\item I privilegi possono essere su: intero DB, tabelle, viste, colonne, domini.
	\end{itemize}
	
	\subsection{\texttt{GRANT} e \texttt{REVOKE}}
	\begin{itemize}
		\item \textbf{\texttt{GRANT}}: Concede privilegi.
		\begin{itemize}
			\item Sintassi: \texttt{GRANT <Privilegi | ALL PRIVILEGES> ON Risorsa TO Utenti [WITH GRANT OPTION];}
			\item \texttt{WITH GRANT OPTION}: Permette all'utente ricevente di propagare quel privilegio ad altri.
			\item Esempio: \texttt{GRANT SELECT ON DEPARTMENT TO Jack;}
		\end{itemize}
		\item \textbf{\texttt{REVOKE}}: Rimuove privilegi.
		\begin{itemize}
			\item Sintassi: \texttt{REVOKE Privilegi ON Risorsa FROM Utenti [RESTRICT | CASCADE];}
			\item \texttt{RESTRICT} (default): La revoca fallisce se altri utenti dipendono da quel grant.
			\item \texttt{CASCADE}: La revoca si estende a tutti gli utenti a cui il privilegio è stato propagato.
		\end{itemize}
	\end{itemize}
	
	\subsection{Discussione sui Privilegi}
	\begin{itemize}
		\item Il sistema dovrebbe nascondere le parti del DB non accessibili senza dare indizi sulla loro esistenza.
		\item Le \textbf{viste} sono uno strumento chiave per la sicurezza: si possono concedere privilegi su una vista che mostra solo certe righe/colonne.
	\end{itemize}
	
	\section{Autorizzazioni: RBAC (Role-Based Access Control)}
	\begin{itemize}
		\item \textbf{Concetto}: SQL-3 introduce RBAC. Un \texttt{ROLE} (ruolo) è un contenitore di privilegi.
		\begin{enumerate}
			\item Si creano ruoli.
			\item Si concedono privilegi AI RUOLI.
			\item Si concedono I RUOLI AGLI UTENTI.
		\end{enumerate}
		\item \textbf{Comandi RBAC}:
		\begin{itemize}
			\item \texttt{CREATE ROLE NomeRuolo;}
			\item \texttt{GRANT Privilegio ON Risorsa TO NomeRuolo;}
			\item \texttt{GRANT NomeRuolo TO NomeUtente;}
			\item \texttt{SET ROLE NomeRuolo;}
		\end{itemize}
		\item \textbf{Esempio RBAC}:
		\begin{minted}{sql}
-- 1. Crea il ruolo
CREATE ROLE Employee;
-- 2. Concedi un privilegio al ruolo
GRANT CREATE TABLE TO Employee;
-- 3. Assegna il ruolo a un utente
GRANT Employee TO 'specific_user';
		\end{minted}
	\end{itemize}
	
	\section{Transazioni}
	\begin{itemize}
		\item \textbf{Concetto}: Una transazione è un'unità logica di elaborazione del database, trattata come un'operazione atomica.
		\item \textbf{Proprietà ACID}:
		\begin{itemize}
			\item \textbf{Atomicity (Atomicità)}: O tutto o niente.
			\item \textbf{Consistency (Consistenza)}: Porta il DB da uno stato valido a un altro.
			\item \textbf{Isolation (Isolamento)}: Le transazioni concorrenti non interferiscono.
			\item \textbf{Durability (Durabilità)}: Le modifiche confermate (\texttt{COMMIT}) sono permanenti.
		\end{itemize}
		\item \textbf{Supporto SQL per Transazioni}:
		\begin{itemize}
			\item \texttt{START TRANSACTION;} (o \texttt{BEGIN TRANSACTION;})
			\item \texttt{COMMIT [WORK];}: Salva permanentemente le modifiche.
			\item \texttt{ROLLBACK [WORK];}: Annulla tutte le modifiche.
			\item \texttt{AUTOCOMMIT}: Modalità in cui ogni singola istruzione SQL è una transazione.
		\end{itemize}
		\item \textbf{Esempio Transazione}:
		\begin{minted}{sql}
START TRANSACTION;
UPDATE BANKACCOUNT SET Balance = Balance - 10
WHERE AccountNumber = 42177;
UPDATE BANKACCOUNT SET Balance = Balance + 10
WHERE AccountNumber = 12202;
-- Se tutto va bene:
COMMIT WORK;
-- Se c'è un errore (da verificare in logica applicativa):
-- ROLLBACK WORK;
		\end{minted}
	\end{itemize}
	
\end{document}
