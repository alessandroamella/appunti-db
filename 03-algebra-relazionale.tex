%%%%%%%%%%%%%%%%%%%%%%%%%%%%%%%%%%%%%%%%%%%%%%%%%%%%%%%%%%%%%%%%%%%%%%%%%%%%%%%
% PREAMBOLO COMUNE PER APPUNTI (Stile Scuro)
%
% Questo file contiene tutte le impostazioni e i pacchetti comuni.
% NON contiene \begin{document} o \end{document}.
%
% Istruzioni per la compilazione del file principale:
% pdflatex -shell-escape nomefile_principale.tex
%%%%%%%%%%%%%%%%%%%%%%%%%%%%%%%%%%%%%%%%%%%%%%%%%%%%%%%%%%%%%%%%%%%%%%%%%%%%%%%

\documentclass{article}

% --- Encoding e lingua ---
\usepackage[utf8]{inputenc}
\usepackage[italian]{babel}

% --- Margini e layout ---
\usepackage{geometry}
\geometry{a4paper, margin=1in}

% --- Font sans-serif (come Helvetica) ---
\usepackage[scaled]{helvet}
\renewcommand{\familydefault}{\sfdefault}
\usepackage[T1]{fontenc}

% --- Matematica ---
\usepackage{amsmath}
\usepackage{amssymb}

% --- Liste personalizzate ---
\usepackage{enumitem}
% \setlist{nosep}

% --- Immagini e Grafica ---
\usepackage{float}
% \usepackage{graphicx}
\usepackage{tikz}
\usetikzlibrary{shapes.geometric, positioning, calc}

% --- Tabelle Avanzate ---
\usepackage{array}
\usepackage{booktabs}
\usepackage{longtable}

% --- Hyperlink e Metadati PDF ---
\usepackage{hyperref}
\hypersetup{
    colorlinks=true,
    linkcolor=white,
    filecolor=magenta,
    urlcolor=cyan,
    citecolor=green,
    % pdftitle, pdfauthor, ecc. verranno impostati nel file principale
    pdfpagemode=FullScreen,
    bookmarksopen=true,
    bookmarksnumbered=true
}

% --- Colori e Sfondo Nero ---
\usepackage{xcolor}
\pagecolor{black}
\color{white}

% --- Evidenziazione del Codice ---
\usepackage{minted}
\setminted{
    frame=lines,
    framesep=2mm,
    fontsize=\small,
    breaklines=true,
    style=monokai,
    bgcolor=black!80
}
\usemintedstyle{monokai}

% --- Comandi Personalizzati (Opzionali) ---
% \newcommand{\Rel}[1]{\textit{#1}}
% ... (altri comandi se necessario) ...

% NESSUN \title, \author, \date, \begin{document} o \end{document} QUI


% --- Hyperlink ---
\usepackage{hyperref}
\hypersetup{
	colorlinks=true,
	linkcolor=white,
	filecolor=magenta,
	urlcolor=cyan,
	pdftitle={Algebra Relazionale e Calcolo Relazionale},
	pdfauthor={Alessandro Amella},
	pdfpagemode=FullScreen,
}


% --- Comandi personalizzati per algebra relazionale ---
\newcommand{\Rel}[1]{\textit{#1}} % Per i nomi delle relazioni
\newcommand{\Attr}[1]{\textsf{#1}} % Per i nomi degli attributi

\newcommand{\myunion}{\cup}
\newcommand{\myintersection}{\cap}
\newcommand{\mydifference}{-}
\newcommand{\myrename}[2]{\rho_{#1}(#2)}
\newcommand{\myselectop}[2]{\sigma_{#1}(#2)}
\newcommand{\myproject}[2]{\pi_{#1}(#2)}
\newcommand{\mycartesian}{\times}
\newcommand{\mynaturaljoin}{\bowtie} % Usare \Join da amssymb se disponibile e preferito
\newcommand{\mythetajoin}[3]{#1 \bowtie_{#2} #3} % R1 \bowtie_cond R2

% --- Comandi personalizzati per logica ---
\newcommand{\mylandop}{\wedge}
\newcommand{\myvel}{\vee}
\newcommand{\mynegop}{\neg}
\newcommand{\myforallop}{\forall}
\newcommand{\myexistsop}{\exists}

% --- Join esterni (outer join) ---
% Definizione standard per i join esterni
\def\ojoin{\setbox0=\hbox{$\mynaturaljoin$}%
	\rule[-.02ex]{.25em}{.4pt}\llap{\rule[\ht0]{.25em}{.4pt}}}
\newcommand{\myleftouterjoin}{\mathbin{\ojoin\mkern-5.8mu\mynaturaljoin}}
\newcommand{\myrightouterjoin}{\mathbin{\mynaturaljoin\mkern-5.8mu\ojoin}}
\newcommand{\myfullouterjoin}{\mathbin{\ojoin\mkern-5.8mu\mynaturaljoin\mkern-5.8mu\ojoin}}


% --- Titolo ---
\title{Algebra Relazionale e Calcolo Relazionale}
\author{Basato sulle slide del Prof. Danilo Montesi}
\date{\today}

\begin{document}
	
	\maketitle
	\tableofcontents
	\newpage
	
	\section{Introduzione ai Linguaggi per Database}
	
	I linguaggi per database si dividono principalmente in due categorie:
	\begin{itemize}
		\item \textbf{DDL (Data Definition Language):} Utilizzato per definire e modificare lo schema del database (es. creare tabelle, definire attributi e tipi).
		\item \textbf{DML (Data Manipulation Language):} Utilizzato per operare sui dati. Si suddivide ulteriormente in:
		\begin{itemize}
			\item \textbf{Istruzioni di Query:} Per estrarre dati di interesse.
			\item \textbf{Istruzioni di Aggiornamento:} Per inserire nuovi dati o modificare quelli esistenti.
		\end{itemize}
	\end{itemize}
	
	I linguaggi di query possono essere:
	\begin{itemize}
		\item \textbf{Dichiarativi:} Specificano \textbf{cosa} si vuole ottenere, le proprietà del risultato. L'utente non si preoccupa di come il database recupererà i dati. SQL è prevalentemente dichiarativo.
		\item \textbf{Imperativi/Procedurali:} Specificano \textbf{come} il risultato deve essere ottenuto, descrivendo una sequenza di operazioni.
	\end{itemize}
	
	Panoramica dei linguaggi trattati:
	\begin{itemize}
		\item \textbf{Algebra Relazionale:} Procedurale (fondamento teorico).
		\item \textbf{Calcolo Relazionale:} Dichiarativo (fondamento teorico).
		\item \textbf{SQL (Structured Query Language):} Parzialmente dichiarativo (ampiamente implementato).
		\item \textbf{QBE (Query by Example):} Dichiarativo (implementato in alcuni sistemi).
	\end{itemize}
	
	\section{Algebra Relazionale}
	
	L'algebra relazionale è un linguaggio di query formale, procedurale, che definisce un insieme di operatori che agiscono su relazioni (tabelle) per produrre nuove relazioni come risultato. Gli operatori possono essere composti.
	
	\subsection{Operatori dell'Algebra Relazionale}
	\begin{itemize}
		\item Operatori insiemistici: \textbf{Unione ($\myunion$)}, \textbf{Intersezione ($\myintersection$)}, \textbf{Differenza ($\mydifference$)}.
		\item \textbf{Ridenominazione ($\myrename{}$)}.
		\item \textbf{Selezione ($\myselectop{}{}$)}.
		\item \textbf{Proiezione ($\myproject{}{}$)}.
		\item \textbf{Join ($\mynaturaljoin$)}: natural join, prodotto cartesiano ($\mycartesian$), theta-join ($\bowtie_{\theta}$).
	\end{itemize}
	
	\subsection{Operatori Insiemistici}
	Le relazioni sono insiemi di tuple. Questi operatori funzionano come le loro controparti nella teoria degli insiemi, ma richiedono che le relazioni coinvolte abbiano lo \textbf{stesso schema} (stessi nomi di attributi, nello stesso ordine e con tipi compatibili).
	\begin{itemize}
		\item \textbf{Unione ($\Rel{R1} \myunion \Rel{R2}$):} Restituisce una relazione contenente tutte le tuple che sono in $\Rel{R1}$, in $\Rel{R2}$, o in entrambe. Le tuple duplicate vengono eliminate.
		\item \textbf{Intersezione ($\Rel{R1} \myintersection \Rel{R2}$):} Restituisce una relazione contenente solo le tuple che sono presenti sia in $\Rel{R1}$ sia in $\Rel{R2}$.
		\item \textbf{Differenza ($\Rel{R1} \mydifference \Rel{R2}$):} Restituisce una relazione contenente le tuple che sono in $\Rel{R1}$ ma non in $\Rel{R2}$.
	\end{itemize}
	\textbf{Esempio pratico:} Se hai due tabelle, $\Rel{StudentiMagistrale}$ e $\Rel{StudentiDottorato}$, entrambe con colonne $\Attr{IDStudente}$ e $\Attr{Nome}$, puoi fare l'unione per ottenere una lista unica di tutti gli studenti post-laurea. Se le colonne avessero nomi diversi (es. $\Attr{Matricola}$ vs $\Attr{IDStudente}$), dovresti prima usare l'operatore di ridenominazione.
	
	\subsection{Ridenominazione ($\myrename{nuovo \leftarrow vecchio}{R}$)}
	Operatore unario che cambia i nomi degli attributi o della relazione stessa, senza alterare i dati.
	\begin{itemize}
		\item $\myrename{\Attr{NuovoNomeAttr} \leftarrow \Attr{VecchioNomeAttr}}{(\Rel{R})}$: Rinomina l'attributo $\Attr{VecchioNomeAttr}$ in $\Attr{NuovoNomeAttr}$ nella relazione $\Rel{R}$.
		\item $\myrename{\Rel{NuovoNomeRel}}{(\Rel{R})}$: Rinomina la relazione $\Rel{R}$ in $\Rel{NuovoNomeRel}$.
		\item $\myrename{\Rel{NuovoNomeRel}(\Attr{A1}, \Attr{A2}, \dots)}{(\Rel{R})}$: Rinomina la relazione e i suoi attributi.
	\end{itemize}
	\textbf{Esempio pratico:} Se hai una tabella $\Rel{Impiegati}$ con una colonna $\Attr{stip}$ e vuoi renderla più chiara come $\Attr{StipendioAnnuale}$, useresti $\myrename{\Attr{StipendioAnnuale} \leftarrow \Attr{stip}}{(\Rel{Impiegati})}$.
	
	\subsection{Selezione ($\myselectop{predicato}{R}$)}
	Operatore unario che restituisce un sottoinsieme delle tuple di una relazione $\Rel{R}$ che soddisfano un dato \textit{predicato} (condizione).
	Lo schema del risultato è identico a quello di $\Rel{R}$.
	\textbf{Esempio pratico (SQL):}
	$\myselectop{\Attr{Eta} > 30 \mylandop \Attr{Dipartimento} = \text{'IT'}}{(\Rel{Impiegati})}$
	è equivalente a:
	\begin{minted}{sql}
SELECT *
FROM Impiegati
WHERE Eta > 30 AND Dipartimento = 'IT';
	\end{minted}
	
	\subsection{Proiezione ($\myproject{lista\_attributi}{R}$)}
	Operatore unario che restituisce una nuova relazione contenente solo gli attributi specificati nella \textit{lista\_attributi} dalla relazione $\Rel{R}$.
	Le tuple duplicate nel risultato vengono eliminate (poiché le relazioni sono insiemi).
	\textbf{Esempio pratico (SQL):}
	$\myproject{\Attr{Nome}, \Attr{Cognome}}{(\Rel{Impiegati})}$
	è equivalente a:
	\begin{minted}{sql}
SELECT DISTINCT Nome, Cognome
FROM Impiegati;
	\end{minted}
	Nota l'uso di `DISTINCT` in SQL per replicare il comportamento insiemistico della proiezione.
	
	\subsection{Combinazione di Selezione e Proiezione}
	Questi operatori sono spesso usati insieme per estrarre dati specifici.
	\textbf{Esempio pratico (SQL):} Trovare nome e cognome degli impiegati nel dipartimento 'Vendite' con stipendio superiore a 50000.
	Algebra: $\myproject{\Attr{Nome}, \Attr{Cognome}}{(\myselectop{\Attr{Dipartimento} = \text{'Vendite'} \mylandop \Attr{Stipendio} > 50000}{(\Rel{Impiegati})})}$
	SQL:
	\begin{minted}{sql}
SELECT DISTINCT Nome, Cognome
FROM Impiegati
WHERE Dipartimento = 'Vendite' AND Stipendio > 50000;
	\end{minted}
	
	\subsection{Join}
	Il join è un operatore fondamentale per combinare informazioni da due o più relazioni.
	\begin{itemize}
		\item \textbf{Prodotto Cartesiano ($\Rel{R1} \mycartesian \Rel{R2}$):}
		Combina ogni tupla di $\Rel{R1}$ con ogni tupla di $\Rel{R2}$. Il numero di tuple risultanti è $|\Rel{R1}| \times |\Rel{R2}|$. Lo schema è la concatenazione degli schemi di $\Rel{R1}$ e $\Rel{R2}$.
		In SQL, è spesso scritto come `FROM R1, R2` (sintassi più vecchia) o `FROM R1 CROSS JOIN R2`. Di solito è seguito da una selezione per filtrare le combinazioni significative.
		
		\item \textbf{Theta-Join ($\mythetajoin{\Rel{R1}}{condizione}{\Rel{R2}}$):}
		È un prodotto cartesiano seguito da una selezione. La \textit{condizione} è un predicato che coinvolge attributi di $\Rel{R1}$ e $\Rel{R2}$.
		Sintassi formale: $\myselectop{condizione}{(\Rel{R1} \mycartesian \Rel{R2})}$.
		\textbf{Esempio pratico (SQL):}
		$\mythetajoin{\Rel{Impiegati}}{\Attr{Impiegati.IDDip} = \Attr{Dipartimenti.ID}}{\Rel{Dipartimenti}}$
		è equivalente a:
		\begin{minted}{sql}
SELECT *
FROM Impiegati, Dipartimenti -- o Impiegati JOIN Dipartimenti
WHERE Impiegati.IDDip = Dipartimenti.ID;
		\end{minted}
		
		\item \textbf{Equi-Join:} Un Theta-Join in cui la condizione contiene solo confronti di uguaglianza ($=$). L'esempio precedente è un equi-join.
		
		\item \textbf{Natural Join ($\Rel{R1} \mynaturaljoin \Rel{R2}$):}
		Un tipo speciale di equi-join. Le relazioni vengono combinate basandosi sull'uguaglianza dei valori degli attributi che hanno lo \textbf{stesso nome} in entrambe le relazioni. Gli attributi comuni appaiono una sola volta nel risultato.
		\textbf{Esempio pratico (SQL):} Se $\Rel{Impiegati}$ e $\Rel{Assegnazioni}$ hanno entrambe una colonna $\Attr{IDProgetto}$.
		$\Rel{Impiegati} \mynaturaljoin \Rel{Assegnazioni}$
		è equivalente a:
		\begin{minted}{sql}
SELECT *
FROM Impiegati NATURAL JOIN Assegnazioni;
		\end{minted}
		\textit{Attenzione:} Il Natural Join può essere pericoloso se ci sono attributi con lo stesso nome ma significato diverso, o se si aggiungono/rimuovono colonne. È spesso preferibile usare join espliciti con clausola `ON`.
	\end{itemize}
	
	\subsection{Outer Join (Join Esterni)}
	I join visti finora (inner join) scartano le tuple che non trovano una corrispondenza nell'altra relazione. Gli outer join includono queste tuple, riempiendo con `NULL` gli attributi mancanti.
	\begin{itemize}
		\item \textbf{Left Outer Join ($\Rel{R1} \myleftouterjoin \Rel{R2}$):} Mantiene tutte le tuple di $\Rel{R1}$. Se una tupla di $\Rel{R1}$ non ha corrispondenze in $\Rel{R2}$, viene inclusa nel risultato con valori `NULL` per gli attributi di $\Rel{R2}$.
		SQL: `R1 LEFT OUTER JOIN R2 ON condizione`
		\item \textbf{Right Outer Join ($\Rel{R1} \myrightouterjoin \Rel{R2}$):} Mantiene tutte le tuple di $\Rel{R2}$. Simmetrico al left outer join.
		SQL: `R1 RIGHT OUTER JOIN R2 ON condizione`
		\item \textbf{Full Outer Join ($\Rel{R1} \myfullouterjoin \Rel{R2}$):} Mantiene tutte le tuple di entrambe le relazioni. Se non c'è corrispondenza, i campi dell'altra relazione sono `NULL`.
		SQL: `R1 FULL OUTER JOIN R2 ON condizione`
	\end{itemize}
	\textbf{Esempio pratico:} Trovare tutti gli impiegati e, se assegnati a un dipartimento, il nome del dipartimento. Se un impiegato non ha dipartimento, vogliamo comunque vederlo.
	$\Rel{Impiegati} \myleftouterjoin \Rel{Dipartimenti}$ (assumendo un join su $\Attr{IDDip}$)
	SQL:
	\begin{minted}{sql}
SELECT Impiegati.Nome, Dipartimenti.NomeDip
FROM Impiegati
LEFT OUTER JOIN Dipartimenti ON Impiegati.IDDip = Dipartimenti.ID;
	\end{minted}
	
	\subsection{Espressioni Equivalenti e Ottimizzazione}
	Esistono diverse espressioni algebriche che producono lo stesso risultato. I DBMS (Database Management Systems) utilizzano regole di equivalenza per trasformare una query in una forma equivalente ma più efficiente da eseguire.
	Ad esempio, "spingere" le selezioni il più presto possibile (`pushdown selection`) riduce la dimensione delle relazioni intermedie, velocizzando i join successivi.
	\textbf{Esempio:}
	$\myselectop{\Attr{Stipendio}>100K}{(\Rel{Impiegati} \mynaturaljoin \Rel{Dipartimenti})}$
	potrebbe essere più efficiente se riscritta come:
	$(\myselectop{\Attr{Stipendio}>100K}{\Rel{Impiegati}}) \mynaturaljoin \Rel{Dipartimenti}$
	(se $\Attr{Stipendio}$ è solo in $\Rel{Impiegati}$). Il DBMS fa queste ottimizzazioni automaticamente.
	
	\subsection{Selezione con Valori NULL}
	I valori `NULL` rappresentano l'assenza di un valore o un valore sconosciuto.
	Nei predicati di selezione:
	\begin{itemize}
		\item Un confronto con `NULL` (es. $\Attr{Eta} > \text{NULL}$ o $\Attr{Stipendio} = \text{NULL}$) restituisce `UNKNOWN`.
		\item L'operatore $\sigma$ seleziona solo le tuple per cui il predicato è `TRUE`.
		\item Per testare esplicitamente i `NULL`, si usano i predicati $\Attr{Attributo} \text{ IS NULL}$ e $\Attr{Attributo} \text{ IS NOT NULL}$.
	\end{itemize}
	\textbf{Esempio pratico (SQL):} Trovare gli impiegati senza un numero di telefono assegnato.
	$\myselectop{\Attr{Telefono} \text{ IS NULL}}{(\Rel{Impiegati})}$
	SQL:
	\begin{minted}{sql}
SELECT * FROM Impiegati WHERE Telefono IS NULL;
	\end{minted}
	
	\section{Views (Viste)}
	Una vista è una tabella virtuale il cui contenuto è definito da una query sull'algebra relazionale (o SQL). Non memorizza dati propriamente, ma li deriva dalle tabelle base al momento della query sulla vista.
	\begin{itemize}
		\item \textbf{Tabelle Base:} Tabelle che contengono fisicamente i dati.
		\item \textbf{Tabelle Derivate (Viste):} Definite da query.
	\end{itemize}
	Le viste sono utili per:
	\begin{itemize}
		\item \textbf{Schema Esterno:} Fornire diverse rappresentazioni dei dati a utenti diversi, semplificando la complessità e implementando la sicurezza (mostrando solo dati pertinenti).
		\item \textbf{Strumento di Programmazione:} Semplificare query complesse riutilizzando sotto-espressioni comuni, o per mantenere la compatibilità con applicazioni esistenti quando lo schema delle tabelle base cambia.
	\end{itemize}
	\textbf{Esempio pratico (SQL):} Creare una vista che mostra solo gli impiegati del dipartimento IT.
	Algebra: $\Rel{ImpiegatiIT} := \myselectop{\Attr{Dipartimento} = \text{'IT'}}{(\Rel{Impiegati})}$
	SQL:
	\begin{minted}{sql}
CREATE VIEW ImpiegatiIT AS
SELECT *
FROM Impiegati
WHERE Dipartimento = 'IT';

-- Successivamente si può interrogare la vista:
SELECT Nome, Cognome FROM ImpiegatiIT WHERE Stipendio > 60000;
	\end{minted}
	Il DBMS traduce la query sulla vista in una query sulle tabelle base (es. $\myselectop{\Attr{Stipendio} > 60000}{(\myselectop{\Attr{Dipartimento} = \text{'IT'}}{(\Rel{Impiegati})})}$ ).
	
	\section{Calcolo Relazionale}
	Il calcolo relazionale è un linguaggio di query formale, \textbf{dichiarativo}, basato sulla logica dei predicati del primo ordine. Specifica \textit{cosa} si vuole ottenere, non \textit{come}.
	Esistono due forme principali:
	\begin{itemize}
		\item \textbf{Domain Relational Calculus (DRC):} Le variabili assumono valori dai domini degli attributi.
		\item \textbf{Tuple Relational Calculus (TRC):} Le variabili rappresentano tuple di relazioni.
	\end{itemize}
	
	\subsection{Domain Relational Calculus (DRC)}
	Una query DRC ha la forma:
	$\{ \Attr{A}_1:x_1, \dots, \Attr{A}_k:x_k \mid \text{Formula}(x_1, \dots, x_k) \}$
	dove:
	\begin{itemize}
		\item $\Attr{A}_1:x_1, \dots, \Attr{A}_k:x_k$ è la \textbf{target list}: specifica gli attributi del risultato e le variabili che ne conterranno i valori.
		\item $x_1, \dots, x_k$ sono variabili che variano sui domini dei rispettivi attributi.
		\item $\text{Formula}(x_1, \dots, x_k)$ è una formula della logica del primo ordine che usa:
		\begin{itemize}
			\item Predicati corrispondenti alle relazioni nel database (es. $\Rel{IMPIEGATO}(\Attr{Num}:m, \Attr{Nome}:n, \dots)$).
			\item Operatori di confronto ($=, >, <, \dots$).
			\item Operatori logici ($\mylandop$ AND, $\myvel$ OR, $\mynegop$ NOT).
			\item Quantificatori ($\myforallop$ per tutti, $\myexistsop$ esiste).
		\end{itemize}
	\end{itemize}
	Il risultato è l'insieme di tuple $(\Attr{A}_1:v_1, \dots, \Attr{A}_k:v_k)$ tali che, quando le variabili $x_i$ assumono i valori $v_i$, la Formula è vera.
	
	\textbf{Esempio DRC:} Trovare numero, nome, età e salario degli impiegati che guadagnano più di 40.
	\begin{verbatim}
		{ Numero:m, Nome:n, Eta:a, Salario:w |
			IMPIEGATO(Numero:m, Nome:n, Eta:a, Salario:w)  w > 40 }
	\end{verbatim}
	Se vogliamo solo nome ed età:
	\begin{verbatim}
		{ Nome:n, Eta:a |
			existsop w (IMPIEGATO(Numero:m, Nome:n, Eta:a, Salario:w)  w > 40) }
	\end{verbatim}
	(Il `Numero:m` nella seconda query è una variabile libera nella formula `IMPIEGATO`, ma non nella target list, quindi la sua esistenza è implicitamente richiesta. Le slide mostrano `EMPLOYEE(Number:m, Name:n, Age:a, Wage:w)` anche quando `m` non è nella target list; questo significa che deve esistere una tupla con qualche `m` che soddisfi il resto. Per essere più precisi, si quantificherebbero le variabili non nella target list).
	
	\subsection{Tuple Relational Calculus (TRC) with Range Declarations}
	Una query TRC ha la forma:
	$\{ \text{TargetList} \mid \text{RangeList} \mid \text{Formula} \}$
	dove:
	\begin{itemize}
		\item $\text{TargetList}$: Specifica gli attributi da restituire, spesso nella forma $t.\Attr{A}$ (attributo $\Attr{A}$ della tupla $t$).
		\item $\text{RangeList}$: Dichiara le variabili di tupla e le relazioni a cui appartengono (es. $e(\Rel{IMPIEGATO})$, $s(\Rel{SUPERVISORE})$).
		\item $\text{Formula}$: Una condizione sulle variabili di tupla dichiarate.
	\end{itemize}
	Le variabili variano sull'insieme delle tuple della relazione specificata.
	
	\textbf{Esempio TRC:} Trovare tutte le informazioni sugli impiegati che guadagnano più di 40.
	\begin{verbatim}
		{ e.* | e(IMPIEGATO) | e.Salario > 40 }
	\end{verbatim}
	Trovare nome ed età degli impiegati che guadagnano più di 40:
	\begin{verbatim}
		{ e.Nome, e.Eta | e(IMPIEGATO) | e.Salario > 40 }
	\end{verbatim}
	Il TRC è spesso considerato più vicino a come si pensa in SQL, poiché si ragiona in termini di "tuple che soddisfano certe condizioni".
	
	\subsection{Equivalenza tra Algebra e Calcolo}
	Per le query "safe" (che non producono risultati infiniti e dipendono solo dai dati nel database), l'Algebra Relazionale, il Domain Relational Calculus (safe) e il Tuple Relational Calculus (safe) sono \textbf{espressivamente equivalenti}. Ciò significa che qualsiasi query esprimibile in uno di questi linguaggi può essere espressa anche negli altri. Questo è un risultato teorico importante (Teorema di Codd).
	
	\section{Limiti dell'Algebra e del Calcolo Relazionale Standard}
	Nonostante la loro potenza, l'algebra e il calcolo relazionale standard hanno dei limiti:
	\begin{itemize}
		\item \textbf{No Calcoli Aritmetici/Nuovi Valori:} Non possono calcolare nuovi valori (es. stipendio + bonus) o eseguire aggregazioni (somma, media, conteggio). SQL estende queste capacità con funzioni aritmetiche e di aggregazione (`SUM()`, `AVG()`, `COUNT()`, `GROUP BY`).
		\item \textbf{No Chiusura Transitiva (Recursion):} Non possono esprimere query ricorsive, come trovare tutti i superiori di un impiegato (il capo, il capo del capo, ecc.) o tutte le tratte aeree possibili tra due città (dirette e indirette). Questo richiederebbe un numero potenzialmente illimitato di join.
	\end{itemize}
	
	\section{Datalog}
	Datalog è un linguaggio di query e programmazione logica orientato ai database, che supera alcuni limiti di RA/RC, in particolare per le query ricorsive. È un sottoinsieme di Prolog.
	Concetti chiave:
	\begin{itemize}
		\item \textbf{Predicati Estensionali (EDB - Extensional Database):} Corrispondono alle relazioni base del database (fatti).
		\item \textbf{Predicati Intensionali (IDB - Intensional Database):} Corrispondono a viste o relazioni derivate, definite tramite \textbf{regole}.
		\item \textbf{Regole:} Hanno la forma `testa :- corpo.` (o `testa <- corpo` nelle slide).
		Significa: "la \textit{testa} è vera se il \textit{corpo} è vero".
		La testa è un singolo predicato intensionale. Il corpo è una congiunzione (AND) di predicati (estensionali o intensionali) e condizioni.
		\item \textbf{Query:} Indicate con un prefisso ? davanti a un predicato.
	\end{itemize}
	
	\textbf{Esempio Datalog (non ricorsivo):} Trovare i capi degli impiegati che guadagnano più di 40.
	Relazioni EDB: $\Rel{IMPIEGATO}(\Attr{Num}, \Attr{Nome}, \Attr{Eta}, \Attr{Salario})$, $\Rel{SUPERVISORE}(\Attr{Capo}, \Attr{Impiegato})$
	\begin{minted}{prolog}
% Predicato intensionale: IMPIEGATO_RICCO
IMPIEGATO_RICCO(Num, Nome, Eta, Salario) :- IMPIEGATO(Num, Nome, Eta, Salario), Salario > 40.

% Predicato intensionale: CAPO_DI_IMPIEGATO_RICCO
CAPO_DI_IMPIEGATO_RICCO(NumCapo) :- IMPIEGATO_RICCO(NumImp, _, _, _),
SUPERVISORE(NumCapo, NumImp).

% Query
? CAPO_DI_IMPIEGATO_RICCO(X).
	\end{minted}
	
	\textbf{Esempio Datalog (Ricorsione - Chiusura Transitiva):} Trovare tutti i superiori (diretti e indiretti) di un impiegato.
	Relazione EDB: $\Rel{CAPO\_DIRETTO}(\Attr{Superiore}, \Attr{Subordinato})$
	\begin{minted}{prolog}
% Caso base: un capo diretto e' un superiore
SUPERIORE(X, Y) :- CAPO_DIRETTO(X, Y).

% Caso ricorsivo: il superiore di un mio superiore e' anche mio superiore
SUPERIORE(X, Y) :- CAPO_DIRETTO(X, Z), SUPERIORE(Z, Y).

% Query: trovare tutti i superiori di 'Rossi' (assumendo che 'Rossi' sia un ID)
? SUPERIORE(Capo, 'Rossi').
	\end{minted}
	Datalog, grazie alla sua capacità di esprimere la ricorsione, è più potente dell'algebra e del calcolo relazionale standard. Le estensioni ricorsive di SQL (come `WITH RECURSIVE`) sono ispirate a Datalog.
	
	\section{Conclusioni}
	L'Algebra Relazionale e il Calcolo Relazionale forniscono le fondamenta teoriche per i linguaggi di query dei database relazionali come SQL.
	\begin{itemize}
		\item L'\textbf{Algebra Relazionale} è procedurale e definisce come costruire il risultato passo dopo passo. È cruciale per l'implementazione interna e l'ottimizzazione delle query nei DBMS.
		\item Il \textbf{Calcolo Relazionale} è dichiarativo, permettendo di specificare le proprietà del risultato desiderato senza dettagliarne il processo di ottenimento.
		\item Entrambi (nelle loro forme "safe") hanno lo stesso potere espressivo ma non possono gestire calcoli complessi o ricorsione.
		\item \textbf{Datalog} estende questi concetti introducendo la ricorsione, aumentando il potere espressivo.
	\end{itemize}
	Comprendere questi modelli teorici aiuta a capire meglio il funzionamento e le potenzialità di SQL e dei sistemi di gestione di database moderni.
	
\end{document}
